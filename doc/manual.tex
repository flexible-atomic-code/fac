\documentclass[twoside,letterpaper]{refrep}
\usepackage{makeidx}
\usepackage{natbib}
\usepackage{xspace}
\usepackage{graphicx}
\usepackage{verbatim}
\usepackage{threeparttable}
\usepackage[pdftex,bookmarks,colorlinks=false,pdfborder=0]{hyperref}

\settextfraction{1.0}

\newcommand{\facversion}{{1.1.5}\xspace}
\newcommand{\opt}[1]{
  {\textnormal{[}}{#1}\hspace{0.5mm}{\textnormal{]}}}
\newcommand{\var}[1]{\textit{#1}}
\newcommand{\key}[1]{\texttt{#1}}
\newcommand{\mod}[1]{\texttt{#1}}
\newcommand{\atom}[1]{\textbf{\texttt{#1}}}

\newcommand{\threej}[6]{\ensuremath{\left({#1\atop #4}{#2\atop #5}
{#3\atop #6}\right)}}
\newcommand{\sixj}[6]{\ensuremath{\left\{{#1\atop #4}{#2\atop #5}
{#3\atop #6}\right\}}}

\newenvironment{ttscript}[1]{
	\begin{list}{}{
	\settowidth{\labelwidth}{\texttt{#1}}
	\setlength{\leftmargin}{\labelwidth}
	\addtolength{\leftmargin}{\labelsep}
	\setlength{\parsep}{0.5ex plus0.2ex minus0.2ex}
	\setlength{\itemsep}{0.3ex}
	\renewcommand{\makelabel}[1]{\texttt{##1\hfill}}}}
	{\end{list}}
\newenvironment{dbdesc}{\textbf{Field Description:} \begin{list}
	{:}{\setlength{\labelwidth}{2in}
	   \setlength{\leftmargin}{2in}
	   \setlength{\labelsep}{0.1in}
	   \setlength{\rightmargin}{0.2in}}}
	{\end{list}}
\newenvironment{fundesc}[2]{
	\begin{center}
	\begin{minipage}{\textwidth}
	\subsubsection*{\key{\textbf{#1}}(\var{#2}):}
	\index{#1}
	\addcontentsline{toc}{subsubsection}{#1}}
	{\end{minipage}\end{center}}
\newenvironment{vardesc}[1]{
	\begin{center}
	\begin{minipage}{\textwidth}
	\index{#1}
	\addcontentsline{toc}{subsubsection}{#1}
	\key{\textbf{#1}}:\\}
	{\end{minipage}\end{center}}

\newcounter{faq}[section]
\newcommand{\faq}[2]{\stepcounter{faq}
	\begin{minipage}{\textwidth}
	\textbf{Q\arabic{faq}: #1?}\\#2
	\end{minipage}}

\setcounter{tocdepth}{3}

\makeindex

\begin{document}

\title{FAC \facversion Manual}
\author{M. F. Gu\thanks{mfgu@ssl.berkeley.edu}}

\date{}

\maketitle

\tableofcontents

\chapter{Overview}
\label{cha:overview}

\section{What Is FAC}
FAC stands for The Flexible Atomic Code. It is an
integrated software package to calculate various atomic radiative and
collisional processes, including energy levels, radiative transition rates,
collisional excitation and
ionization by electron impact, photoionization, autoionization, radiative
recombination and dielectronic capture. The package also includes a
collisional radiative model to construct synthetic spectra for plasmas under
different physical conditions.

The atomic structure calculation in FAC is based on the relativistic
configuration interaction with independent particle basis wavefunctions. These
basis wavefunctions are derived from a local central potential, which is
self-consistently determined to represent electronic screening of the nuclear
potential. Relativistic effects are fully taken into account using the Dirac
Coulomb Hamiltonian. Higher order QED effects are included with Breit
interaction in the zero energy limit for the exchanged photon, and hydrogenic
approximations for self-energy and vacuum polarization effects.
Continuum processes are treated
in the distorted-wave (DW) approximation. Systematic application of the
factorization-interpolation method of \citet{barshalom88} makes the present
code highly efficient for large scale calculations. The details of theoretical
background and computational methods are not discussed in this manual,
instead, they are described in a series of papers which are distributed along
with this package and this manual.

FAC is a step forward to bring detailed atomic model accessable to a wide
community of laboratory and astrophysical plasma diagnostics. Its flexible
interface is designed to be useful even for people without a deep
understanding of the underlying atomic theories. It is also powerful enough
for experienced users to explore the effects of algorithmic choices and
different physical approximations.

FAC is freely distributed in the hope that it will be useful. The author makes
every effort to ensure its correctness. However, he does not guarantee its
fitness to any specific purpose. The author is not responsible for any damage
resulting from the use of this program, including failure to obtain or loss of
tenure.

\section{Obtain and Install FAC}
\index{Install}
\label{sec:install}
The latest version of FAC is \facversion. It can be obtained from
\textbf{http://sprg.ssl.berkeley.edu/~mfgu/fac}. I can also send a copy to you
through email. Please request to \textbf{mfgu@ssl.berkeley.edu}. It is being
continously developed at present, so please check regularly to get the newest
version.

Much of the FAC package is written in ANSI C and Fortran 77. It should
therefore work on any platform with a C and Fortran 77 compilers. However,
this is only true to the rather simple command parser that comes with FAC,
referred
to as SFAC. The flexibility of FAC is realized when the Python interface
(PFAC) is used. The numerical subroutines implemented in FAC are exported
through several Python modules. The computation task can therefore be
completed by programming in the scripting language Python. These python
modules are compiled as shared objects, and are dynamically loaded. This
primarily works under ELF systems, such as almost all modern Unix and Linux
systems. It has also been tested to work under Mac OS X and Windows (In the
case of Windows, the Unix API emulation by Cygwin is required, which is
available at \textbf{www.cygwin.com}). To fully utilize the strength of
FAC, it is strongly recommended that Python be installed, which can be obtained
from \textbf{www.python.org}.

Step-by-step instructions for installation can be found in the README file in
the top directory of FAC distribution.

\section{Quick Start}
\label{sec:start}
\subsection{SFAC Interface}
The SFAC interface is basically a stripped down command interpreter modeled
after Python syntax, with the omission of flow control freatures, such as
conditional execution and loops. Therefore simple Python scripts may be
converted to SFAC input files without difficulty. The Python interface
actually contains functions to do the convertion automatically. Through out
this manual, we mainly focus on the more useful Python interface. Most of the
Python fuctions implemented in the extension modules are also available in
SFAC interface with identical calling sequences. To use SFAC, one passes the
input files to the 3 executables \verb|sfac|, \verb|scrm| and \verb|spol| on
the command line such as
\begin{verbatim}
    sfac input.sf
    scrm input.sf
\end{verbatim}
or, one may invoke \verb|sfac| and \verb|scrm| without arguments, in which
case, they read from \verb|stdin| for inputs, where commands are interpreted
line by line. The program \verb|sfac| handles atomic calculations,
\verb|scrm| is used to construct collisional radiative spectral models, and
\verb|spol| is used to calculate line polarizations due to collisional
excitation.

\subsection{PFAC Interface}
To use the PFAC interface, one needs to be familiar with the basics of Python
scripting language. Python has excellent documentations that come with the standard
distribution. It is an extremely well designed language to learn, and to use.

Perhaps, the quickest way to get familiar with FAC is to inspect the simple
demo scripts in the \verb|demo/| directory in FAC distribution. There
are individual scripts and their SFAC conterparts demonstrating the
calculation of energy levels, radiative transition rates, collisional
excitation and ionization cross sections, radiative recombination cross
sections and autoionization rates. There is also a more advanced example for
the calculation of iron L-shell atomic data, and their application in the
collisional radiative model.

In this section, we look into the details of one of these scripts,
\verb|demo/structure/fe17_structure.py| for the calculation of Ne-like iron
energy levels and radiative transition rates between $n = 2$ and $n = 3$
complexes. The following is a duplication of that script.
\begin{verbatim}
 1: from pfac import fac

 2: fac.SetAtom('Fe')
 3: # 1s shell is closed
 4: fac.Closed('1s')
 5: fac.Config('2*8', group = 'n2')
 6: fac.Config('2*7 3*1', group = 'n3')

 7: # Self-consistent iteration for optimized central potential
 8: fac.ConfigEnergy(0)
# the configurations passed to OptimizeRadial should always
# be one or two of the lowest lying ones. If you need more highly
# excited levels, such as n=4, 5, 6, ..., do not put them into
# OptimizeRadial.
 9: fac.OptimizeRadial(['n2'])
10: fac.ConfigEnergy(1)
11: fac.Structure('ne.lev.b', ['n2', 'n3'])
12: fac.MemENTable('ne.lev.b')
13: fac.PrintTable('ne.lev.b', 'ne.lev', 1)

14: fac.TransitionTable('ne.tr.b', ['n2'], ['n3'])
15: fac.PrintTable('ne.tr.b', 'ne.tr', 1)
\end{verbatim}
Line numbers are added for easy reference, they are not part of the script.
As is evident from the above list, all functions implemented in the FAC
extension modules have a naming convention of concatenated capitalized words.
Line 1 imports the extension module \verb|fac| from the package \verb|pfac|.
Alternatively, one could have used
\begin{verbatim}
    from pfac.fac import *
\end{verbatim}
then, all module qualifiers \verb|fac.| in the following lines can be omitted.
Line 2 set the atomic element to be iron. Line 3 is a comment, which starts
with a \verb|#|. Line 4--6 specifies the electronic configurations to be
included in the calculation. The closed shells specified by the function
\verb|Closed| must be inactive in this calculation. In the \verb|Config|
functions, \verb|2*8| stands for an $n = 2$ complexes with 8 electrons, while
\verb|2*7 3*1| stands for all configurations resulting from excitation of one
electron from $n = 2$ to $n = 3$. For more possibilities in the specification
of electronic configurations, one is referred to Chapter \ref{cha:function}.
Line 8--10 carries out a Dirac-Fock-Slater self-consistent calculation to
derive a local central potential which represents the electronic screening of
the nuclear potential. In this calculation, the potential is optimized to the
average electron clouds of configurations \verb|n2| and \verb|n3|, since in
FAC, all atomic processes are treated with basis wavefunctions generated from
a single potential. This
results in the potential to be less optimized for \verb|n2| and \verb|n3|
individually. Lines 8 and 10 are used to make a crude correction to the
resulting energy levels due to this effect. The first call to
\verb|ConfigEnergy(0)| will make individual optimization to all configuration
groups. The average energy of each configuration group with these indivudually
optimized potential is then calculated and stored. The
second call to \verb|ConfigEnergy(1)| will then recalculate the average energy
of configuration groups under the potential taking into account all
configuration groups. The difference between the two represents the effect of
a less optimized potential, and are used to adjust the final energy levels. If
this procedure is not needed, one can omit line 8 and 10 in this script. Line
11 sets up the Hamiltonian matrix for levels in $n = 2$ and $n = 3$ complexes,
diagonalize it, and saves to the energy level information in the binary file
\verb|ne.lev.b|. Line 12 builds an in-memory table of energy levels, which is
used to convert the binary files to their ASCII counterparts in verbose mode,
such as done in Line 13, which converts \verb|ne.lev.b| to \verb|ne.lev| (the
last argument to \verb|PrintTable| indicates it be done in verbose mode). For
the conversion in simple mode (the last argument is 0), the in-memory table is
not needed, and Line 12 may be omitted. For the difference between the verbose
and simple ASCII files, see Chapter \ref{cha:format}. Line 14 calculates the
oscillator strength and transition rates between cofiguration groups
\verb|n2| and \verb|n3|, and saves the results in the binary file
\verb|ne.tr.b|. The function \verb|TransitionTable| accepts an optional 4th
integer argument specifying the transition type. A negative integer means
electric multipole and a positive integer for magnetic multipole. The absolute
value of the integer indicates the rank of the multipole. Therefore, $-1$ would
be E1, $+1$ would be M1, etc. Without this argument, the default is 0, where
we sum up all the multipoles for the rate, as is done here.
Line 15 converts the binary output to an ASCII file in verbose
mode. The exact formats of binary and ASCII files are explained in Chapter
\ref{cha:format}. Here we list the two ASCII files \verb|ne.lev| and
\verb|ne.tr| resulted from this calculation.
\begin{verbatim}
File ne.lev:

FAC 1.0.7
Endian	= 1
TSess	= 1103048155
Type	= 1
Verbose	= 1
Fe Z	=  26.0
NBlocks	= 1
E0	= 0, -3.12289011E+04

NELE	= 10
NLEV	= 37
  ILEV  IBASE    ENERGY       P   VNL 2J
     0     -1  0.00000000E+00 0   201  0 1*2 2*8       2p6         2p+4(0)0
     1     -1  7.23817529E+02 1   300  4 1*2 2*7 3*1   2p5 3s1     2p+3(3)3 3s+1(1)4
     2     -1  7.25866864E+02 1   300  2 1*2 2*7 3*1   2p5 3s1     2p+3(3)3 3s+1(1)2
     3     -1  7.36421878E+02 1   300  0 1*2 2*7 3*1   2p5 3s1     2p-1(1)1 3s+1(1)0
     4     -1  7.37744083E+02 1   300  2 1*2 2*7 3*1   2p5 3s1     2p-1(1)1 3s+1(1)2
     5     -1  7.54155726E+02 0   301  2 1*2 2*7 3*1   2p5 3p1     2p+3(3)3 3p-1(1)2
     6     -1  7.57795103E+02 0   301  4 1*2 2*7 3*1   2p5 3p1     2p+3(3)3 3p-1(1)4
     7     -1  7.59348028E+02 0   301  6 1*2 2*7 3*1   2p5 3p1     2p+3(3)3 3p+1(3)6
     8     -1  7.60559034E+02 0   301  2 1*2 2*7 3*1   2p5 3p1     2p+3(3)3 3p+1(3)2
     9     -1  7.62368042E+02 0   301  4 1*2 2*7 3*1   2p5 3p1     2p+3(3)3 3p+1(3)4
    10     -1  7.68005929E+02 0   301  0 1*2 2*7 3*1   2p5 3p1     2p+3(3)3 3p+1(3)0
    11     -1  7.69846810E+02 0   301  2 1*2 2*7 3*1   2p5 3p1     2p-1(1)1 3p-1(1)2
    12     -1  7.73062840E+02 0   301  2 1*2 2*7 3*1   2p5 3p1     2p-1(1)1 3p+1(3)2
    13     -1  7.73470206E+02 0   301  4 1*2 2*7 3*1   2p5 3p1     2p-1(1)1 3p+1(3)4
    14     -1  7.90365155E+02 0   301  0 1*2 2*7 3*1   2p5 3p1     2p-1(1)1 3p-1(1)0
    15     -1  8.00169329E+02 1   302  0 1*2 2*7 3*1   2p5 3d1     2p+3(3)3 3d-1(3)0
    16     -1  8.01137358E+02 1   302  2 1*2 2*7 3*1   2p5 3d1     2p+3(3)3 3d-1(3)2
    17     -1  8.02966076E+02 1   302  4 1*2 2*7 3*1   2p5 3d1     2p+3(3)3 3d+1(5)4
    18     -1  8.03096178E+02 1   302  8 1*2 2*7 3*1   2p5 3d1     2p+3(3)3 3d+1(5)8
    19     -1  8.03812355E+02 1   302  6 1*2 2*7 3*1   2p5 3d1     2p+3(3)3 3d-1(3)6
    20     -1  8.05502105E+02 1   302  4 1*2 2*7 3*1   2p5 3d1     2p+3(3)3 3d-1(3)4
    21     -1  8.06580377E+02 1   302  6 1*2 2*7 3*1   2p5 3d1     2p+3(3)3 3d+1(5)6
    22     -1  8.11326891E+02 1   302  2 1*2 2*7 3*1   2p5 3d1     2p+3(3)3 3d+1(5)2
    23     -1  8.16455702E+02 1   302  4 1*2 2*7 3*1   2p5 3d1     2p-1(1)1 3d-1(3)4
    24     -1  8.17103423E+02 1   302  4 1*2 2*7 3*1   2p5 3d1     2p-1(1)1 3d+1(5)4
    25     -1  8.17679088E+02 1   302  6 1*2 2*7 3*1   2p5 3d1     2p-1(1)1 3d+1(5)6
    26     -1  8.25272086E+02 1   302  2 1*2 2*7 3*1   2p5 3d1     2p-1(1)1 3d-1(3)2
    27     -1  8.60711742E+02 0   300  2 1*2 2*7 3*1   2s1 3s1     2s+1(1)1 3s+1(1)2
    28     -1  8.67823477E+02 0   300  0 1*2 2*7 3*1   2s1 3s1     2s+1(1)1 3s+1(1)0
    29     -1  8.93685214E+02 1   301  0 1*2 2*7 3*1   2s1 3p1     2s+1(1)1 3p-1(1)0
    30     -1  8.94146680E+02 1   301  2 1*2 2*7 3*1   2s1 3p1     2s+1(1)1 3p-1(1)2
    31     -1  8.96450212E+02 1   301  4 1*2 2*7 3*1   2s1 3p1     2s+1(1)1 3p+1(3)4
    32     -1  8.98437005E+02 1   301  2 1*2 2*7 3*1   2s1 3p1     2s+1(1)1 3p+1(3)2
    33     -1  9.38592464E+02 0   302  2 1*2 2*7 3*1   2s1 3d1     2s+1(1)1 3d-1(3)2
    34     -1  9.38724254E+02 0   302  4 1*2 2*7 3*1   2s1 3d1     2s+1(1)1 3d-1(3)4
    35     -1  9.38975219E+02 0   302  6 1*2 2*7 3*1   2s1 3d1     2s+1(1)1 3d+1(5)6
    36     -1  9.43734651E+02 0   302  4 1*2 2*7 3*1   2s1 3d1     2s+1(1)1 3d+1(5)4


File ne.tr:

FAC 1.0.7
Endian  = 1
TSess   = 1103048155
Type    = 2
Verbose = 1
Fe Z    =  26.0
NBlocks = 1

NELE    = 10
NTRANS  = 7
MULTIP  = -1
GAUGE   = 2
MODE    = 1
     2  2      0  0  7.2587E+02  1.130597E-01  8.616084E+11  1.127617E-01
     4  2      0  0  7.3774E+02  9.944485E-02  7.828559E+11  1.048997E-01
    16  2      0  0  8.0114E+02  9.438239E-03  8.761793E+10 -3.101188E-02
    22  2      0  0  8.1133E+02  6.221187E-01  5.923155E+12 -2.501928E-01
    26  2      0  0  8.2527E+02  2.493449E+00  2.456309E+13  4.966355E-01
    30  2      0  0  8.9415E+02  3.203146E-02  3.704097E+11  5.407792E-02
    32  2      0  0  8.9844E+02  2.652003E-01  3.096259E+12 -1.552313E-01
\end{verbatim}

In file \verb|ne.lev|, the energy, parity, $2J$ ($J$ is the total
angular momentum of the level), and configuration coupling informations are
listed. In file \verb|ne.tr|, the upper and lower level indexes, the $2J$
values of these levels, the transition energy, $gf$-values, radiative
decay rates, and the reduced multipole matrix elements are given.

\section*{Acknowledgments}
Throughout the development of this work, the discussion with Ehud Behar, Masao
Sako, Peter Beiersdorfer, Ali Kinkhabwala and Steven Kahn has been very
useful. Many Fortran 77 subroutines were retrieved from Netlib repository
(\textbf{www.netlib.org}) and used in this package, as well as several
programs from Computer Physics Communications Program Library at
\textbf{www.cpc.cs.qub.ac.uk}.

The original development of this code (during Dec 2000 -- Aug 2003, or prior
to version 1.0.2) was
supported by NASA through Chandra Postdoctoral Fellowship Award Number
PF01-10014 issued by the Chandra X-ray Observatory Center, which is operated
by Smithsonian Astrophysical Observatory for and on behalf of NASA under
contract NAS8-39073.

Any opinions, findings and conclusions or
recommendations expressed in this manual are those of the author and do not
necessrarily reflect the views of the National Aeronautics Space
Administration and/or the Smithonian Astrophysical Observatory.

\chapter{Description of Output Files}
\label{cha:format}
The primary output files of FAC are in binary format. The I/O functionality
and the conversion from binary to ASCII format are implemented in the source
files \verb|faclib/dbase.h| and \verb|faclib/dbase.c|. In this chapter, we
describe the structure of these files in detail.

\section{Binary Format}
\label{sec:binary}
\index{Binary format}
Presently, FAC produces different types of files. Each type is asigned
a unique integer, which corresponds to a macro define in the file
\verb|faclib/dbase.h|. These types are
\begin{description}
\item[\texttt{DB\_EN = 1}] Energy levels produced by the function
\verb|fac.Structure|.
\item[\texttt{DB\_TR = 2}] Radiative transition rates produced by
\verb|fac.TransitionTable|.
\item[\texttt{DB\_CE = 3}] Collisional excitation cross sections produced by
\verb|fac.CETable|.
\item[\texttt{DB\_RR = 4}] Radiative recombination and photoionization cross
sections produced by \verb|fac.RRTable|.
\item[\texttt{DB\_AI = 5}] Autoionization rates produced by \verb|fac.AITable|.
\item[\texttt{DB\_CI = 6}] Collisional ionization cross sections produced by
\verb|fac.CITable|.
\item[\texttt{DB\_SP = 7}] Spectral line strengths produced by
\verb|crm.SpecTable|.
\item[\texttt{DB\_RT = 8}] Various population rates produced by
\verb|crm.RateTable|.
\item[\texttt{DB\_DR = 9}] Various population rates produced by
\verb|crm.DRStrength|.
\item[\texttt{DB\_AIM = 10}] Magnetic sublevel autoionization rates and DR
capture strengths produced by \verb|fac.AITableMSub|.
\item[\texttt{DB\_CIM = 11}] Magnetic sublevel collisional ionization cross
  sections produced by \verb|fac.CITableMSub|.
\end{description}

All files have a common structure. It consists of a file header and one or
more data blocks. Each data block is comprised of a data header and one or
more data records. In the following, we show the C definition of all structs
and describe each field in detail. When one field is a pointer, it means that
an array is saved in the database. The pointer points to the memory location
where the data is stored. In versions 1.0.8 or earlier, the value of the
pointer itself is also saved in the file followed by the data stored in the
array. Obviously, the saved pointer itself has no meaning once the program
exits (since it is a memory location). When reading out the data from the
database file, these pointer values should be ignored. In version 1.0.9 or
later, the pointers are no longer saved, and the file IO are rewritten in a
platform independent way, i.e., the structure fields are written and read one
by one, instead of dealing with the structure as an integrated object. This
avoids the different memory padding added by the compilers which may be
different on different machines.

\subsection{\texttt{F\_HEADER}}
\index{F\_HEADER}
\texttt{F\_HEADER} is the file header common to all data files.

\begin{verbatim}
typedef struct _F_HEADER_ {
  long  tsession;
  int   version;
  int   sversion;
  int   ssversion;
  int   type;
  float atom;
  char  symbol[4];
  int   nblocks;
} F_HEADER;
\end{verbatim}

\begin{dbdesc}
\item[\texttt{long tsession}:] Time stamp when the file is created. This is the
value returned by the C lib function time(0). It is platform dependent.
\item[\texttt{int version}:] Major version number of FAC.
\item[\texttt{int sversion}:] Minor version number of FAC.
\item[\texttt{int ssversion}:] Release number of FAC.
\item[\texttt{int type}:] Type of the data file.
\item[\texttt{float atom}:] Atomic number.
\item[\texttt{char symbol[4]}:] The first 3 bytes contains a NULL
terminated C string representing the 2-charactor abbreviation of the atomic
symbol. The 4th byte is either 0 or 1, indicating whether the platform stores
data in litle or big endian.
\item[\texttt{int nblocks}:] Number of data blocks in this file.
\end{dbdesc}

\subsection{\texttt{EN\_HEADER}}
\index{EN\_HEADER}
\texttt{EN\_HEADER} is the data header for energy level data blocks.

\begin{verbatim}
typedef struct _EN_HEADER_ {
  long position;
  long length;
  int nele;
  int nlevels;
} EN_HEADER;
\end{verbatim}

\begin{dbdesc}
\item[\texttt{long position}:] The number of bytes from the beginning of the
file to the place where this data block starts.
\item[\texttt{long length}:] Number of bytes in this data block, excluding the
length of the header.
\item[\texttt{int nele}:] Number of electrons in the ion for this block.
\item[\texttt{int nlevels}:] Number of levels in this block.
\end{dbdesc}

\subsection{\texttt{EN\_RECORD}}
\index{EN\_RECORD}
\texttt{EN\_RECORD} represents an energy level.

\begin{verbatim}
#define LNCOMPLEX   32
#define LSNAME      48
#define LNAME       128

typedef struct _EN_RECORD_ {
  short p;
  short j;
  int ilev;
  int ibase;
  double energy;
  char ncomplex[LNCOMPLEX];
  char sname[LSNAME];
  char name[LNAME];
} EN_RECORD;
\end{verbatim}

Note that FAC before 1.1.5, we used shorter length strings for the level names,
\begin{verbatim}
#define LNCOMPLEX   32
#define LSNAME      24
#define LNAME       56
\end{verbatim}

\begin{dbdesc}
\item[\texttt{LNCOMPLEX}:] The length of array holding the complex name.
\item[\texttt{LSNAME}:] The length of array holding the non-relativistic
configuration name.
\item[\texttt{LNAME}:] The length of array holding the relativistic
configuration array.
\item[\texttt{short p}:] The parity of the level. This parameter was changed
in version 0.7.6, and it becomes $\pm(100\times n + l)$, where $n$ and $l$ are
the principle quantum number and orbital angular number of the valence
electron, and the $\pm$ sign indicates an even ($+$) or odd ($-$) parity state.
\item[\texttt{short j}:] 2 $\times$ the total angular momentum of the
  level. In UTA mode, \texttt{j} is supposed to be the statistical weight
  minus 1 of the UTA level. However, because a \texttt{short} variable is
  sometimes insufficient to store that value, the code stores it in
  \texttt{ibase} instead. In this case, \texttt{j} is always $-1$.
\item[\texttt{int ilev}:] The index of the level.
\item[\texttt{int ibase}:] The index of the base level. The base level
  obtained by peeling off the valence electron, and the resulting levels must
  be present in the same level file. Other wise, its value is -1. It is not
  always possible to determine the base level, e.g., when the valence orbital
  is occupied by more than one electrons. In such cases, \texttt{ibase} is
  also -1. The value of \texttt{ibase} is primarily used in dealing with DR
  and RE rates, where it may help the distinguish different resonance channels
  and facilitate easy extrapolation. This variable is only added in version
  FAC 1.0.4. So these later versions are not compartible with the binary
  output of the ealier versions.
\item[\texttt{energy}:] The energy of the level in Hartree.
\item[\texttt{char ncomplex[LNCOMPLEX]}:] The complex name. It is in the format
of \verb|n1*nq1 n2*nq2|$\cdots$, where \verb|n1| and \verb|n2| are the
principle quantum numbers of the shell, \verb|nq1| and \verb|nq2| are the
occupation number of these shells.
\item[\texttt{char sname[LSNAME]}:] The non-relativstic configuration name of
the level. Each non-relativistic shell is denoted by the standard
spectroscopic notation, e.g., \verb|2p2| for 2 electrons in $2p$ shell. Only
open and non-empty shells are given. No coupling information is available in
this name.
\item[\texttt{char name[LNAME]}:] The relativstic configuration name of the
level. Each shell is denoted such that \verb|2p+2(2)| represents 2 electrons in
$2p_{3/2}(J=1)$ and \verb|2p-2(2)| represents 2 electrons in
$2p_{1/2}(J=1)$. The number in the parenthesis is 2 times the total angular
momentum of the coupled shell. Immediately after the parenthesis, there is a
number indicate the $2J$ value when all preceding shells are
coupled. Therefore, \verb|2p+2(2)2 2p-2(2)0| represents a state
$[2p_{3/2}^{2}(J=1) 2p_{1/2}^2(J=1)]J=0$.
\end{dbdesc}

\subsection{\texttt{TR\_HEADER}}
\index{TR\_HEADER}
\texttt{TR\_HEADER} is the data header for the radiative transition data
blocks.

\begin{verbatim}
typedef struct _TR_HEADER_ {
  long position;
  long length;
  int nele;
  int ntransitions;
  int gauge;
  int mode;
  int multipole;
} TR_HEADER;
\end{verbatim}

\begin{dbdesc}
\item[\texttt{long position}:] The number of bytes from the beginning of the
file to the place where this data block starts.
\item[\texttt{long length}:] Number of bytes in this data block, excluding the
length of the header.
\item[\texttt{int nele}:] Number of electrons in the ion for this block.
\item[\texttt{int ntransitions}:] Number of transitions in this block.
\item[\texttt{int gauge}:] Gauge used in the calculation. 1 is Coulomb gauge, or
the velocity form in non-relativistic limit. 2 is Babushkin gauge or the
length form.
\item[\texttt{int mode}:] Mode used in the calculation. 0 is fully
relativistic. 1 is non-relativistic approximation for multipole operators.
\item[\texttt{int multipole}:] Multipole type of the transition. Its absolute
value is the rank of the multipole, 1 for dipole, 2 for quadrupole, etc. The
positive sign represents magnetic type and negative sign represents electric type.
\end{dbdesc}

\subsection{\texttt{TR\_RECORD}}
\index{TR\_RECORD}
\texttt{TR\_RECORD} is the for radiative transition data.

\begin{verbatim}
typedef struct _TR_RECORD_ {
  int lower;
  int upper;
  float strength;
} TR_RECORD;
\end{verbatim}

\begin{dbdesc}
\item[\texttt{int lower}:] The lower level index of the transition.
\item[\texttt{int upper}:] The upper level index of the transition.
\item[\texttt{float strength}:] In version 1.0.6 or older, This is the
  weighted oscillator strength $gf$ of the transition. The weighted radiative
  transition rate is related to $gf$ as (in atomic units):
\begin{equation}
gA = 2\alpha^3 \omega^2 gf,
\end{equation}
where $\alpha$ is the fine structure constant, and $\omega$ is transition
energy in Hartree atomic units.

In version 1.0.7 or newer, the stored value depends on the multipole values
used in \key{TransitionTable}.
By default (\var{m} = 0), this stores the total oscillator strengths $gf$ for
all the multipoles.
If other value is used for \var{m}, this stores the multipole matrix elements
$M$ instead. It is related to the $gf$ value as
\begin{equation}
gf =
\left(2L+1\right)^{-1}\omega\left(\alpha\omega\right)^{2L-2}\left|M\right|^2,
\end{equation}
where $L = |m|$ is the multipole rank.
\end{dbdesc}

\subsection{\texttt{TR\_EXTRA}}
\index{TR\_EXTRA}
\texttt{TR\_EXTRA} contains the UTA related transition data, namely, the
transition energy including the UTA shift, the UTA Gaussian width, and the
configuration interaction multipler. This structure is written to the
\texttt{DB\_TR} file only in UTA mode, which is set by \key{SetUTA()}
function.

\begin{verbatim}
typedef struct _TR_EXTRA_ {
  float energy;
  float sdev;
  float sci;
} TR_EXTRA;
\end{verbatim}

\begin{dbdesc}
\item[\texttt{float energy}:] The transition energy including UTA shift.
\item[\texttt{float sdev}:] The Gaussian standard deviation of the UTA.
\item[\texttt{float sci}:] The configuration interaction multipler, which
  accounts for the CI within the same non-relativistic configurations.
\end{dbdesc}

\subsection{\texttt{CE\_HEADER}}
\index{CE\_HEADER}
\texttt{CE\_HEADER} is the data header for collisional excitation data blocks.

\begin{verbatim}
typedef struct _CE_HEADER_ {
  long position;
  long length;
  int nele;
  int ntransitions;
  int qk_mode;
  int n_tegrid;
  int n_egrid;
  int egrid_type;
  int n_usr;
  int usr_egrid_type;
  int nparams;
  int pw_type;
  int msub;
  float te0;
  double *tegrid;
  double *egrid;
  double *usr_egrid;
} CE_HEADER;
\end{verbatim}

\begin{dbdesc}
\item[\texttt{long position}:] The number of bytes from the beginning of the
file to the place where this data block starts.
\item[\texttt{long length}:] Number of bytes in this data block, excluding the
length of the header.
\item[\texttt{int nele}:] Number of electrons in the ion for this block.
\item[\texttt{int ntransitions}:] Number of transitions in this block.
\item[\texttt{int qk\_mode}:] The mode for the calculation of radial
integrals. There are 3 choices for collisional excitation. 0 for EXACT, 1 for
INTERPOLATE, and 2 for FIT. In the EXACT mode, the collison strengths are
calculated at the energy grid specified as is, so the \texttt{egrid} and
\texttt{usr\_egrid} must be the same. In the INTERPOLATE mode, the collision
strengths are calculated at \texttt{egrid}, and interpolated to
\texttt{usr\_egrid}. In the FIT mode, the collision strengths are fitted to an
analytic formula and the parameters are output as well. For collision
strengths of magnetic sublevels, the FIT mode is not implemented.
\item[\texttt{int n\_tegrid}:] Number of points for the transition energy grid.
\item[\texttt{int n\_egrid}:] Number of points for the collision energy grid.
\item[\texttt{int egrid\_type}:] Type of the energy grid. 0 for the incident
electron energy, 1 for scattered electron energy. In the present
implementation, only scattered electron energy grid is supported.
\item[\texttt{int n\_usr}:] Number of points for the user collision energy
grid.
\item[\texttt{int usr\_egrid\_type}:] Type of the user energy grid. 0 for the
incident electron energy, 1 for scattered electron energy. In the present
implementation, only scattered electron energy grid is supported.
\item[\texttt{int nparams}:] Number of parameters in the fitting formula if the
collision strengths are calculated in the FIT mode. At present,
\texttt{nparams} is 4.
\item[\texttt{int pw\_type}:] Partial wave type for the last summation. 0 for
the incident electron, 1 for the scattered electron.
\item[\texttt{int msub}:] 0 for total collision strength, 1 for magnetic
sublevel specific collision strength.
\item[\texttt{float te0}:] The characteristic transition energy of the
transition array. This is used for the automatic construction of the
collision energy grid. The grid has equal space in $\ln$(\texttt{egrid+te0})
if \texttt{egrid\_type = 1}, otherwise, this variable is not used.
\item[\texttt{double *tegrid}:] The transition energy grid, the number of
elements is given by \texttt{n\_tegrid}.
\item[\texttt{double *egrid}:] The energy grid, the number of elements is
given by \texttt{n\_egrid}.
\item[\texttt{double *usr\_egrid}:] The user energy grid, the number of
elements is given by \texttt{n\_usr}.
\end{dbdesc}

\subsection{\texttt{CE\_RECORD}}
\index{CE\_RECORD}
\texttt{CE\_RECORD} is for collisional excitation data.

\begin{verbatim}
typedef struct _CE_RECORD_ {
  int lower;
  int upper;
  int nsub;
  float bethe;
  float born[2];
  float *params;
  float *strength;
} CE_RECORD;
\end{verbatim}

\begin{dbdesc}
\item[\texttt{int lower}:] The lower level index.
\item[\texttt{int upper}:] The upper level index.
\item[\texttt{int nsub}:] Number of magnetic sublevel transitions. Because of
time reversal symmetry, $\sigma_{m_1\to m_2}=\sigma_{-m_1\to -m_2}$, only cross
sections with $m_1 <= 0$ are tabulated.
\item[\texttt{float bethe}:] The Bethe coefficients in the first-Born
approximation. It is the logarithmic coefficients at high energies. If
\texttt{bethe[0]}$<0$, it is a spin forbidden transition. Otherwise, it is
either a optical-allowed transition or other multipole-allowed transitions.
\item[\texttt{float born[2]}:] The Born limit of the collision strengths at
high energies, which is
\begin{eqnarray}
x &=& \frac{E_0}{E_{th}} \nonumber\\
\Omega &=& b_0\ln(x) + b_1,
\end{eqnarray}
where $b_0$ is given by \texttt{bethe}, if it is an allowed transition. The
parameter $b_1$ is calculated at an energy given by $b_2$, which is chosen to
be very high, about $10^{2}E_{th}$ or higher.
For spin forbidden transitions, $b_0 = 0$. $b_1$, $b_2$ are stored in the array
\texttt{born[2]}. These numbers are useful to extrapolate the collision
strengths to high energies with correct aysmptotic behaviour.
\item[\texttt{float *params}:] Parameters for the fitting formula, if the
fitting mode is used. The number of elements is given by \texttt{nparams} in
\texttt{CE\_HEADER}. In the present implementation, different fitting formulae
are used for allowed and forbidden transitions. The number of parameters is 4
in all cases. The FIT mode is not robust, avoid using it.

For dipole and higher multipole allowed transitions, the
collision strength $\Omega$ is given by
\begin{eqnarray}
x &=& \frac{E_0}{E_{th}} \nonumber\\
\Omega &=& p_0\left(\frac{1}{x}\right)^{p_1} +
p_2\left(1-\frac{1}{x}\right)^{p_3} + b\ln x,
\end{eqnarray}
where $E_0$ is the energy of the incident electron, $E_{th}$ is the transition
threshold, $p_0$, $p_1$, $p_2$ and $p_4$ are four parameters, and $b$ is the
Bethe coeffificient, which is 0 for non-dipole transitions.

For forbidden transitions, the collision strength is given by
\begin{eqnarray}
\gamma &=& -2.0 + p_1\frac{1}{p_3+x} +
p_2\left(\frac{1}{p_3+x}\right)^2\nonumber\\
\Omega &=& p_0x^\gamma.
\end{eqnarray}

The FIT mode only applies to the calculation of total cross sections. For
magnetic sublevel cross sections, \texttt{params} has \texttt{nsub} elements,
which are the ratios of magnetic sublevel collision strengths to the total
collision strength at high energy limit for allowed transitions. For forbidden
transitions, these numbers are all 0.

\item[\texttt{float *stregnth}:] Collision stregnth on the user energy
grid. The number of elements is given by \texttt{n\_usr} in
\texttt{CE\_HEADER}. It is related to the excitation cross section as (in
atomic units):
\begin{equation}
\sigma = \frac{\pi}{k_0^2g_0}\Omega,
\end{equation}
where $g_0$ is the statistical weight of the initial state, and $k_0$ is the
kinetic momentum of the incident electron. The number of elements in this
array is \texttt{nsub}$\times$\texttt{n\_usr}.
\end{dbdesc}

\subsection{\texttt{RR\_HEADER}}
\index{RR\_HEADER}
\texttt{RR\_HEADER} is the data header for radiative recombination and
photoionization data blocks.

\begin{verbatim}
typedef struct _RR_HEADER_ {
  long position;
  long length;
  int nele;
  int ntransitions;
  int qk_mode;
  int multipole;
  int n_tegrid;
  int n_egrid;
  int egrid_type;
  int n_usr;
  int usr_egrid_type;
  int nparams;
  double *tegrid;
  double *egrid;
  double *usr_egrid;
} RR_HEADER;
\end{verbatim}

\begin{dbdesc}
\item[\texttt{long position}:] The number of bytes from the beginning of the
file to the place where this data block starts.
\item[\texttt{long length}:] Number of bytes in this data block, excluding the
length of the header.
\item[\texttt{int nele}:] Number of electrons in the ion for this block.
\item[\texttt{int ntransitions}:] Number of transitions in this block.
\item[\texttt{int qk\_mode}:] The mode for the calculation of radial
integrals. There are 3 choices at present. 0 for EXACT, 1 for INTERPOLATE, and
2 for FIT, similar to collsional excitation. However, even if the FIT
mode is used, the fitting formula is only valid in the high energy asymptotic
regions. The low energy results should be obtained by interpolation.
\item[\texttt{int multipole}:] Multipole type of the transition. Its absolute
value is the rank of the multipole, 1 for dipole, 2 for quadrupole, etc. The
positive sign for magnetic type and negative sign for electric type. Usually,
only E1 type is relavent for radiative recombination and photoionization.
\item[\texttt{int n\_tegrid}:] Number of points for the transition energy grid.
\item[\texttt{int n\_egrid}:] Number of points for the collision energy grid.
\item[\texttt{int egrid\_type}:] Type of the energy grid. 0 for the incident
photon energy, 1 for photo-electron energy.
\item[\texttt{int n\_usr}:] Number of points for the user collision energy
grid.
\item[\texttt{int usr\_egrid\_type}:] Type of the user energy grid. 0 for the
incident photon energy, 1 for photo-electron energy.
\item[\texttt{int nparams}:] Number of parameters in the fitting formula if the
bound-free oscillator strengths are calculated in the FIT mode. In the present
imprementation, \texttt{nparams} is 4.
\item[\texttt{double *tegrid}:] The transition energy grid, the number of
elements is given by \texttt{n\_tegrid}.
\item[\texttt{double *egrid}:] The energy grid, the number of elements is
given by \texttt{n\_egrid}.
\item[\texttt{double *usr\_egrid}:] The user energy grid, the number of
elements is given by \texttt{n\_usr}.
\end{dbdesc}

\subsection{\texttt{RR\_RECORD}}
\index{RR\_RECORD}
\texttt{RR\_RECORD} is for radiative recombination and photoionization data.

\begin{verbatim}
typedef struct _RR_RECORD_ {
  int b;
  int f;
  int kl;
  float *params;
  float *strength;
} RR_RECORD;
\end{verbatim}

\begin{dbdesc}
\item[\texttt{int b}:] The bound state index.
\item[\texttt{int f}:] The free state index.
\item[\texttt{int kl}:] The orbital angular momentum of the ionized shell for
the dominant wavefunction component.
\item[\texttt{float *params}:] The parameters in the fitting formula for the
bound-free oscillator strength, if the FIT mode is
used. The fitting formula only provides a high energy asymptotic behavior. Low
energy values should be interpolated from the tabulated strengths. The fitting
formula is
\begin{eqnarray}
x &=& \frac{E_e+p_3}{p_3} \nonumber\\
y &=& \frac{1+p_2}{\sqrt{x}+p_2} \nonumber\\
\frac{d(gf)}{dE} &=&
\frac{E_\gamma}{E_e+p_3}p_0x^{-3.5-l+\frac{1}{2}p_1}y^{p_1},
\end{eqnarray}
where $E_e$ is the photo-electron energy, $E_\gamma$ is the photon energy,
$E_{th}$ is the ionization threshold, $p_0$, $p_1$, $p_2$, and $p_3$ are the
parameters, and $l$ is the orbital angular momentum of the ionized
shell. The asymptotic behavior represented by the power law only takes into
account the ionization of the dominant basis in the wavefunction
expansion. The result is in atomic unit Hartree$^{-1}$.
\item[\texttt{float *strength}:] The weighted bound-free oscillator strength in
atomic units. It is related to photoionization and radiative recombination as
(in atomic units):
\begin{eqnarray}
\sigma_{PI} &=& 2\pi\alpha\frac{d f}{d E} \nonumber\\
            &=& \frac{2\pi\alpha}{g_i}
		 \frac{1+\alpha^2\varepsilon}{1+\frac{1}{2}\alpha^2 \varepsilon}
		 \frac{d(gf)}{d E} \nonumber\\
\sigma_{RR} &=& \frac{\alpha^2}{2}\frac{g_i}{g_f}
                \frac{\omega^2}{\varepsilon \left(1+\frac{1}{2}\alpha^2
                  \varepsilon\right)} \sigma_{PI},
\end{eqnarray}
where $\alpha$ is the fine structure constant, $g_i$ and $g_f$ are the
statistical weight of the bound states before and after the photoionzation
takes place respectively, $\omega$ is the photon energy, and $\varepsilon$ is
the energy of the ejected photo-electron. The tabulated values are $d(gf)/dE$.
\end{dbdesc}

\subsection{\texttt{AI\_HEADER}}
\index{AI\_HEADER}
\texttt{AI\_HEADER} is the data header for autoionization data blocks.

\begin{verbatim}
typedef struct _AI_HEADER_ {
  long position;
  long length;
  int nele;
  int ntransitions;
  int channel;
  int n_egrid;
  double *egrid;
} AI_HEADER;
\end{verbatim}

\begin{dbdesc}
\item[\texttt{long position}:] The number of bytes from the beginning of the
file to the place where this data block starts.
\item[\texttt{long length}:] Number of bytes in this data block, excluding the
length of the header.
\item[\texttt{int nele}:] Number of electrons in the ion for this block.
\item[\texttt{int ntransitions}:] Number of transitions in this block.
\item[\texttt{int channel}:] This an identifier to label the autoionization
channel, which does not have specific physical meaning.
\item[\texttt{int n\_egrid}:] The number of points for the Auger electron
energy grid. The autoionzation radial integrals are calculated on this grid
and interpolated to the actual discrete energies.
\item[\texttt{double *egrid}:] The energy grid. The number of elements is
given by \texttt{n\_egrid}.
\end{dbdesc}

\subsection{\texttt{AI\_RECORD}}
\index{AI\_RECORD}
\texttt{AI\_RECORD} is for autoionization data.

\begin{verbatim}
typedef struct _AI_RECORD_ {
  int b;
  int f;
  float rate;
} AI_RECORD;
\end{verbatim}

\begin{dbdesc}
\item[\texttt{int b}:] The bound state index.
\item[\texttt{int f}:] The free state index.
\item[\texttt{float rate}:] The autoionization rate.
\end{dbdesc}

\subsection{\texttt{CI\_HEADER}}
\index{CI\_HEADER}
\texttt{CI\_HEADER} is the data header for collisional ionization data blocks.

\begin{verbatim}
typedef struct _CI_HEADER_ {
  long position;
  long length;
  int nele;
  int ntransitions;
  int qk_mode;
  int n_tegrid;
  int n_egrid;
  int egrid_type;
  int n_usr;
  int usr_egrid_type;
  int nparams;
  int pw_type;
  double *tegrid;
  double *egrid;
  double *usr_egrid;
} CI_HEADER;
\end{verbatim}

\begin{dbdesc}
\item[\texttt{long position}:] The number of bytes from the beginning of the
file to the place where this data block starts.
\item[\texttt{long length}:] Number of bytes in this data block, excluding the
length of the header.
\item[\texttt{int nele}:] Number of electrons in the ion for this block.
\item[\texttt{int ntransitions}:] Number of transitions in this block.
\item[\texttt{int qk\_mode}:] The mode for the calculation of radial
integrals. At present, there are 3 choices. 3 for CB mode (Coulomb-Born), 4
for DW mode (distorted-wave), and 5 for BED mode (binary-encounter-dipole). In
CB mode, the radial integrals are obtained by looking up a table of
Coulomb-Born-Exchange results from \citet{golden77,golden80}, which is very
fast. In DW mode, the integrals are calculated using the distorted-wave
approximation, which is very slow. In BED mode, the binary-encounter-dipole
theory of \citet{kim94} is used which makes use of bound-free oscillator
strength of the same transition. This method is also very fast.
\item[\texttt{int n\_tegrid}:] Number of points for the transition energy grid.
\item[\texttt{int n\_egrid}:] Number of points for the collision energy grid.
\item[\texttt{int egrid\_type}:] Type of the energy grid. 0 for the incident
electron energy, 1 for the total energy of scattered and ejected electron.
\item[\texttt{int n\_usr}:]N umber of points for the user collision energy
grid.
\item[\texttt{int usr\_egrid\_type}:] Type of the user energy grid. 0 for the
incident electron energy, 1 for the total energy of scattered and ejected
electrons .
\item[\texttt{int nparams}:] Number of parameters in the fitting formula. The
final collision strength for total ionization cross sections are fitted with a
4 parameter formula.
\item[\texttt{int pw\_type}:] Partial wave type for the last summation. 0 for
the incident electron, 1 for the scattered electron. It is always 0 for
distorted-wave calculation of ionization.
\item[\texttt{double *tegrid}:] The transition energy grid, the number of
elements is given by \texttt{n\_tegrid}.
\item[\texttt{double *egrid}:] The energy grid, the number of elements is
given by \texttt{n\_egrid}.
\item[\texttt{double *usr\_egrid}:] The user energy grid, the number of
elements is given by \texttt{n\_usr}.
\end{dbdesc}

\subsection{\texttt{CI\_RECORD}}
\index{CI\_RECORD}
\texttt{CI\_RECORD} is for collisional ionization data.

\begin{verbatim}
typedef struct _CI_RECORD_ {
  int b;
  int f;
  int kl;
  float *params;
  float *strength;
} CI_RECORD;
\end{verbatim}

\begin{dbdesc}
\item[\texttt{int b}:] The bound state index.
\item[\texttt{int f}:] The free state index.
\item[\texttt{int kl}:] The orbital angular momentum of the ionized shell for
the dominant wavefunction component.
\item[\texttt{float *params}:] The parameters in the fitting formula for the
collision strength. The number of elements is given by \texttt{nparams} in
\texttt{CI\_HEADER}, which is 4. The formula used is
\begin{eqnarray}
x &=& \frac{E_0}{E_{th}} \nonumber\\
y &=& 1-\frac{1}{x} \nonumber\\
\Omega &=& p_0\ln x + p_1y^2 + p_2\frac{1}{x}y + p_3\frac{1}{x^2}y,
\end{eqnarray}
where $E_0$ is the energy of the incident electron, $E_{th}$ is the ionization
threshold, $p_0$, $p_1$, $p_2$, and $p_3$ are the four parameters. The
parameter $p_0$ is actually obtained from the bound-free oscillator strength,
which is more reliable than one would get by fitting the calculated collision stregnths.
\item[\texttt{float *strength}:] The collision strength for ionization. It is
related to the ionization cross section as (in atomic units):
\begin{equation}
\sigma = \frac{1}{k_0^2g_0}\Omega,
\end{equation}
where ${k_0}$ is the kinetic momentum of the incident electron, and $g_0$ is
the statistical weight of the initial state. The missing of the factor $\pi$
as compared to the formula for collisional excitation is due to the different
normalization for bound and free states.
\end{dbdesc}

\subsection{\texttt{SP\_HEADER}}
\index{SP\_HEADER}
\label{subsec:sp_header}
\texttt{SP\_HEADER} is the data header for spectral line data blocks. The
spectral data are generated by CRM model. A \texttt{DB\_SP} data file contains
two parts. The first part is the detailed level population table. The second
part is the spectral line emissivity table. Energy levels in a CRM model are
divided into superlevel blocks. One superlevel block occupies one data
block in the first part, and all transitions between two superlevel blocks
make up one data block in the second part.

\begin{verbatim}
typedef struct _SP_HEADER_ {
  long position;
  long length;
  int nele;
  int ntransitions;
  int iblock;
  int fblock;
  char icomplex[LNCOMPLEX];
  char fcomplex[LNCOMPLEX];
  int type;
} SP_HEADER;
\end{verbatim}

\begin{dbdesc}
\item[\texttt{long position}:] The number of bytes from the beginning of the
file to the place where this data block starts.
\item[\texttt{long length}:] Number of bytes in this data block, excluding the
length of the header.
\item[\texttt{int nele}:] Number of electrons in the ion for this block.
\item[\texttt{int ntransitions}:] For the first part, this the number of
levels in the block. For the second part, this is the number of spectral lines.
\item[\texttt{int iblock}:] For the first part, this the superlevel block
index. For the second part, this is the superlevel block index for the initial
states of transitions.
\item[\texttt{int fblock}:] For the first part, this is always 0. For the
second part, this is the superlevel block index for the final states of the
transitions.
\item[\texttt{char icomplex[LNCOMPLEX]}:] The configuration complex name for
the initial states.
\item[\texttt{char fcomplex[LNCOMPLEX]}:] The configuration complex name for
the final states. For the first part, this is always an empty string.
\item[\texttt{int type}:] Type of the block. For the first part, the type is
always 0. For the second part, the type is encoded as $10000\times n_0 +
100\times n_1 + n_2$, where $n_1$ is the initial principle quantum number of
the electron making the transition, and $n_2$ is the final principle quantum
number. If $n_0 \ne 0$, this transition is the so called dielectronic
recombination satellite line, which have a spectator electron at the orbital
with principle quantum number $n_0$. If $n_0=n_1=0$, then the line is
radiative recombination continuum onto the orbital with principal quantum
number $n_2$.
\end{dbdesc}

\subsection{\texttt{SP\_RECORD}}
\index{SP\_RECORD}
\begin{verbatim}
typedef struct _SP_RECORD_ {
  int lower;
  int upper;
  float energy;
  float strength;
  float rrate;
  float trate;
} SP_RECORD;
\end{verbatim}

\begin{dbdesc}
\item[\texttt{int lower}:] For the level population table, this is level index
within the same ion. For the line emissivity table, this is the level index for
the lower state of the transition.
\item[\texttt{int upper}:] For the level population table, this is the level
index within the same superlevel block. For the line emissivity, this is the
index for the upper state of the transition.
\item[\texttt{float energy}:] For the level population table, this is the
energy of the level in atomic unit. For the line emissivity, this is the
transition energy in eV.
\item[\texttt{float strength}:] For the level population table, this is the
concentration of the level. For the line emissivity, this is the line
luminosity in photons/s.
\item[\texttt{float rrate}:] The radiative transition rate from \key{upper} to
  \key{lower}.
\item[\texttt{float trate}:] The total decay rate of \key{upper}.
\end{dbdesc}

\subsection{\texttt{SP\_EXTRA}}
\index{SP\_EXTRA}
\texttt{SP\_EXTRA} contains the UTA width of the transition. It is written to
the \texttt{DB\_SP} file only in the UTA mode.

\begin{verbatim}
typedef struct _SP_EXTRA_ {
  float sdev;
} SP_EXTRA;
\end{verbatim}

\begin{dbdesc}
\item[\texttt{float sdev}:] The Gaussian UTA standard deviation of the
  transition.
\end{dbdesc}

\subsection{\texttt{RT\_HEADER}}
\index{RT\_HEADER}
\label{subsec:rt_header}
\texttt{RT\_HEADER} is the header for the rate data file. The rates are
generated by the CRM model. It tabulates the rates for individual processes
from and to specific levels or superlevel blocks. In the \texttt{DB\_RT}
files, a data block consists the rates from all other states to one level or
one superlevel depending on how the file is created.

\begin{verbatim}
typedef struct _RT_HEADER_ {
  long position;
  long length;
  int ntransitions;
  int iedist;
  int np_edist;
  double *p_edist;
  float eden;
  int ipdist;
  int np_pdist;
  double *p_pdist;
  float pden;
} RT_HEADER;
\end{verbatim}

\begin{dbdesc}
\item[\texttt{long position}:] The number of bytes from the beginning of the
file to the place where this data block starts.
\item[\texttt{long length}:] Number of bytes in this data block, excluding the
length of the header.
\item[\texttt{int ntransitions}:] The number of superlevel blocks from which
the rates are tabulated in this block.
\item[\texttt{int iedist}:] The type of electron energy distribution for the
CRM model. At present, two types are supported. 0 for Maxwellian, and 1 for an
Gaussian monoenergetic beam.
\item[\texttt{int np\_edist}:] Number of parameters for the electron energy
distribution.
\item[\texttt{double *p\_edist}:] Parameters for the electron energy
distribution. The following table describes the parameters for exsisting
distributions:

\hspace{0.6in}\centerline{
\begin{minipage}{6in}
\index{Electron Energy Distribution}
\begin{tabular}{|c|c|ccc|}
\hline
ID& Dist& \multicolumn{2}{c}{Params}& Unit\\
\hline
0& Maxwellian& 0:& $T_e$ & eV\\
 &           & 1:& $E_{min}$ & eV\\
 &           & 2:& $E_{max}$ & eV\\
\hline
1& Gaussian&   0:& $E_0$ & eV\\
 &         &   1:& $\sigma_E$ & eV\\
 &         &   2:& $E_{min}$ & eV\\
 &         &   3:& $E_{max}$ & eV\\
\hline
2& MaxPower& 0:& $T_e$ & eV\\
 &         & 1:& $\alpha$ & \\
 &         & 2:& $\frac{E_p}{E_m}$ & \\
 &         & 3:& $E_{min}^{p}$ & eV\\
 &         & 4:& $E_{max}^{p}$ & eV\\
 &         & 5:& $E_{min}^{m}$ & eV\\
 &         & 6:& $E_{max}^{m}$ & eV\\
\hline
3& PowerLaw& 0:& $\alpha$ & \\
 &         & 1:& $E_{min}$ & eV\\
 &         & 2:& $E_{max}$ & eV\\
\hline
\end{tabular}
\end{minipage}}
\item[\texttt{float eden}:] The electron density in unit of $10^{10}$
cm$^{-3}$.
\item[\texttt{int ipdist}:] The type of photon energy distribution for the CRM
model. At present, only one type is supported. 0 for power law.
\item[\texttt{int np\_pdist}:] Number of parameters for the photon energy
distribution.
\item[\texttt{double *p\_pdist}:] Parameters for the photon energy
distribution. The following table describes the parameters for exsisting
distributions:

\hspace{0.6in}\centerline{
\begin{minipage}{6in}
\index{Photon Energy Distribution}
\begin{tabular}{|c|c|ccc|}
\hline
ID& Dist& \multicolumn{2}{c}{Params}& Unit\\
\hline
0& Black Body & 0:& $T$ & eV\\
 &           & 1:& $E_{min}$ & eV\\
 &           & 2:& $E_{max}$ & eV\\
1& PowerLaw& 0:& $\alpha$ & \\
 &           & 1:& $E_{min}$ & eV\\
 &           & 2:& $E_{max}$ & eV\\
\hline
\end{tabular}
\end{minipage}}
\item[\texttt{float pden}:] The photon energy density of the radiation field in
unit of erg cm$^{-3}$. If the distribution is Black Body, this is the dilution
factor.
\end{dbdesc}

\subsection{\texttt{RT\_RECORD}}
\index{RT\_RECORD}
\begin{verbatim}
typedef struct _RT_RECORD_ {
  int dir;
  int iblock;
  float nb;
  float tr;
  float ce;
  float rr;
  float ai;
  float ci;
  char icomplex[LNCOMPLEX];
} RT_RECORD;
\end{verbatim}

\begin{dbdesc}
\item[\texttt{int dir}:] The direction of the transition. If \key{dir} is 0,
  the rate is into this superlevel, if \key{dir} is 1, the rate is out of this
  superlevel.
\item[\texttt{int iblock}:] The index of the superlevel block from which the
rates originate.
\item[\texttt{float nb}:] The concentration of the superlevel block from which
the rates originate.
\item[\texttt{float tr}:] The radiative transition rates.
\item[\texttt{float ce}:] The collisional excitation rates.
\item[\texttt{float rr}:] The radiative recombination and possible
photoionization rates (if there is radiation field included in the model).
\item[\texttt{float ai}:] The autoionization rates and dielectronic capture
rates.
\item[\texttt{float ci}:] The collisional ionization rates.
\item[\texttt{char icomplex[LNCOMPLEX]}:] The configuration complex name of
the superlevel block from which the rates originate.
\end{dbdesc}

\subsection{\texttt{DR\_HEADER}}
\index{DR\_HEADER}
\begin{verbatim}
typedef struct _DR_HEADER_ {
  long int position;
  long int length;
  int nele;
  int ilev;
  int ntransitions;
  int vn;
  int j;
  float energy;
} DR_HEADER;
\end{verbatim}

\begin{dbdesc}
\item[\texttt{long position}:] The number of bytes from the beginning of the
file to the place where this data block starts.
\item[\texttt{long length}:] Number of bytes in this data block, excluding the
length of the header.
\item[\texttt{int nele}:] Number of electrons in the recombining ion for this
block.
\item[\texttt{int ilev}:] The level index of the recombining ion for this
block.
\item[\texttt{int ntransitions}:] The number of superlevel blocks from which
the rates are tabulated in this block.
\item[\texttt{int vn}:] The principle quantum number of the captured electron.
\item[\texttt{int j}:] The $2j$ value of the recombining state.
\item[\texttt{float energy}:] The energy of the recombining state.
\end{dbdesc}

\subsection{\texttt{DR\_RECORD}}
\index{DR\_RECORD}
\begin{verbatim}
typedef struct _DR_RECORD_ {
  int ilev;
  int flev;
  int ibase;
  int fbase;
  int vl;
  int j;
  float energy;
  float etrans;
  float br;
  float ai;
  float total_rate;
} DR_RECORD;
\end{verbatim}

\begin{dbdesc}
\item[\texttt{int ilev}:] The level index of the autoionizing state.
\item[\texttt{int flev}:] The level index of the final state due to the decay
(either autoionization or radiative) of the autoionizing state. For the total
DR strength, this variable is always -1, since the result is the sum of all
radiative stabilazation routes.
\item[\texttt{int ibase}:] The level index of the core for the resonance
state.
\item[\texttt{int fbase}:] The level index of the core for the final
state, if applicable.
\item[\texttt{int vl}:] The orbital angular momentum of the captured
electron.
\item[\texttt{int j}:] The $2j$ value of the autoionizaing state.
\item[\texttt{float energy}:] The energy of the autoionizing state relative to
the recombining state, whose energy is given in the \texttt{DR\_HEADER}.
\item[\texttt{float etrans}:] The transition energy from \texttt{ilev} to
  \texttt{flev}.
\item[\texttt{float br}:] The branching ratio of the autoionizing state to
various final states. For total DR strength, this is the total radiative
branching ratio. For resonance excitation, this is the branching ratio of
autoionization to individual states. For satellite calculations, this is the
braching ratio of radiative transition to individual states.
\item[\texttt{float ai}:] The autoionization rate to the recombining level.
\item[\texttt{float total\_rate}:] The total decay rate of the autoionization
state, i.e., including all autoionization and radiative decay channels.
\end{dbdesc}

\subsection{\texttt{AIM\_HEADER}}
\index{AIM\_HEADER}
\begin{verbatim}
typedef struct _AIM_HEADER_ {
  long int position;
  long int length;
  int nele;
  int ntransitions;
  int channel;
  int n_egrid;
  double *egrid;
} AIM_HEADER;
\end{verbatim}

\begin{dbdesc}
\item[\texttt{long position}:] The number of bytes from the beginning of the
file to the place where this data block starts.
\item[\texttt{long length}:] Number of bytes in this data block, excluding the
length of the header.
\item[\texttt{int nele}:] Number of electrons in the ion for this block.
\item[\texttt{int ntransitions}:] Number of transitions in this block.
\item[\texttt{int channel}:] This an identifier to label the autoionization
channel, which does not have specific physical meaning.
\item[\texttt{int n\_egrid}:] The number of points for the Auger electron
energy grid. The autoionzation radial integrals are calculated on this grid
and interpolated to the actual discrete energies.
\item[\texttt{double *egrid}:] The energy grid. The number of elements is
given by \texttt{n\_egrid}.
\end{dbdesc}

\subsection{\texttt{AIM\_RECORD}}
\index{AIM\_RECORD}
\begin{verbatim}
typedef struct _AIM_RECORD_ {
  int b;
  int f;
  int nsub;
  float *rate;
} AIM_RECORD;
\end{verbatim}
\begin{dbdesc}
\item[\texttt{int b}:] The bound state index.
\item[\texttt{int f}:] The free state index.
\item[\texttt{int nsub}:] The number of entries in array \texttt{rate}. It is
twice the number of transitions tabulated, because both autoionziation and DR
capture strength need to be stored.
\item[\texttt{float *rate}:] An array containing magnetic sublevel
autoionization rates and DR capture strength. Suppose $A(M_1,M_2)$ represents
autoionization rate from $M_1$ sublevel of \texttt{b} to $M_2$ sublevel of
\texttt{f}, and $R(M_1,M_2)$ is capture strength from $M_2$ sublevel of
\texttt{f} to $M_1$ sublevel of \texttt{b}. Then the array \texttt{rate} is
generated in the order consistent with the following code:
\begin{verbatim}
t = 0
for (M1 = -J1; M1 <= 0; M1++) {
  for (M2 = -J2; M2 <= J2; M2++) {
     rate[t++] = A(M1,M2);
     rate[t++] = R(M1,M2);
  }
}
\end{verbatim}
Due to axial symmetry, values for $M_1 > 0$ can be obtained from those with
$M_1 \le 0$.
\end{dbdesc}

\subsection{\texttt{CIM\_HEADER}}
\index{CIM\_HEADER}
\texttt{CIM\_HEADER} is the data header for magnetic sublevel collisional
ionization data blocks.

\begin{verbatim}
typedef struct _CIM_HEADER_ {
  long position;
  long length;
  int nele;
  int ntransitions;
  int n_egrid;
  int egrid_type;
  int n_usr;
  int usr_egrid_type;
  double *egrid;
  double *usr_egrid;
} CIM_HEADER;
\end{verbatim}

\begin{dbdesc}
\item[\texttt{long position}:] The number of bytes from the beginning of the
file to the place where this data block starts.
\item[\texttt{long length}:] Number of bytes in this data block, excluding the
length of the header.
\item[\texttt{int nele}:] Number of electrons in the ion for this block.
\item[\texttt{int ntransitions}:] Number of transitions in this block.
\item[\texttt{int n\_egrid}:] Number of points for the collision energy grid.
\item[\texttt{int egrid\_type}:] Type of the energy grid. 0 for the incident
electron energy, 1 for the total energy of scattered and ejected electron.
\item[\texttt{int n\_usr}:]N umber of points for the user collision energy
grid.
\item[\texttt{int usr\_egrid\_type}:] Type of the user energy grid. 0 for the
incident electron energy, 1 for the total energy of scattered and ejected
electrons .
\item[\texttt{double *tegrid}:] The transition energy grid, the number of
elements is given by \texttt{n\_tegrid}.
\item[\texttt{double *egrid}:] The energy grid, the number of elements is
given by \texttt{n\_egrid}.
\item[\texttt{double *usr\_egrid}:] The user energy grid, the number of
elements is given by \texttt{n\_usr}.
\end{dbdesc}

\subsection{\texttt{CIM\_RECORD}}
\index{CIM\_RECORD}
\texttt{CIM\_RECORD} is for magnetic sublevel collisional ionization data.

\begin{verbatim}
typedef struct _CIM_RECORD_ {
  int b;
  int f;
  int nsub;
  float *strength;
} CIM_RECORD;
\end{verbatim}

\begin{dbdesc}
\item[\texttt{int b}:] The bound state index.
\item[\texttt{int f}:] The free state index.
\item[\texttt{int nusb}:] The number of sublevel transitions in the
  \texttt{strength} array.
\item[\texttt{float *strength}:] The magnetic sublevel collision strength for
  ionization. It is related to the ionization cross section as (in atomic
  units):
\begin{equation}
\sigma = \frac{1}{k_0^2}\Omega,
\end{equation}
where ${k_0}$ is the kinetic momentum of the incident electron. The different
magnetic sublevel transitions are arranged in the order similar to those for
excitation, i.e., $-J_i\to -J_f$, $-J_i\to -J_f+1$, $\cdots$, $-J_i\to J_f$,
$-J_i+1\to -J_f$, $-J_i+1\to -J_f+1$, $\cdots$, $-J_i+1\to J_f$,
$\cdots$. Only the cross sections with $M_i \le 0$ are included, since those
with $M_i \ge 0$ can be obtained using time reversal symmetry.

\end{dbdesc}

\subsection{\texttt{RO\_HEADER}}
\index{RO\_HEADER}
\texttt{RO\_HEADER} is the data header for recombination orbital occupancy
file. This file tabulates the weight of each recombining orbital for a given
bound-free channel.

\begin{verbatim}
typedef struct _RO_HEADER_ {
  long position;
  long length;
  int nele;
  int ntransitions;
} RO_HEADER;
\end{verbatim}

\begin{dbdesc}
\item[\texttt{long position}:] The number of bytes from the beginning of the
file to the place where this data block starts.
\item[\texttt{long length}:] Number of bytes in this data block, excluding the
length of the header.
\item[\texttt{int nele}:] Number of electrons in the ion for this block.
\item[\texttt{int ntransitions}:] Number of transitions in this block.
\end{dbdesc}

\subsection{\texttt{RO\_RECORD}}
\index{RO\_RECORD}
\texttt{RO\_RECORD} is for recombination orbital occupancy data. The weight
factors for each orbital can be used to convert Hydrogenic recombination or
ionization cross section into those between fine-structure levels.

\begin{verbatim}
typedef struct _RO_RECORD_ {
  int b;
  int f;
  int n;
  int *nk;
  double *nq;
  double *dn;
} CIM_ROCORD;
\end{verbatim}

\begin{dbdesc}
\item[\texttt{int b}:] The bound state index.
\item[\texttt{int f}:] The free state index.
\item[\texttt{int n}:] Number of recombining orbitals for this channel.
\item[\texttt{int *nk}:] Decompose to $+/-(n\times100+k)$, where n is the
  principle 
  quantum number of orbital, k is the orbital quantum number. if positive,
  the total angular momentum is $k+\frac{1}{2}$, if negative the total angular
  momentum is $k-\frac{1}{2}$.
\item[\texttt{double *nq}:] The weight of each recombining orbital.
\item[\texttt{double *dn}:] The quantum defect of the recombining orbital
  
\end{dbdesc}
\subsection{\texttt{CX\_HEADER}}
\index{CX\_HEADER}
\texttt{CX\_HEADER} is the data header for charge exchange cross section data.

\begin{verbatim}
typedef struct _CX_HEADER_ {
  long position;
  long length;
  int nele;
  int ntransitions;
  char tgts[128];
  double tgtz;
  double tgtm;
  double tgta;
  double tgtb;
  double tgte;
  double tgtx;
  int ldist;
  int te0;
  int ne0;
  double *e0;
} CX_HEADER;
\end{verbatim}

\begin{dbdesc}
\item[\texttt{long position}:] The number of bytes from the beginning of the
file to the place where this data block starts.
\item[\texttt{long length}:] Number of bytes in this data block, excluding the
length of the header.
\item[\texttt{int nele}:] Number of electrons in the ion for this block.
\item[\texttt{int ntransitions}:] Number of transitions in this block.
\item[\texttt{char tgts[128]}:] The chemical symbol of the neutral target.
\item[\texttt{double tgtz}:] The total nuclear charge of the target.
\item[\texttt{double tgtm}:] The mass of target in AMU.
\item[\texttt{double tgta}:] The static dipole polarizability of the target.
\item[\texttt{double tgtb}:] The screening parameter of the short-range
  inter-nucleus potention.
\item[\texttt{double tgte}:] The ionization potential of the target.
\item[\texttt{double tgtx}:] A parameter to soften the polarization potential
  at very short distrance.
\item[\texttt{int ldist}:] $l$-distribution type applied to obtain
  $l$-resolved cross sections. 
\item[\texttt{int te0}:] Indicate the unit of the collision energy, 0 for per
  AMU, 1 for per ion.
\item[\texttt{int ne0}:] Number of collision energy points.
\item[\texttt{double *e0}:] The collision energy grid.
\end{dbdesc}

\subsection{\texttt{RO\_RECORD}}
\index{CX\_RECORD}
\texttt{CX\_RECORD} is for charge exchange cross section data.

\begin{verbatim}
typedef struct _RO_RECORD_ {
  int b;
  int f;
  int n;
  int vnl;
  double *cx;
} CIM_ROCORD;
\end{verbatim}

\begin{dbdesc}
\item[\texttt{int b}:] The bound state index.
\item[\texttt{int f}:] The free state index.
\item[\texttt{int vnl}:] Decompose to $n\times 100+k$, where n is the
  principle quantum number of orbital, k is the orbital quantum number of the
  captured electron.
\item[\texttt{double *cx}:] The cross sections on the energy grid defined in
  the \texttt{CX\_HEADER}.
  
\end{dbdesc}

\section{ASCII Format}
\index{ASCII Format}
FAC provides functions to convert the binary output to ASCII files. There are
two types of ASCII formats, a simple translation of binary files and a more
verbose version that adds more derived information for the sake of
convenience. If the ASCII files are created to be human-readable, the
verbose form should be used.

In the simple form, the contents of binary files are converted to ASCII format
as is. No additional information is added. All physical values are in atomic
units as is in binary files. The different byte-order used by different
platforms are taken into account automatically. Therefore, it is possible to
create the binary files on a little endian machine (probably faster), then
convert them to ASCII format on a slower big endian machine.

In the verbose form, the more common units of physical quanties are
used. Specifically, s$^{-1}$ for transition rates, 10$^{-20}$ cm$^2$ for cross
sections, and eV for energies. For data files other than \texttt{DB\_EN} type,
the energies and angular momenta of the levels involved in the processes are
not included in the binary version. In the verbose form of corresponding ASCII
files, these infomations are added by looking up in the energy level
table. Also, for \texttt{DB\_TR} files, not only matrix elements, but also
$gf$ values and radiative transition rates are tabulated. For \texttt{DB\_CE}
and \texttt{DB\_CI}, cross
sections are tabulated along with the collision strengths. For \texttt{DB\_RR},
radiative recombination and photionization cross sections are tabulated along
with the bound-free differential $gf$ values. For \texttt{DB\_AI}, the energy
integrated dielectronic capture strengths (in unit of 10$^{-20}$ eV cm$^2$)
are tabulated in addition to the autoionization rates.

In the following sections, a portion of each type of database file in the
verbose form is listed and significant fields explained. The lines start with
a ``\verb|#|'' are the added explanation, which are not part of the output
file. These files are generated with the scripts in the \texttt{demo/}
directory come with FAC.

\subsection{\texttt{DB\_EN}}
\index{DB\_EN}
\begin{verbatim}
# version numbers
FAC 1.0.4
# binary order used in the binary file
Endian	= 0
# time stamp when the file was created.
TSess	= 1020438482
# database type
Type	= 1
# this file is in verbose form
Verbose	= 1
# atomic symbol and atomic number
Fe Z	= 26.0
# number of data blocks in this file
NBlocks	= 1
# the index and the absolute energy of the ground state
E0	= 0, -3.12494784E+04

# data block begins
# number of electron for the states in this block
NELE	= 10
# number of levels in this block
NLEV	= 37
  ILEV  IBASE    ENERGY       P   VNL 2J
     0     -1  0.00000000E+00 0   201  0 1*2 2*8      2p6      2p+4(0)0
     1     -1  7.23810448E+02 1   300  4 1*2 2*7 3*1  2p5 3s1  2p+3(3)3 3s+1(1)4
     2     -1  7.25859655E+02 1   300  2 1*2 2*7 3*1  2p5 3s1  2p+3(3)3 3s+1(1)2
     3     -1  7.36414516E+02 1   300  0 1*2 2*7 3*1  2p5 3s1  2p-1(1)1 3s+1(1)0
     4     -1  7.37736604E+02 1   300  2 1*2 2*7 3*1  2p5 3s1  2p-1(1)1 3s+1(1)2
     5     -1  7.54149163E+02 0   301  2 1*2 2*7 3*1  2p5 3p1  2p+3(3)3 3p-1(1)2
     6     -1  7.57788593E+02 0   301  4 1*2 2*7 3*1  2p5 3p1  2p+3(3)3 3p-1(1)4
     7     -1  7.59341459E+02 0   301  6 1*2 2*7 3*1  2p5 3p1  2p+3(3)3 3p+1(3)6
     8     -1  7.60552577E+02 0   301  2 1*2 2*7 3*1  2p5 3p1  2p+3(3)3 3p+1(3)2
     9     -1  7.62361512E+02 0   301  4 1*2 2*7 3*1  2p5 3p1  2p+3(3)3 3p+1(3)4
    10     -1  7.67999430E+02 0   301  0 1*2 2*7 3*1  2p5 3p1  2p+3(3)3 3p+1(3)0
   ......
\end{verbatim}

The column labels by \verb|VNL| is $100\times n + l$, where $n$ and $l$
are the principle and orbital angular quantum numbers of the valence
electron.

\subsection{\texttt{DB\_TR}}
\index{DB\_TR}
\begin{verbatim}
FAC 1.0.7
Endian	= 0
TSess	= 1021577025
Type	= 2
Verbose	= 1
Fe Z	=  26.0
NBlocks	= 1

# the data block begins
NELE	= 10
# number of transitions in this block
NTRANS	= 7
# multipole type of the transition
Multip	= -1
# gauge used in the calculation
Gauge	= 2
# mode used in the radial integral
Mode	= 1
#upper 2J  lower 2J  Delta E     gf            TR rate(1/s)  multipole
     2  2      0  0  7.2587E+02  1.130597E-01  8.616084E+11  1.127617E-01
     4  2      0  0  7.3774E+02  9.944485E-02  7.828559E+11  1.048997E-01
    16  2      0  0  8.0114E+02  9.438239E-03  8.761793E+10 -3.101188E-02
    22  2      0  0  8.1133E+02  6.221187E-01  5.923155E+12 -2.501928E-01
    26  2      0  0  8.2527E+02  2.493449E+00  2.456309E+13  4.966355E-01
    30  2      0  0  8.9415E+02  3.203146E-02  3.704097E+11  5.407792E-02
    32  2      0  0  8.9844E+02  2.652003E-01  3.096259E+12 -1.552313E-01
\end{verbatim}

After version 1.0.8, if the UTA mode is used, the output contains an
additional column after the transition energy, which is the Gaussian standard
deviation of the UTA transition. The $2J$ values in this case are also
redefined to be the statistical weight of the configuraiton minus 1.

\subsection{\texttt{DB\_CE}}
\index{DB\_CE}
\begin{verbatim}
FAC 0.7.9
Endian	= 0
TSess	= 1021577097
Type	= 3
Verbose	= 1
Fe Z	=  26.0
NBlocks	= 1

# data blocks begin
NELE	= 10
NTRANS	= 36
# mode used in the radial integral
QKMODE	= 0
# number of parameters in the fitting formula (only if QKMODE = 2)
NPARAMS	= 0
# 0 for total collision strength. 1 for magnetic sublevel.
MSUB	= 0
# partial wave summation mode. always 0.
PWTYPE	= 0
# number of points in the transition energy grid, followed by the grid
NTEGRID	= 2
	  7.24352072E+02
	  9.45773957E+02
# characteristic transition energy used in grid construction.
TE0	=  9.44829120E+02
# energy grid type.
ETYPE	= 1
# energy grid
NEGRID	= 6
	  4.72414560E+01
	  5.79761386E+02
	  1.39812537E+03
	  2.65576771E+03
	  4.58848260E+03
	  7.55863296E+03
# user energy grid type and the user grid.
UTYPE	= 1
NUSR	= 6
	  4.72414560E+01
	  5.79761386E+02
	  1.39812537E+03
	  2.65576771E+03
	  4.58848260E+03
	  7.55863296E+03
#lower 2J upper 2J Delta E  nsub
    0	 0	    1	 4	 7.2435E+02	1
#The Bethe coefficient and 2 Born coefficients in the Born approximation.
-1.0000E+00  0.0000E+00  0.0000E+00
# if QKMODE = 2, the parameter line is present here.
#user egrid  coll. str.  cross sec.
 4.7241E+01	 1.5347E-03	 2.3789E-01
 5.7976E+02	 9.5137E-04	 8.7207E-02
 1.3981E+03	 5.1906E-04	 2.9211E-02
 2.6558E+03	 2.5016E-04	 8.8294E-03
 4.5885E+03	 1.1322E-04	 2.5376E-03
 7.5586E+03	 5.1291E-05	 7.3523E-04
    0	 0	    2	 2	 7.2639E+02	1
 9.1750E-03 -7.0392E-03  7.2655E-03
 4.7241E+01	 1.8857E-03	 2.9153E-01
 5.7976E+02	 3.7280E-03	 3.4119E-01
 1.3981E+03	 6.3378E-03	 3.5633E-01
 2.6558E+03	 9.4893E-03	 3.3472E-01
 4.5885E+03	 1.2996E-02	 2.9116E-01
 7.5586E+03	 1.6729E-02	 2.3974E-01
 ......
\end{verbatim}

If \texttt{MSUB} = 1, then the data for each transition contains \texttt{nsub}
blocks, representing several $m_i\to m_f$ transitions. Before each block, the
ratio of the magnetic sublevel collision strengths to the total collision
strength at high energy limit is given. Due to the time
reversal symmetry, the cross section for $-m_i \to -m_f$ is the same as that
for $m_i \to m_j$, only the cross sections with $m_i \le 0$ are tabulated in
the order $-J_i\to -J_f$, $-J_i\to -J_f+1$, $\cdots$, $-J_i\to J_f$,
$-J_i+1\to -J_f$, $-J_i+1\to -J_f+1$, $\cdots$, $-J_i+1\to J_f$, $\cdots$.


\subsection{\texttt{DB\_RR}}
\index{DB\_RR}
\begin{verbatim}
FAC 0.7.3
Endian	= 0
TSess	= 1021577047
Type	= 4
Verbose	= 1
Fe Z	=  26.0
NBlocks	= 1

# the data blocks begin
NELE	= 3
NTRANS	= 3
QKMODE	= 2
# multipole type
MULTIP	= -1
# number of parameters in the fitting formula
NPARAMS	= 4
NTEGRID	= 1
	  2.01377924E+03
ETYPE	= 1
NEGRID	= 6
	  1.00688962E+02
	  1.23568529E+03
	  2.97992069E+03
	  5.66042025E+03
	  9.77974833E+03
	  1.61102339E+04
UTYPE	= 1
NUSR	= 6
	  1.00688962E+02
	  1.23568529E+03
	  2.97992069E+03
	  5.66042025E+03
	  9.77974833E+03
	  1.61102339E+04
#bound 2J  free 2J  Delta E    L
    7	 1	    0	 0	 2.0465E+03	 0
# the parameters in the fitting formula
 3.8124E-02  4.9724E+00  1.2195E+00  2.1768E+03
#user egrid RR cross sec. PI cross sec. gf
 1.0069E+02	 1.7567E-01	 1.9604E+00	 3.0537E-02
 1.2357E+03	 1.4375E-02	 8.4255E-01	 1.3124E-02
 2.9799E+03	 5.5123E-03	 3.3223E-01	 5.1751E-03
 5.6604E+03	 2.4847E-03	 1.2100E-01	 1.8848E-03
 9.7797E+03	 1.1476E-03	 4.1004E-02	 6.3872E-04
 1.6110E+04	 5.2175E-04	 1.3029E-02	 2.0295E-04
    8	 1	    0	 0	 1.9975E+03	 1
 3.4223E-02  5.3145E+00  1.2206E+00  2.1537E+03
 1.0069E+02	 1.5214E-01	 1.7781E+00	 2.7697E-02
 1.2357E+03	 8.4472E-03	 5.1024E-01	 7.9480E-03
 2.9799E+03	 2.1814E-03	 1.3407E-01	 2.0885E-03
 5.6604E+03	 6.5910E-04	 3.2509E-02	 5.0639E-04
 9.7797E+03	 2.0448E-04	 7.3672E-03	 1.1476E-04
 1.6110E+04	 6.2231E-05	 1.5624E-03	 2.4338E-05
    9	 3	    0	 0	 1.9810E+03	 1
 6.9754E-02  5.1620E+00  1.2206E+00  2.1350E+03
 1.0069E+02	 2.9691E-01	 1.7625E+00	 5.4910E-02
 1.2357E+03	 1.6320E-02	 4.9796E-01	 1.5513E-02
 2.9799E+03	 4.1721E-03	 1.2907E-01	 4.0209E-03
 5.6604E+03	 1.2475E-03	 3.0899E-02	 9.6262E-04
 9.7797E+03	 3.8274E-04	 6.9143E-03	 2.1541E-04
 1.6110E+04	 1.1506E-04	 1.4471E-03	 4.5081E-05
\end{verbatim}

\subsection{\texttt{DB\_AI}}
\index{DB\_AI}
\begin{verbatim}
FAC 0.7.3
Endian	= 0
TSess	= 1021577153
Type	= 5
Verbose	= 1
Se Z	=  34.0
NBlocks	= 1

# data blocks begin
NELE	= 10
# number of transitions
NTRANS	= 92
# channel number (no physical meaning)
CHANNE	= 0
# free electron energy grid
NEGRID	= 2
	  3.43213508E+02
	  5.58605513E+02
#bound 2J free 2J  Delta E     AI rate    DC strength
    2	 4    0	 3	 3.8679E+02	 1.3705E+13	 1.0962E+01
    2	 4    1	 1	 3.4322E+02	 1.6996E+11	 3.0642E-01
    3	 0    0	 3	 4.0594E+02	 1.2973E+13	 1.9775E+00
    3	 0    1	 1	 3.6236E+02	 7.9903E+11	 2.7289E-01
    4	 4    0	 3	 4.1845E+02	 2.3652E+11	 1.7487E-01
    4	 4    1	 1	 3.7487E+02	 2.2905E+09	 3.7807E-03
    ......
\end{verbatim}

\subsection{\texttt{DB\_CI}}
\index{DB\_CI}
\begin{verbatim}
FAC 0.7.3
Endian	= 0
TSess	= 1021577194
Type	= 6
Verbose	= 1
Fe Z	=  26.0
NBlocks	= 1

# data blocks begin
NELE	= 10
NTRANS	= 3
QKMODE	= 5
NPARAMS	= 4
PWTYPE	= 0
NTEGRID	= 2
	  1.26072567E+03
	  1.39546596E+03
ETYPE	= 1
NEGRID	= 6
	  6.64047908E+01
	  8.14939607E+02
	  1.96527014E+03
	  3.73307079E+03
	  6.44978485E+03
	  1.06247665E+04
UTYPE	= 1
NUSR	= 8
	  5.00000000E+02
	  9.00000000E+02
	  1.30000000E+03
	  1.70000000E+03
	  2.10000000E+03
	  4.20000000E+03
	  6.00000000E+03
	  8.00000000E+03
#bound 2J  free 2J  Delta E    L
    0	 0	    1	 3	 1.2607E+03	 1
# parameters in the fitting formula
 1.2549E-01  6.7308E-01 -5.4651E-01  7.2856E-01
#user egrid  coll. str.  cross sec.
 5.0000E+02	 9.0588E-02	 1.9535E+00
 9.0000E+02	 1.5721E-01	 2.7605E+00
 1.3000E+03	 2.1672E-01	 3.2084E+00
 1.7000E+03	 2.6991E-01	 3.4532E+00
 2.1000E+03	 3.1773E-01	 3.5785E+00
 4.2000E+03	 5.0773E-01	 3.5050E+00
 6.0000E+03	 6.1976E-01	 3.2065E+00
 8.0000E+03	 7.1292E-01	 2.8808E+00
    0	 0	    2	 1	 1.2737E+03	 1
 6.3179E-02  3.3088E-01 -2.6847E-01  3.5591E-01
 5.0000E+02	 4.4336E-02	 9.4904E-01
 9.0000E+02	 7.7083E-02	 1.3454E+00
 1.3000E+03	 1.0639E-01	 1.5671E+00
 1.7000E+03	 1.3263E-01	 1.6894E+00
 2.1000E+03	 1.5624E-01	 1.7529E+00
 4.2000E+03	 2.5023E-01	 1.7233E+00
 6.0000E+03	 3.0577E-01	 1.5791E+00
 8.0000E+03	 3.5203E-01	 1.4205E+00
    0	 0	    3	 1	 1.3955E+03	 0
 5.7526E-02  2.8628E-01 -2.4531E-01  3.1267E-01
 5.0000E+02	 3.4409E-02	 6.8908E-01
 9.0000E+02	 6.0280E-02	 9.9604E-01
 1.3000E+03	 8.4082E-02	 1.1823E+00
 1.7000E+03	 1.0585E-01	 1.2950E+00
 2.1000E+03	 1.2576E-01	 1.3615E+00
 4.2000E+03	 2.0705E-01	 1.3946E+00
 6.0000E+03	 2.5609E-01	 1.3005E+00
 8.0000E+03	 2.9733E-01	 1.1840E+00
\end{verbatim}

\subsection{\texttt{DB\_SP}}
\index{DB\_SP}
\begin{verbatim}
FAC 0.7.3
Endian	= 0
TSess	= 1021484432
Type	= 7
Verbose	= 1
Fe Z	=  26.0
NBlocks	= 2952

# data blocks begin
NELE	= 10
NTRANS	= 1
# type 0 is for level population
TYPE	= 000000
# block number
IBLK	= 5
# block complex
ICOMP	= 1*2 2*8
FBLK	= 0
FCOMP	=
#block level  abs. energy      population
    0	    0	-3.12496016E+04	 9.99987960E-01

NELE	= 10
NTRANS	= 1
TYPE	= 000000
IBLK	= 6
ICOMP	= 1*2 2*7 3*1
FBLK	= 0
FCOMP	=
    1	    0	-3.05250918E+04	 5.11909866E-06

NELE	= 10
NTRANS	= 1
TYPE	= 000000
IBLK	= 7
ICOMP	= 1*2 2*7 3*1
FBLK	= 0
FCOMP	=
    2	    0	-3.05230508E+04	 6.65310797E-13

NELE	= 10
NTRANS	= 1
TYPE	= 000000
IBLK	= 8
ICOMP	= 1*2 2*7 3*1
FBLK	= 0
FCOMP	=
    3	    0	-3.05122168E+04	 6.89876970E-06

 ......

# this block is for line emissivity
NELE	= 10
NTRANS	= 113
# transition type 16->4 transition
TYPE	= 001604
# initial block
IBLK	= 54
# initial block complex
ICOMP	= 1*2 2*7 16*1
# final block
FBLK	= 42
# final block complex
FCOMP	= 1*2 2*7 4*1
#upper lower  Delta E         emissivity
 1831	  158	 2.55395828E+02	 6.58072258E-06
 1831	  163	 2.48938431E+02	 2.54685597E-06
 1831	  164	 2.54269775E+02	 7.99868485E-06
 1831	  165	 2.53695114E+02	 6.73144177E-06
 1832	  157	 2.56149841E+02	 1.01010974E-05
 1832	  158	 2.55389191E+02	 8.33697595E-06
 1832	  165	 2.53688477E+02	 8.52791436E-06
 1832	  166	 2.54721527E+02	 2.41980142E-05
 1833	  160	 2.55734634E+02	 5.87857039E-06
 1833	  167	 2.54585327E+02	 1.21048633E-05
 1834	  159	 2.52442841E+02	 4.88687328E-06
 1834	  160	 2.55741287E+02	 1.01619125E-05
 1834	  167	 2.54591980E+02	 5.23122890E-06
 ......
\end{verbatim}

\subsection{\texttt{DB\_RT}}
\index{DB\_RT}
\begin{verbatim}
FAC 1.1.0
Endian	= 0
TSess	= 1021484432
Type	= 8
Verbose	= 1
Fe Z	=  26.0
NBlocks	= 87

# data blocks begin
NELE	= 9
NTRANS	= 49
# block index
IBLK	= 0
# level index within the ion
ILEV	= 0
ICOMP	= 1*2 2*7
# electron density
EDEN	=  1.00000000E+00
# electron energy distribution, 0 for Maxwellian.
EDIST	= 0
# parameters for electron energy distribution
NPEDIS	= 3
# temperature
	  1.00000000E+00
# Emin
	  1.00000000E-20
# Emax
	  1.00000000E+02
# photon energy density
PDEN	=  0.00000000E+00
# photon energy distribution
PDIST	= 0
NPPDIS	= 3
	  2.00000000E+00
	  1.00000000E+01
	  1.00000000E+05
# population of this block
DENS	=  1.00000000E+00
# statistical weight of this block
STWT     =  4.00000000E+00
# each line is contribution to this level or block by other blocks
         NB         TR         CE         RR         AI         CI
   1  2.9983E-10 6.1831E-06 0.0000E+00 0.0000E+00 0.0000E+00 0.0000E+00 1*2 2*7
  44  6.0874E-12 0.0000E+00 0.0000E+00 0.0000E+00 8.8866E-04 0.0000E+00 1*2 2*7 6*1
  45  1.0708E-11 0.0000E+00 0.0000E+00 0.0000E+00 1.6693E-18 0.0000E+00 1*2 2*7 7*1
  46  1.7088E-11 0.0000E+00 0.0000E+00 0.0000E+00 1.0329E-27 2.0448E-37 1*2 2*7 8*1
  47  2.5457E-11 0.0000E+00 0.0000E+00 0.0000E+00 7.2230E-34 1.9952E-31 1*2 2*7 9*1
  48  3.1069E-11 0.0000E+00 0.0000E+00 0.0000E+00 3.3995E-38 3.8637E-27 1*2 2*7 10*1
  49  1.2877E-10 0.0000E+00 0.0000E+00 0.0000E+00 4.1495E-41 2.8210E-24 1*2 2*7 11*1
  50  5.2642E-11 0.0000E+00 0.0000E+00 0.0000E+00 0.0000E+00 8.2326E-22 1*2 2*7 12*1
  51  5.5315E-11 0.0000E+00 0.0000E+00 0.0000E+00 0.0000E+00 9.8721E-20 1*2 2*7 13*1
  52  6.3140E-11 0.0000E+00 0.0000E+00 0.0000E+00 0.0000E+00 4.8804E-18 1*2 2*7 14*1
  53  7.3613E-11 0.0000E+00 0.0000E+00 0.0000E+00 0.0000E+00 1.2070E-16 1*2 2*7 15*1
  54  8.4200E-11 0.0000E+00 0.0000E+00 0.0000E+00 0.0000E+00 1.7049E-15 1*2 2*7 16*1
  55  9.3432E-11 0.0000E+00 0.0000E+00 0.0000E+00 0.0000E+00 1.5542E-14 1*2 2*7 17*1
  56  1.5446E-10 0.0000E+00 0.0000E+00 0.0000E+00 8.3172E+00 1.0338E-13 1*2 2*7 18*1
  57  1.2547E-10 0.0000E+00 0.0000E+00 0.0000E+00 1.9623E+00 5.2262E-13 1*2 2*7 19*1
  58  1.2279E-10 0.0000E+00 0.0000E+00 0.0000E+00 5.6143E-01 2.1193E-12 1*2 2*7 20*1
  ......
  -1  0.0000E+00 0.0000E+00 0.0000E+00 0.0000E+00 0.0000E+00 0.0000E+00   8
  -2  1.0000E+00 6.1831E-06 0.0000E+00 0.0000E+00 1.1178E+01 0.0000E+00   9
  -3  1.0000E+00 0.0000E+00 0.0000E+00 0.0000E+00 0.0000E+00 1.2304E-04  10
  ......
\end{verbatim}

\subsection{\texttt{DB\_DR}}
\index{\texttt{DB\_DR}}
\begin{verbatim}
FAC 0.8.6
Endian	= 0
TSess	= 1036783874
Type	= 9
Verbose	= 1
Fe Z	=  26.0
NBlocks	= 45

# number of electrons of the recombining ion.
NELE	= 9
# number of transitions in this block.
NTRANS	= 22
# level index of the recombining state.
ILEV	= 37
# energy of the recombining state.
E	= -2.99679153E+04
# 2J value of the recombing state.
JLEV	= 3
# principle quantum number of the captured electron.
NREC	= 6
#AI 2J Rec 2J ibase flev fbase NREC L ERes       ETrans     AI Rate     Total Rate  Branching
275  0  37  3    39   -1    -1    6 0 1.2462E+01 0.0000E+00 1.1031E+13  1.1348E+13  2.7940E-02
......
\end{verbatim}

\subsection{\texttt{DB\_AIM}}
\index{\texttt{DB\_AIM}}
\begin{verbatim}
FAC 1.0.0
Endian	= 0
TSess	= 1059655031
Type	= 10
Verbose	= 1
Fe Z	=  26.0
NBlocks	= 1

# data blocks begin
NELE	= 4
# number of transitions
NTRANS	= 87
# channel number (no physical meaning)
CHANNE	= 0
# free electron energy grid
NEGRID	= 1
	  4.70510700E+03
#bound 2J free 2J Delta E     nsub
   13	 0   10	 1	 4.6448E+03	 4
#AI rates    DC strength
 3.5842E+13	 1.9099E+00
 3.5842E+13	 1.9099E+00
   13	 0   11	 1	 4.5958E+03	 4
 5.7277E+13	 3.0847E+00
 5.7277E+13	 3.0847E+00
 ......
\end{verbatim}

\subsection{\texttt{DB\_CIM}}
\index{\texttt{DB\_CIM}}
\begin{verbatim}
FAC 1.0.5
Endian	= 1
TSess	= 1077996307
Type	= 11
Verbose	= 1
Se Z	=  34.0
NBlocks	= 1

NELE	= 11
NTRANS	= 1
ETYPE	= 1
NEGRID	= 2
	  6.80000000E+01
	  8.84000000E+02
UTYPE	= 1
NUSR	= 2
	  6.80000000E+01
	  8.84000000E+02
#bound 2J   free 2J  Delta E     nsub
     0  1      8  2  2.5178E+03  3
# -1/2 -> -1
 6.8000E+01  1.7855E-03  2.6242E-02
 8.8400E+02  1.8495E-02  2.0646E-01
--------------------------------------------
# -1/2 -> 0
 6.8000E+01  8.9710E-04  1.3185E-02
 8.8400E+02  9.3091E-03  1.0392E-01
--------------------------------------------
# -1/2 -> 1
 6.8000E+01  2.8326E-06  4.1632E-05
 8.8400E+02  3.1205E-05  3.4833E-04
 ......
\end{verbatim}

\subsection{\texttt{DB\_RO}}
\index{DB\_RO}
\begin{verbatim}
FAC 1.1.5
Endian	= 0
TSess	= 1542047284
Type	= 16
Verbose	= 1
C Z	=   6.0
NBlocks	= 14

NELE	= 3
NTRANS	= 3
#  bound  free Energy            index  norbs  vnl  weight       quantum defect
     0    224  6.42565608E+01      0     14    200  1.99718E+00  1.97709E-01
     0    224  6.42565608E+01      1     14    300  2.02816E-03  1.86719E-01
     0    224  6.42565608E+01      2     14    400  4.12667E-04  1.84284E-01
     0    224  6.42565608E+01      3     14    500  1.59235E-04  1.83286E-01
     0    224  6.42565608E+01      4     14    600  7.97438E-05  1.82775E-01
     0    224  6.42565608E+01      5     14    700  4.61200E-05  1.82477E-01
     0    224  6.42565608E+01      6     14    800  2.92428E-05  1.82288E-01
     0    224  6.42565608E+01      7     14    900  1.97715E-05  1.82161E-01
......
\end{verbatim}

\subsection{\texttt{DB\_CX}}
\index{DB\_CX}
\begin{verbatim}
FAC 1.1.5
Endian	= 0
TSess	= 1542048606
Type	= 17
Verbose	= 1
C Z	=   6.0
NBlocks	= 13

NELE	= 3
NTRANS	= 3
TGTS	= H
TGTZ	= 1
TGTM	= 1.008
TGTA	= 4.5
TGTB	= 1.8
TGTE	= 0.49973
TGTX	= -1
LDIST	= 5
TE0	= 0
NE0	= 36
	  1.00000000E-02
	  1.58489319E-02
	  2.51188643E-02
	  3.98107171E-02
	  6.30957344E-02
	  1.00000000E-01
	  1.58489319E-01
	  2.51188643E-01
	  3.98107171E-01
	  6.30957344E-01
	  1.00000000E+00
	  1.58489319E+00
	  2.51188643E+00
	  3.98107171E+00
	  6.30957344E+00
	  1.00000000E+01
	  1.58489319E+01
	  2.51188643E+01
	  3.98107171E+01
	  6.30957344E+01
	  1.00000000E+02
	  1.58489319E+02
	  2.51188643E+02
	  3.98107171E+02
	  6.30957344E+02
	  1.00000000E+03
	  1.58489319E+03
	  2.51188643E+03
	  3.98107171E+03
	  6.30957344E+03
	  1.00000000E+04
	  1.58489319E+04
	  2.51188643E+04
	  3.98107171E+04
	  6.30957344E+04
	  1.00000000E+05
# bound 2J  free  2J vnl  Energy
     0  1    224  0  200  6.42565608E+01
# Energy      cross section
 1.00000E-02  6.31443E-11
 1.58489E-02  1.00078E-10
 2.51189E-02  1.58615E-10
 3.98107E-02  2.51393E-10
 6.30957E-02  3.98440E-10
 1.00000E-01  6.31503E-10
 1.58489E-01  1.00090E-09
 2.51189E-01  1.58639E-09
 3.98107E-01  2.51441E-09
 6.30957E-01  3.98536E-09
 1.00000E+00  6.31696E-09
 1.58489E+00  1.00131E-08
 2.51189E+00  1.58811E-08
 3.98107E+00  2.54417E-08
 6.30957E+00  4.49154E-08
 1.00000E+01  1.20142E-07
 1.58489E+01  5.39695E-07
 2.51189E+01  2.65845E-06
 3.98107E+01  1.26490E-05
 6.30957E+01  7.57326E-05
 1.00000E+02  6.87748E-04
 1.58489E+02  5.49272E-03
 2.51189E+02  3.13705E-02
 3.98107E+02  1.29338E-01
 6.30957E+02  4.04120E-01
 1.00000E+03  1.00231E+00
 1.58489E+03  2.05237E+00
 2.51189E+03  3.58498E+00
 3.98107E+03  5.48974E+00
 6.30957E+03  7.54018E+00
 1.00000E+04  9.46891E+00
 1.58489E+04  1.10486E+01
 2.51189E+04  1.21412E+01
 3.98107E+04  1.27048E+01
 6.30957E+04  1.27725E+01
 1.00000E+05  1.24245E+01
\end{verbatim}

\chapter{FAC Function Reference}
\label{cha:function}
\index{Functions}
This chapter describes the functions available in the PFAC interface which are
organized in the package named \mod{pfac}. Each
python module is documented in a separate section, where the global variables,
functions, classes are listed in alphabetic order. All functions in the
extension modules \mod{fac}, \mod{crm}, and \mod{pol} are also available in
SFAC interface with identical calling syntax using the 3 executables
\mod{sfac}, \mod{scrm}, and \mod{spol}, unless otherwise indicated
explicitly. Their usage in SFAC
interface is therefore not documented separately. Functions and classes of
other python modules are generally not implemented in SFAC interface. In the
documentation of each function, the arguments in brackets are optional,
arguments separated by ``$\mid$'' are alternative forms of calling syntax,
``...'' in the argument list denotes variable number of arguments, and
keyword arguments are indicated by \var{key=arg} pair.

\section{\mod{fac}--Core FAC Module}
\label{sec:fac}
\index{fac}
\subsection{Variables}
There are several global variables one may make use of.
\begin{vardesc}{ATOMICMASS}
This is a list of atomic masses of all elements in the periodic
table. \key{ATMICMASS}[\var{i}] is the mass for atom with nuclear charge equal
to \var{i}.
\end{vardesc}
\begin{vardesc}{ATOMICSYMBOL}
This is a list of strings representing the atomic symbols of all elements in
the periodic table. The first element of this list is empty. Therefore
\key{ATOMICSYMBOL}[\var{i}] is for atom with nuclear charge equal to \var{i}.
\end{vardesc}
\begin{vardesc}{QKMODE}
A dictionary mapping the name of radial integral computational modes to its
integer values.
\begin{verbatim}
QKMODE = {
    'default':      -1,
    'exact':        0,
    'interpolate':  1,
    'fit':          2,
    'cb':           3,
    'dw':           4,
    'bed':          5}
\end{verbatim}
The corresponding upper case names can also be used in accessing these
numbers.

The modes \key{'exact'}, \key{'interpolate'}, and \key{'fit'} are
used for collisional excitation and radiative recombination. \key{'exact'}
requires the radial integrals are calculated on the specified collision energy
grid as is. \key{'interpolate'} requires the calculation on the \key{egrid}
energy grid, and interpolated to \key{usr\_egrid} user energy grid. \key{'fit'}
requires the radial integrals to be fitted by an analytic formula. The modes
\key{'cb'}, \key{'dw'}, and \key{'bed'} are used for collisional
ionization. \key{'cb'} inidcates the Coulomb-Born values to be used for
the radial integrals. \key{'dw'} requires the radial integrals to be
calculated using the distorted-wave approximation. \key{'bed'} requres the
binary-encounter-dipole theory to be used. The mode \key{'default'} is one of
the mode discussed above depending on the atomic process. For collisional
excitation, \key{'default'} is \key{'exact'}, for radiative recombination,
\key{'default'} is \key{'fit'}, for collisional ionization, \key{'default'} is
\key{'bed'}.
\end{vardesc}
\begin{vardesc}{VERSION}
This is a string representing the version of FAC. It is in the form of
\var{major.minor.release}, where \var{major}, \var{minor}, and \var{release}
are numbers.
\end{vardesc}

\subsection{Functions}
The module contains the following functions.
\begin{fundesc}{AIBranch}{fn, b, f}
Looking up the autoionization rate from \var{b} to \var{f} and the total
autoionization rate from \var{b} in the binary file \var{fn}. It returns a
Tuple consisting of the resonance energy, partial autoionization rate, and
total autoionization rate.
\end{fundesc}

\begin{fundesc}{AITable}{fn, b, f\opt{, c}}
Calculate the autoionization rates between the bound configuration group
\var{b} and the free configuration group \var{f}. The results are saved in
file \var{fn}. The optional channel number \var{c} can be supplied as an
identification of the transition array.
\end{fundesc}

\begin{fundesc}{AITableMSub}{fn, b, f\opt{, c}}
Calculate the magnetic sublevel autoionization rates and dielectronic capture
strength between the bound configuration group \var{b} and the free
configuration group \var{f}. The results are saved in file \var{fn}. The
optional channel number \var{c} can be supplied as an identification of the
transition array.
\end{fundesc}

\begin{fundesc}{AppendTable}{fn}
By default, when a new script is executed, existing binary files are
overwritten. If instead the new data should be appended to the file, use
this function to set the append flag.
\end{fundesc}

\begin{fundesc}{Asymmetry}{fn, s\opt{, m}}
Calculate the photoionization asymmetry parameters for given relativistic
subshells. \var{s} is a string which gives the subshells. It should be in
spectroscopic notation, e.g., '1s' for $1s_{1/2}$, '2p' for $2p_{1/2,3/2}$
shells, '3p-' for $3p_{1/2}$ and '3p+' for $3p_{3/2}$, etc. The optional
\var{m} specifies the maximum multipole expansion, counting in the order E1,
M1, E2, M2, ..., therefore, $m=1$ includes only E1, $m=2$ includes E1 and M1,
etc. Default is $m=1$. The results are stored in file \var{fn}. For each
subshell, the output starts with one line indicating which subshell it is,
its $nlj$ values, the ionization energy, the number of energy points, and the
value of \var{m} used. It is followed by a block, which contains one line for
each energy point and tabulates the electron energy, photon energy, total
photoionization cross section ($10^{-20}$ cm$^2$), cross section for electron
direction perpendicular to that of photon, the total radiative recombination
cross section, the cross section at 90$^\circ$, and the ratio of
$\sigma_\perp$/$\sigma_\parallel$, where $\sigma_\perp$ is the 90$^\circ$
radiative cross section for photons polarized in the direction perpendicular
to the electron direction, and $\sigma_\parallel$ is the 90$^\circ$ cross
section for photons polarized in the direction parallel to the electron
direction. The last piece of data for each subshell are the $B_\lambda$ and
$B_\lambda^\phi$ parameters defined as
\begin{equation}
\frac{d\sigma}{d\Omega}=\frac{\sigma}{4\pi}\left[\sum_{\lambda\ge 0} B_\lambda
  P_{\lambda}(\cos\theta) - \sum_{\lambda\ge 2} \lambda^{-1}(\lambda-1)^{-1}
  B_\lambda^\phi P_{\lambda}^2(\cos\theta)\cos(2\phi)\right],
\end{equation}
where $\sigma$ is the total cross section, $\theta$ is the angle between the
electron and photon directions, $\phi$ is the azimuth angle of polarization
direction of the photon relative to the electron direction.
\end{fundesc}

\begin{fundesc}{AverageAtom}{pref, m, d, t}
Calculate the average atom model. \var{pref} is the file name prefix for the
output. \var{m} is a mode parameter with values of 3 or 4. The difference
between the two is the treatment of quasi-bound orbitals. With mode 4, some
positive energy orbitals are treated quantum mechanically, while with 3, all
positive energy electrons are treated in Thomas-Fermi model. \var{d} is the
density of the atom in g/cc. \var{t} is the temperature in eV. Two output
files are created. \verb|pref.den| and \verb|pref.pot|. The first tabulates
varius electron density profiles, the 2nd is the potentials obtained with
\key{GetPotential} function.
\end{fundesc}

\begin{fundesc}{AvgConfig}{c}
Setup a mean configuration for the optimization of central potential as
specified by a string \var{c}. The format of the string is the same as in the
function \key{Config}, except that the occupation can be a non-integer number
in this routine. The mean configuration setup by this function is effective
only if the function \key{OptimizeRadial} is called with no
arguments. Otherwise, the configurations given in that function are used to
generate the mean configuration automatically. It is important that this
function be called before \key{OptimizeRadial} and after
\key{ConfigEnergy(0)}, if the later is used.
\end{fundesc}

\begin{fundesc}{BasisTable}{fn\opt{,m}}
Print out a table of basis wavefunctions and mixing coefficients in the file
\var{fn}. If \var{m} is 0, then the basis table for the ordinary atom is
given. If \var{m} is 1, the basis table for atom in magnetic and electric fields are
given.

If \var{m} is 10, this function generates multiple files (\var{fn}\_***.c and
\var{fn}\_***.cm), which are GRASP2K-compatible-basis list file and mixing
coefficients files.
It makes it easy to use GRASP2K.JJ2LS module to get \var{LS}-coupling notation.
However, some bugs should be corrected before using GRASP2K.JJ2LS.
See https://github.com/flexible-atomic-code/fac/tree/master/doc/misc/jj2lsj.md for the details.
\end{fundesc}

\begin{fundesc}{CECross}{ifn, ofn, low, up, e\opt{, m}}
Calculate the collision strength and cross sections at energies given in list
\var{e} between levels \var{low} and \var{up} by interpolating the data from
CE binary data file \var{ifn}. The results are written in \var{ofn} in simple
ASCII format. Before this routine is called, \key{MemENTable} must be called
to establish the energy table that is consistent with \var{ifn}. The energy
given in \var{e} is that of the scattering energy if $m=1$, which is the
default, or that of the incident energy if $m = 0$.
\end{fundesc}

\begin{fundesc}{CERate}{ifn, ofn, low, up, t}
Calculate the effective collision strength and rate coefficients at
temperatures given in list \var{t} between levels \var{low} and \var{up} by
interpolating the data from CE binary data file \var{ifn}. If either \var{low}
or \var{up} is negative, then the level index is looped over to give rate
coefficients for all transitions. The results are
written in \var{ofn} in simple ASCII format. Before this routine is called,
\key{MemENTable} must be called to establish the energy table that is
consistent with \var{ifn}
\end{fundesc}

\begin{fundesc}{CETable}{fn, low, up}
Calculate the collision strength for the excitation of states in the
configuration group list \var{low} to those in the group list\var{up}. The
results are saved in the file \var{fn}.
\end{fundesc}

\begin{fundesc}{CETableEB}{fn, low, up\opt{, m}}
Calculate the collision strength for the excitation of states in the
configuration group list \var{low} to those in the group list\var{up}, but for
atoms in magnetic and electric fields. The
results are saved in the file \var{fn}. If \var{m} is 0, then incident
electron is assumed to be isotropic, otherwise, cross sections at different
incident directions are calculated.
\end{fundesc}

\begin{fundesc}{CETableMSub}{fn, low, up}
Calculate the magnetic sublevel collision strength for the excitation of
states in the configuration groups \var{low} to those in the groups
\var{up}. The results are saved in the file \var{fn}.
\end{fundesc}

\begin{fundesc}{CITable}{fn, b, f}
Calculate the collision strength for the ionization of states in the bound
configuration groups \var{b} to those in the free configuration groups
\var{f}. The results are saved in the file \var{fn}.
\end{fundesc}

\begin{fundesc}{CheckEndian}{\opt{fn}}
Check the byte order of database file \var{fn}. It returns 0 for little endian
and 1 for big endian. If the optional file name \var{fn} is omitted, the
endian for the current platform is returned.
\end{fundesc}

\begin{fundesc}{CITableMSub}{fn, b, f}
Calculate the magnetic sublevel collision strength for the ionization of
states in the bound configuration groups \var{b} to those in the free
configuration groups \var{f}. The results are saved in the file \var{fn}.
\end{fundesc}

\begin{fundesc}{ClearLevelTable}{}
Clear the energy level table in the memory.
\end{fundesc}

\begin{fundesc}{ClearOrbitalTable}{\opt{m}}
Clear the radial orbital table in the memory. If the optional argument \var{m}
= 0, the entire table is cleared. If \var{m}$>$0, only the continuum orbitals
are cleared.
\end{fundesc}

\begin{fundesc}{CloseSFAC}{}
Close the file containing the SFAC input file converted from the current
Python script. This function must be called after \key{ConvertToSFAC}. Only the
statements between the call to \key{ConvertToSFAC} and \key{CloseSFAC} are
converted to SFAC input file. This routine is only available in PFAC interface.
\end{fundesc}

\begin{fundesc}{Closed}{s, ...}
Specify the closed shells in the electronic configurations. It takes variable
number of arguments, each of them is a non-relativistic or relativistic shell
in the spectroscopic notation. e.g., \key{2s} for $2s$ shell, \key{2p-} for
$2p_{1/2}$ shell, and \key{2p+} for $2p_{3/2}$ shell.
\end{fundesc}

\begin{fundesc}{Config}{c, ..., group=g $\mid$ g, c, ...}
Add one or more configurations to the configuration group \var{g}. In the
first form, the group name \var{g} is given as a keyword, while in the second
form, the first argument must be a group name instead of a configuration. It
takes one or more strings for the configuration specification. A configuration
\var{c} is a string comprised of one or more non-relativistic or relativistic
shells in spectroscopic notation separated by white spaces. e.g.,
\key{2[p+]3} is a $2p_{3/2}$ shell with 3 electrons. If a \key{*} is given
instead of the orbital angular momentum symbol, configurations with all
legitimate values are generated. It is also possible to use \key{[s,p,d]} to
indicate that the orbital angular momentum may take $s$, $p$, or $d$
values. For $l > 20$, no spectroscopic symbol is available, it is specified as
[$l$] such as [21,22] for $l$ = 21 and 22 shells. This numerical notation also
works for $l \le 20$ shells, for which spectroscopic symbols \key{s, p, d, f,
g, h, i, k, l, m, n, o, q, r, t, u, v, w, x, y, z} are available. The bracket
for the orbital angular momentum can be omitted if it comprises of a single
character. Otherwise, the bracket must be present. For example, $2p-2$ is
illegal, but $2[p-]2$ is. Each shell may be followed by multiple conditions on
the occupation number, separated by ``;'', e.g., $3*10;3s>0;3p>5$ generate
configuarations that have at least 1 electron in the $3s$ and 6 electrons in
the $3p$ shells. The logical relations allowed in conditions include $=$, $>$,
and $<$. If the configuration specification starts with the character ``@'',
then it is assumed what follows is a file name where configurations are given
one per line.
\end{fundesc}

\begin{fundesc}{ConfigEnergy}{m\opt{, n\opt{, g, ...}}}
This function should be called twice just before (with \var{m} = 0) and after
(with \var{m} = 1) \key{OptimizeRadial} if used. If \key{AvgConfig} is
called, then \key{ConfigEnergy} with \var{m}=0 must be called before that.
The call with \var{m} = 0
performs a radial optimization for the configuration groups given by the list
\var{g}, and calculate the average energy of each configuration under
such potentials. Multiple optimizations are performed if more than one list
are given. If none is given, the optimization are carried out for each
configuration group. If \var{m} = 0, one may also
specify an integer \var{n} to indicate that \key{RefineRadial(n)} should be
called after the optimization. The call with \var{m} = 1 does not accept
additional arguments. It recalculates the average energy of each configuration
under the potential obtained by \key{OptimizeRadial} issured by the user, The
diagonal elements of the Hamiltonian calculated in the \key{Structure} call is
then adjusted by the difference of the two average energies for each
configuration. The purpose of this routine is to remove some of the errors in
the level energies introduced by using a single central potential for all
configurations.
\end{fundesc}

\begin{fundesc}{ConvertToSFAC}{fn}
Converts the statements between this call and the \key{CloseSFAC} call to SFAC
input file \var{fn}. The resulting file can then be run using the \key{sfac}
executable. This routine is only available in PFAC interface.
\end{fundesc}

\begin{fundesc}{CorrectEnergy}{fn, nmin $\mid$ ilev, e, nmin}
Correct the energies of certain levels by given amount. This is used if the
exact energy of some levels is critical. In the first form, the indexes and the
energy corrections are listed as two columns in the file \var{fn}. In the
second form, the indexes are given in the Python list \var{ilev}, and the
energy corrections are given in the list \var{e}. Only levels with valence
electron in $n\ge nmin$ are corrected, if these levels are not constructed
with \key{RecStates}. The corrections are given in
units of eV. For the first level in the correction list, the amount is added
to the energies calculated by FAC, and that level is taken as the reference
level for all other entries in the list. The correction energies in the list
\var{e} for these other entries with \var{ilev}$\ge 0$ are the desired energies
relative to the reference level, i.e., they replace the energies calculated by
FAC. However, if \var{ilev}$<0$, then these corrections are also interpreted
as shifts to be added to the energyies calculated by FAC.
\end{fundesc}

\begin{fundesc}{CutMixing}{g0, g1\opt{, c}}
For each levels in the group list \var{g0}, eliminate all mixing components
that are not in the group list \var{g1} or the mixing coefficients less than
\var{c}.
\end{fundesc}

\begin{fundesc}{CXTable}{fn, b, f}
Calculate the charge exchange cross sections between bound $b$ and free $f$
configuration groups. The results are saved in the file \var{fn}. The
multi-channel Landau-Zener approximation is used.
\end{fundesc}

\begin{fundesc}{DROpen}{g}
Calculate the principle quantum numbers at which the dielectronic recombnation
starts to open for the core excitations from the ground state to the states in
the configuration group list \var{g}. It returns a list of integers.
\end{fundesc}

\begin{fundesc}{FinalizeMPI}{}
If FAC is compiled with MPI support, this invokes MPI finalize method before
exiting. This function should be called as the last line of the script.
\end{fundesc}

\begin{fundesc}{GetCFPOld}{j, q, dj, dw, pj, pw}
Return the coefficient of fractional parentage
$(j^{q-1},pj,pw|\}j^{q},dj,dw)$, where $j$ is the angular momentum of the
shell. $q$ is the occupation of the daughter state. $pj$ and $dj$ are the
total angular momenta of the parent and daughter states. $pw$ and $dw$ are
the seneority of parent and daughter states. All angular momenta appearing
here are twice of their actual values.
\end{fundesc}

\begin{fundesc}{GetCG}{j1, j2, j3, m1, m2, m3}
Return the Clebsch-Gordan coefficient $<$$j1m1, j2m2|j3m3$$>$. All angular
momenta appearing here are twice of their actual values.
\end{fundesc}

\begin{fundesc}{GetConfigNR}{c, ...}
This functions returns a list of non-relativisitc configurations corresponding
to the supplied configuration strings, which may contain wild casts as in the
function \key{Config}. It is useful, e.g., when one wants to know the
non-relativistic configurations of \key{1*2 2*2 3*1}.
\end{fundesc}

\begin{fundesc}{GetPotential}{fn}
Print the radial potential obtained by \key{OptimizeRadial} to the file
\var{fn}. The file starts with the parameter for the analytic fit the to
potential, $\lambda$, in the formula
\begin{equation}
V_0(r) = -\frac{Z}{r} + \frac{N-1}{r}\left(1-\exp(-\lambda r)\right).
\end{equation}
After that, the mean configuration used to generate the potential is printed
in 3 columns representing the principle quantum number, the relativistic
angular quantum number $\kappa$, and the fractional occupation number
respectively. Finally, the file gives 8 columns which are $i$, $r$, $Z(r)$,
$V(r)$, $V_d(r)$, $V_e(r)$, $V_e^\prime(r)$, and $U(r)$, where $i$ is the index
for the radial grid, $r$ is the radial grid, $Z(r)$ is the nuclear charge at
radius $r$ taking into account the nuclear charge distribution, $V(r)$ is the
optimal potential, $V_d(r)$ is the direct interaction part of the potential,
$V_e(r)$ is the exchange interaction part of the potential, $V_e^\prime(r)$ is
the Slater approximation of the exchange interaction, and $U(r)$ is the
Uehling potential which approximates the vacuum polarization effects.
\end{fundesc}

\begin{fundesc}{GetW3j}{j1, j2, j3, m1, m2, m3}
Return the Wigner $3j$ symbol \threej{j1}{j2}{j3}{m1}{m2}{m3}. All angular
momenta appearing here are twice of their actual values.
\end{fundesc}

\begin{fundesc}{GetW6j}{j1, j2, j3, i1, i2, i3}
Return the Wigner $6j$ symbol \sixj{j1}{j2}{j3}{i1}{i2}{i3}. All angular
momenta appearing here are twice of their actual values.
\end{fundesc}

\begin{fundesc}{GetW9j}{j1, j2, j3, i1, i2, i3, k1, k2, k3}
Return the Wigner $9j$ symbol. All angular momenta appearing
here are twice of their actual values.
\end{fundesc}

\begin{fundesc}{Info}{}
Print out the version information of FAC and contact information of the
author.
\end{fundesc}

\begin{fundesc}{InitializeMPI}{\opt{opt}}
If FAC is compiled with MPI support, this function initializes the MPI
system. \var{opt} specifies the command line options processed by the MPI init
function. This function should be invoked as the first line.
\end{fundesc}

\begin{fundesc}{InterpCross}{ifn, ofn, low, up, e, m}
Interpolate cross sections in the binary file \var{ifn}, and save outputs in
\var{ofn}.
\var{low} and \var{up} are lower and upper indices of the transitions. If
either one is $-1$, then loop over all indices for that level. \var{e} is a
list of energies where the cross sections are needed. If \var{m} is 0, then
\var{e} is interpreted as the incident energy, if \var{m} is 1, then \var{e}
is interpreted as the incident energy minus transition energy.
\end{fundesc}

\begin{fundesc}{JoinTable}{fn1, fn2, fn}
Join two binary files \var{fn1} and \var{fn2} to produce a single file
\var{fn}. \var{fn1} and \var{fn2} must have been produced on the same
platform, have the same type and for the same element.
\end{fundesc}

\begin{fundesc}{LandauZenerCX}{fn, z, e\opt{, m}}
Calculate the multi-channel Landau-Zener charge exchange cross sections for
bare ions with atomic number $z$, on collision energies $e$ in eV/u, and save
the results in file \var{fn} in the Kronos db format. if $m<0$ 
or not present, $n$-resolved cross sections are output, otherwise, the
$nl$-resolved cross sections are output with an $l$-distribution applied.
\end{fundesc}

\begin{fundesc}{LandauZenerLD}{z, n\opt{, m, j, mj}}
Return an array contains the $l$-distribution of the charge exchange capture
for nuclear charge $z$, principal quantum number $n$, distribution mode $m$,
average angular momentum $j$, and the mixing parameter for the combination of
low energy and statistical distribution.
\end{fundesc}

\begin{fundesc}{LevelInfor}{fn, ilev}
If \var{ilev}$\ge 0$, look up in the energy database file \var{fn} for the
level indexed
\var{ilev}. It returns a tuple of length 6. The first element is the energy in
atomic units, the second is the parity and the valence $nl$ values, the
third is the 2J value, the fourth is the complex name, the fifth is the
non-relativsitic configuration, and the sixth is the relativistic
configuration and the coupling. The parity and the $nl$ values are encoded
as $\pm n*100+l$, where $+$ sign is for even parity and $-$ sign is for odd
parity. If \var{ilev}$<0$, and its absolute value is less than 1000, it returns
the index of the first level with number of electrons equal to $-ilev$. If
\var{ilev}$=-1000$, it returns the index of the first level of the bare
ion. If \var{ilev}$=-1001$, it returns the total number of levels in the
file. If \var{ilev}$=-1002$, it returns the total number of level blocks in
the file. If \var{ilev}$\le-2000$, it returns the index of the first level of
the block $-ilev-2000$.
\end{fundesc}

\begin{fundesc}{ListConfig}{\opt{fn, g}}
Print the configurations in the list \var{g} to file \var{fn}. If \var{fn} is
not given or if it is ``-'', the results are written to the stdout. If \var{g}
is not given, then all configurations currently defined are printed.
\end{fundesc}

\begin{fundesc}{MaxwellRate}{ifn, ofn, low, up, t}
Calculate the Maxwellian rate coefficients for collision processes with cross
section data given by the binary file \var{ifn}, the results are saved in
\var{ofn}. \var{low} and \var{up} are lower and upper indices of the
transitions. If
either one is $-1$, then loop over all indices for that level. \var{t} is a
list of temperatures where the rate coefficients are needed.
\end{fundesc}

\begin{fundesc}{MemENTable}{fn}
Build an energy level table in the memory for the \texttt{DB\_EN} type file
\var{fn}. This function must be called before any calls to \key{PrintTable}
in verbose mode.
\end{fundesc}

\begin{fundesc}{ModifyPotential}{fn}
  Modify the model central potential such that it matches the potential given
  in the file. The file must specify the potential in two columns
  corresponding to the radii and potential value in atomic units. This can be
  useful to import a central potential produced by a different atomic
  structure code.
\end{fundesc}

\begin{fundesc}{OptimizeRadial}{\opt{g\opt{, w}}}
Obtain the optimal radial potential based on the mean configuration generated
by the configuration group list \var{g} and the weight \var{w}, or if they are
absent in the call, by the mean configuration specified by \key{AvgConfig}.
\var{g} and \var{w} must be equal length Python lists if both present. \var{g}
is a list of configuration groups, and \var{w} is a list of weights for each
group when generating the mean configuration. If only \var{g} is present, each
configuration group is given an equal weight.
\end{fundesc}

\begin{fundesc}{PICrossH}{Z, E, n, l\opt{, m}}
Calculate the photionization cross section of H-like ion with nuclear charge
\var{Z} at photon energy \var{E}, for the non-relativisitc subshell
\var{nl}. The result is in unit of $10^{-20}$ cm$^2$. If the optional
parameter \var{m} is not 0, then the differential oscillator strength is
returned instead of the cross sections.
\end{fundesc}

\begin{fundesc}{PlasmaScreen}{z,n\opt{,t,m,u}}
Specify a plasma screening model. \var{z} is the number of free electrons per
ion in the plasma, \var{n} is the electron density in unit of
$10^{24}$~cm$^{-3}$. \var{t} is the electron temperature in eV. \var{m} is the
model choices. $m=0$ uses a uniform ioni-sphere model, in which case only
\var{z} and \var{n} are required input. $m=1$ uses a non-uniform ion-sphere
model where free electrons follow Fermi-Dirac distribution, in which case,
\var{t} is a required parameter. $m=2$ uses the Stewart-Pyatt screening model,
in which case, \var{u} is a required parameter, corresponding to the $z^{*}$
in this model.
\end{fundesc}

\begin{fundesc}{PrepAngular}{p\opt{, q}}
Precalculate the angular coefficients between states in \var{p} and
\var{q}. \var{p} and \var{q} are lists of configuration groups. If \var{q} is
not present, the angular coefficients are calculated between states in the
\var{p} list. Only $Z^L(\alpha,\beta)$ and $\tilde{a_\alpha}$ coefficients are
calculated. If the Bra and Ket states have the same number of electrons, $Z^L$
is calculated, otherwise $\tilde{a_\alpha}$ is calculated. This routine should
primarily be used when atomic states are construced with \key{RecStates},
where the angular coefficients between the base states are used many times. It
is therefore more efficient to precalculate these coefficients.
\end{fundesc}

\begin{fundesc}{Print}{args}
Print out the string representation of \var{args}. This function exists to
asist the conversion to SFAC interface, since Python's \key{print} statement
is not converted.
\end{fundesc}

\begin{fundesc}{PrintTable}{fnb, fna\opt{, v}}
Convert the binary database file \var{fnb} to the ASCII file \var{fna}. The
optional argument \var{v} = 1 requires the conversion be done in verbose
mode, otherwise it is done in simple mode. Note that before conversion in
verbose mode is carried out, one must call \key{MemENTable} first.
\end{fundesc}

\begin{fundesc}{PrintNucleus}{\opt{m,fn}}
Print the nucleus model in file \var{fn} if given, or to stdout. if $m=1$ use
the GRASP isodata format.
\end{fundesc}

\begin{fundesc}{PrintQED}{}
Print the QED correction options currently being used.
\end{fundesc}

\begin{fundesc}{PropogateDirection}{m}
\var{m} specifies the direction in which the R-matrix solution in the outer
region is propogated. $m\ge 0$ indicates that the R-matrix in the first zone
is propogated outward to the final matching radius. $m<0$ indicates that the
wavefunction at the final matching radius is propogated inward to the first
zone and matched to the R-matrix there. Default is $m=0$.
\end{fundesc}

\begin{fundesc}{RecStates}{fn, b, n}
Construct recombined states by adding a spectator electron with the principle
quantum number \var{n} onto the basis states in the configuration groups
\var{b}. The orbital angular momentum of the spectator electron is set by two
functions \key{SetRecPWLimits} and \key{SetRecSpectator}. The resulting energy
levels are saved in file \var{fn}.
\end{fundesc}

\begin{fundesc}{RefineRadial}{\opt{n\opt{, m}}}
This function may be called after \key{OptimizeRadial}, which performs a
minimization of the total energy of the mean configuration by adjusting the
parameters in the analytic central potential. \var{n} is the number of energy
evaluations allowed in the minimization, and \var{m} controls the print out
during the calculation. Default: \var{n} = 100, \var{m} = 0 (no print out).
\end{fundesc}

\begin{fundesc}{Reinit}{\textnormal{**}mkey $\mid$ m}
Reinitialize some or all subsystems of FAC package. In the first form, it
accepts variable number of keyword arguments, each represents one
subsystem. The following keyword identifiers may be used:
\begin{ttscript}{recombination}
\item[\key{config}] Electronic configuration and coupling informations. If
\var{config$\ge$0}, exsisting configurations and coupling are cleared.
\item[\key{recouple}] Angular recoupling package. If \var{recouple$\ge$0}, all
recoupling interaction array is cleared.
\item[\key{structure}] Atomic structure calculatoin. If \var{structure$\ge$0},
the energy level table and all angular coefficients are cleared.
\item[\key{radial}] Radial wavefunction and integrals. If \var{radial$=$0}, the
radial orbital table, the potential table, all radial integral tables are
cleared. If \var{radial$=$1}, only continuum orbitals are cleared.
\item[\key{excitation}] Collisional excitation. If \var{excitation$\ge$0}, the
collision energy grid and radial integrals are cleared.
\item[\key{recombination}] Photoionization, recombination and
autoionization. If \var{recombination$=$0}, the energy grid, radial integrals,
and recombined complex table are cleared. If \var{recombination$=$1}, only the
energy grid and radial integrals are cleared.
\item[\key{ionization}] Collisional ionization. If \var{ionization$\ge$0}, the
energy grid and radial integrals are cleared.
\item[\key{dbase}] Database handling. If \var{dbase$=$0}, the energy table in
memory is cleared and all database headers are reinitialized. If
\var{dbase$=$t} with \var{t$>$0}, only the header for the database type
\var{t} is reinitialized.
\end{ttscript}

In the second form, the integer \var{m} requires a certain combination of
subsystems to be reinitialized. It may be $-1$, 0, or 1. If \var{m=0}, then all
keywords are set to 0, i.e., a full reinitialization. If \var{m=1}, keywords
\key{radial}, \key{excitation}, \key{recombination}, \key{ionization},
\key{structure}, and \key{dbase} are set to 0, \key{config} and
\key{recouple} are not reinitialized. This is useful to clear all information
related to radial wavefunctions, but keep the configuration data intact. If
\var{m=$-1$}, all keywords are set to 1.
\end{fundesc}

\begin{fundesc}{ReinitConfig}{m}
Reinitialize electronic configuration module. \var{m} the same as correponding
keywords in \key{Reinit}.
\end{fundesc}

\begin{fundesc}{ReinitDBase}{m}
Reinitialize database module. \var{m} the same as correponding
keywords in
\key{Reinit}.
\end{fundesc}

\begin{fundesc}{ReinitExcitation}{m}
Reinitialize collisional excitation module. \var{m} the same as correponding
keywords in
\key{Reinit}.
\end{fundesc}

\begin{fundesc}{ReinitIonization}{m}
Reinitialize collisional ionization module. \var{m} the same as correponding
keywords in
\key{Reinit}.
\end{fundesc}

\begin{fundesc}{ReinitRadial}{m}
Reinitialize radial module. \var{m} the same as correponding
keywords in
\key{Reinit}.
\end{fundesc}

\begin{fundesc}{ReinitRecombination}{m}
Reinitialize recombination module. \var{m} the same as correponding
keywords in
\key{Reinit}.
\end{fundesc}

\begin{fundesc}{ReinitRecouple}{m}
Reinitialize recouple module. \var{m} the same as correponding
keywords in
\key{Reinit}.
\end{fundesc}

\begin{fundesc}{ReinitStructure}{m}
Reinitialize atomic structure module. \var{m} the same as correponding
keywords in
\key{Reinit}.
\end{fundesc}

\begin{fundesc}{RestorePotential}{fn}
  Restore the model central potential from a file generated by the
  \key{SavePotential} function. This can be useful when one job needs to use
  the exact same potential as a previous one.
\end{fundesc}

\begin{fundesc}{RMatrixBasis}{fn, k, nb}
Calculate the R-matrix basis functions, and buttle corrections, and save in
file \var{fn}. \var{k} is the maximum orbital angular momentum of the
continuum electron. \var{nb} is the number of basis functions per-$\kappa$
orbital.
\end{fundesc}

\begin{fundesc}{RMatrixBoundary}{r0, r1, b}
Set the R-matrix boundary conditions. When $r0=0$, it sets the normal boundary
condition via a call to \key{SetBoundary}(\var{n}, \var{r1}, \var{b}), where
$n$ is the maximum principle quantum numbers of the existing states, i.e., the
R-matrix target states. When $r0\ne 0$, it sets the boundary condition in the
propagation zone via a call to \key{SetBoundary}(-100, \var{r1}, \var{r0}). In
addition, the boundary condition is reset to \var{b}.
\end{fundesc}

\begin{fundesc}{RMatrixCE}{fn, bfn, rfn, emin, emax, de\opt{, m}}
Solve the outer region wavefunction, calculate the collision
strength. \var{fn} is the file for saving results. \var{bfn} is a list a
R-matrix basis files for each R-matrix zone. \var{rfn} is a list of R-matrix
surface files for each R-matrix zone. By default, the R-matrix in the first zone is
propogated outward to the last zone, and matched with the outer region
solution to obtain S-matrix. If the propogation direction has been set inward
in \key{PropogateDirection}, then the outer region solution is propagated
inward to the first zone, and matched with the R-matrix there. \var{emin},
\var{emax}, and \var{de} specifies
the energy grid. The entire energy grid is divided into batches to be
processed in once pass of all symmetris. The number of energy points in one
batch does not exceed 100 by default, or the value set by
\key{RMatrixNBatch}. \var{m} specfies the output content. If $m=0$, only the
total collision strengths are output. If $m=1$, the partial collision
strengths are output in addition. If $m=2$, the R-matrix and T-matrix elements
are output. If $m=3$, both partial collision strengths and R-matrix, T-matrix
elements are output.

R-matrix, T-matrix output format: each line tabulate, isym, its0, ka0, it1,
ka1, et, Real(T), Imag(T), Omega, R, where isym is the symmetry ID. its0 is
the target state index for the initial channel, ka0 is the $\kappa$ value of
the continuum electron in the initial channel. its1 and ka1 are corresponding
values for the final channel. et is the energy if the system relative to the
initial target energy. Real(T) and Imag(T) are the real and imaginary part of
the T-matrix, Omega is the partial collision strength of the symmetry. R is
the R-matrix. These data are repeated for every energy and every symmetry.

Total and partial collision strength output format: for each transition, first
line lists, initial target state index, 2J of initial state, final state
index, 2J of final state, transition energy, number of energy points in this
block, and the number of partial-waves. It is followed by the block for total
collision strengths in two columns: energy, collision strengths. If partial
collision strength output is requested, it is followed by three columns:
energy, orbital angular momentum of the partial wave, and partial collision
strength. The above data are repeated for each block of the energy points
processed in the individual batches.

For details of using the R-matrix functions, see the example in
\key{demo/rmatrix/} directory.
\end{fundesc}

\begin{fundesc}{RMatrixConvert}{ifn, ofn, m}
Convert R-matrix basis and surface files between binary and ascii format. If
$m = 0$, convert binary basis file \var{ifn} to ascii file \var{ofn}, if $m =
1$, convert ascii basis file \var{ifn} to binary file \var{ofn}, if $m = 2$,
convert binary surface file \var{ifn} to ascii file \var{ofn}, if $m = 3$,
convert ascii surface file \var{ifn} to binary file \var{ofn}.
\end{fundesc}

\begin{fundesc}{RMatrixExpansion}{m\opt{, r}}
Set how the channel wavefunctions at the outmost radius should be
calculated. $m$ is the number of terms in the Gailitis asymptotic
expansion. $m = 0$ indicates use the Dirac-Coulomb functions, which is the
default. $r$ is the outmost radius where the wavefunctions should be
calculated. If outer boundary of the last R-matrix zone, $a$ is smaller than
$r$, the the wavefunctions at $r$ is integrated inward until $a$, and matched
with the R-matrix there. If $r \le a$, then $a$ is used as the outmost radius.
\end{fundesc}

\begin{fundesc}{RMatrixFMode}{m}
If $m=0$, use binary format for the R-matrix output files, if $m = 1$, use
ascii format for the R-matrix output files.
\end{fundesc}

\begin{fundesc}{RMatrixNBatch}{n}
$n$ is the number of energy points to process at one pass of all symmetries in
  \key{RMatrixCE}. Default is $n=100$.
\end{fundesc}

\begin{fundesc}{RMatrixNMultipoles}{m}
The maximum multipoles included in the outer region potential. If $m$ is
larger than the number of multipoles saved in the R-matrix surface file, that
smaller value is used. The default is $m=2$.
\end{fundesc}

\begin{fundesc}{RMatrixSurface}{fn}
Calculate the R-matrix surface amplititudes and save in file \var{fn}.
\end{fundesc}

\begin{fundesc}{RMatrixTargets}{t\opt{, c}}
Set the target and correlation states. $t$ and $c$ are lists of configuration
groups, for target and correlation configurations respectively. The atomic
structure for both $t$ and $c$ must have been solved with \key{Structure}.
\end{fundesc}

\begin{fundesc}{ROTable}{fn, b, f}
Calcualate the recombination orbital occupancy data between bound $b$ and free
$f$ configuration groups. The results are saved in file \var{fn}.
\end{fundesc}

\begin{fundesc}{RRCrossH}{Z, E, n, l}
Calculate the radiative recombination cross section of bare ion with nuclear
charge \var{Z} at electron energy \var{E}, onto the non-relativisitc subshell
\var{nl}. The result is in unit of $10^{-20}$ cm$^2$.
\end{fundesc}

\begin{fundesc}{RRMultipole}{fn, b, f\opt{, m}}
Calculate the bound-free multipole matrix elements. Because \key{RRTable}
does not calculate the matrix elements directly, this function exisits to
provide such data. The output are written to the file \var{fn} in ASCII
format. The format is described in the header of the file itself. The
remaining arguments are identical to those in \key{RRTable}
\end{fundesc}

\begin{fundesc}{RRTable}{fn, b, f\opt{, m}}
Calculate the bound-free differetial oscillator strengths between the bound
configuration groups \var{b} and free groups \var{f}, which are related to
radiative recombination and photoionization cross sections. The optional
multipole type \var{m} is set to -1 (E1) by default. In almost all cases, no
other multipole types should be important. The results are saved in file
\var{fn}.
\end{fundesc}

\begin{fundesc}{SavePotential}{fn}
  Save the model central potential in a binary file, which can then be
  restored using the \key{RestorePotential} function in a later job
\end{fundesc}

\begin{fundesc}{SetAICut}{c}
Set the autoionization rate cutoff threshold in the output. Only
autoionization rates greater than \var{c} a.u. are output. The default is
$10^{-16}$ a.u. or $\sim 4.13$ s$^{-1}$, if this routine is not called.
\end{fundesc}

\begin{fundesc}{SetAngZCut}{c}
Set the cutoff threshold for the mixing basis in the calculation of recoupling
coefficients. Only the basis functions with mixing coefficients $>$\var{c} are
included. The default is $10^{-6}$ if this routine is not called.
\end{fundesc}

\begin{fundesc}{SetAtom}{asym\opt{, z\opt{, m\opt{,r}}}}
This function set the atomic element to \var{asym}, where \var{asym} is the
standard elemental symbol. The nuclear charge \var{z}, atomic mass \var{m},
and the nucleus radius of the element can be set optionally. The radius should
be given in fm. If they are not
set, the standard values are used.
\end{fundesc}

\begin{fundesc}{SetAtom}{z\opt{, m\opt{,r}}}
This function set the nuclear charge \var{z}. The atomic symbol is looked up
in the internal table. if \var{z} falls outside the valid range of 1 to 120,
the atomic symbol is set to Xx. The tomic mass \var{m},
and the nucleus radius of the element can be set optionally. The radius should
be given in fm. If they are not set, the standard values are used.
\end{fundesc}

\begin{fundesc}{SetBoundary}{n\opt{, p, b}}
Set the boundary condition for high-n orbitals, so that they can represent the
continuum. In the normal mode when $n>0$, \var{n} is the maximum principle
quantum number represents the
bound states. Above n, the orbitals have the boundary conditions imposed,
which is in the form $Q/P = (b+\kappa)/2ac$, where $Q$ and $P$ are the small
and large components of the wave function, $\kappa$ is the angular quantum
number of the orbital, $b$ is given by the input, $c$ is the speed of light,
and $a$ is the boundary radius, which is determined by the criterion that the
bound states have densities $<p$ beyond the boundary. The default value of $p$
is 0.001. When $n<0$ and $n \ne -100$, $p$ is interpreted as the actual radius
of the boundary, and $-n$ is interpreted as the maximum principle quantum
number of the bound states. When $n==-100$, it sets two boundaries. The radius
of the outer boundary is given by $p$, and that of the inner boundary is given
by $b$, if $b > 0$. Otherwise, the inner boundary takes the value of a
previous call to \key{SetBoundary}. This mode is used to set the R-matrix
boundary conditions in the propagation zones.
\end{fundesc}

\begin{fundesc}{SetBreit}{n\opt{,m}}
Set the calculaiton mode and maximum principle quantum number of the orbitals
for which the Breit interaction should be included in the Hamiltonian. If
\var{n} $<$ 0, then the Breit interaction involving all bound and continuum
states is included. Default is 5. \var{m} is the calculation mode. \var{m=0}
uses a simplified zero-energy approximaiton for the Breit term. \var{m=1} uses
the full frequency dependent Breit interaction. \var{m=2} uses the same
\var{m=1} logic, but set the energy of the exchanged photon to 0.
\end{fundesc}

\begin{fundesc}{SetCEBorn}{e\opt{,x\opt{, x1}}}
\var{e} specifies the asysmtotic energy where the plane-wave Born approximation is
expected to be valid, and is used to deduce the constant component of the
collision strength at high energies. If $e > 0$, it is in unit of the
characteristic transition energy of the array. If $e < 0$, then its absolute
value is the energy in unit of eV. $x$ and $x1$
set the energy boundary between the distroted-wave Born approximation and
plane-wave Born approximation for the collisional excitation. If $x > 0$ and
the scattered
electron energy is less than $xE_{b}$, DW method is chosen, where $E_{b}$ is
the binding energy of the electron being excited.
If $x$ is negative, the switch between DW and PW occurs when the estimated
high partial-wave contributions becomes larger than $-x$ times the low
partial-waves calculated explicitly. If $x=0$, then PW is always used.
default for $x$ is $-0.5$, i.e., use DW when the estimated high partial-wave
contributions represents no more than 50\% of the explicitly calculated low
partial-wave sum. The contributions of high partial waves for monopole and
dipole transitions are calculated with coulomb-bethe approximation, their
cutoff values are treated differently with the switch \var{x1}, whose default
is $-1.0$.
\end{fundesc}

\begin{fundesc}{SetCEGrid}{g $\mid$ n\opt{, e0, e1}}
Set the collsion energy grid for collsional excitation. In the first form, the
grid is given by a Python list \var{g}. In the second form, the grid is
constructed with \var{n} points, from \var{e0} to \var{e1}. The energies are
specified for the scattered electron energy, and in units of eV. This routine
does not need to be called. A default is constructed for a given transition
array with 6 points, minimum and maximum energies specified by
\key{SetCEGridLimits}. Calling this routine with \var{n}=0, reset the grid to
system default.
\end{fundesc}

\begin{fundesc}{SetCEGridLimits}{e0, e1}
Set the minimum and maximum collision energy for collisional excitation for
the automatic construction of the grid. They are in units of average threshold
energy of the transition array being considered. The default is 0.05 and 8.0
if this routine is not called.
\end{fundesc}

\begin{fundesc}{SetCELCB}{m}
Set the orbital angular momentum for Coulomb-Bethe approximation.
\end{fundesc}

\begin{fundesc}{SetCELMax}{m}
Set the maximum of orbital angular momentum for the partial-wave expansion in
collisional excitation.
\end{fundesc}

\begin{fundesc}{SetCELQR}{m}
Set the maxmum orbital angular momentum for quasi-relativistic approximation
in collisional excitation. The default is 0, i.e., always use
quasi-relativistic approximation.
\end{fundesc}

\begin{fundesc}{SetCEQkMode}{mode}
Set the computation mode for the excitation radial integrals. \var{mode} may
be a string or an integer specifying the mode. These values are listed in the
variable \key{QKMODE}.
\end{fundesc}

\begin{fundesc}{SetCIEGrid}{g $\mid$ n\opt{, e0, e1}}
Set the collsion energy grid for collsional ionization. In the first form, the
grid is given by a Python list \var{g}. In the second form, the grid is
constructed with \var{n} points, from \var{e0} to \var{e1}. The energies are
specified for the total energy of the final electrons, and in units of
eV. This routine does not need to be called. A default is constructed for a
given transition array with 6 points, minimum and maximum energies specified
by \key{SetCIEGridLimits}.
Calling this routine with \var{n}=0, reset the grid to system default.
\end{fundesc}

\begin{fundesc}{SetCIEGridLimits}{e0, e1}
Set the minimum and maximum collision energy for collisional ionization for
the automatic construction of the grid. They are in units of average threshold
energy of the transition array being considered. The default is 0.05 and 8.0
if this routine is not called.
\end{fundesc}

\begin{fundesc}{SetCILCB}{m}
Set the orbital angular momentum for the Coulomb-Bethe approximation in DW
collisional ionization.
\end{fundesc}

\begin{fundesc}{SetCILevel}{m}
Set the level of configuration interaction space. By default, $m=0$, the
configuration space is determined by the configuration groups passed to the
\key{Structure} function. CI can be further refined by the value of $m$. If
$m=-1$, then no CI is included, the Hamiltonian is assumed to be diagonal. If
$m=1$, only CI within the same relativistic configuration is included. If
$m=2$, only CI within the same non-relativisitc configuration is included. If
$m=3$, only CI within the same configuration group is included.
\end{fundesc}

\begin{fundesc}{SetCILMax}{m}
Set the maximum orbital angular momentum for the partial-wave expansion in DW
collisional ionization.
\end{fundesc}

\begin{fundesc}{SetCILMaxEject}{m}
Set the maximum orbital angular momentum for the ejected electron in DW
collisional ionization.
\end{fundesc}

\begin{fundesc}{SetCILQR}{m}
Set the orbital angular momentum for quasi-relativistic approximation in DW
collisional ionization.
\end{fundesc}

\begin{fundesc}{SetCITol}{m}
Set the tolerance factor in the partial-wave expansion of DW collisional
ionization.
\end{fundesc}

\begin{fundesc}{SetCIQkMode}{mode}
Set the computation mode for the ionization radial integrals. \var{mode} may
be a string or an integer specifying the mode. These values are listed in the
variable \key{QKMODE}.
\end{fundesc}

\begin{fundesc}{SetCXTarget}{s}
Set the charge exchange neutral target with chemical symbol $s$, e.g., H, H2,
H2O, etc.
\end{fundesc}

\begin{fundesc}{SetCXEGrid}{n, e0, e1, s, t|e, t}
Set the charge exchange collision energy grid. in the first form, a grid is
setup with $n$ points from $e0$ to $e1$. In the second form, the grid is taken
from the list $e$. If $s$ is 1, the grid is setup in logrithmic scale. If $t$
is 1, the energies are given per ion instead of per AMU.
\end{fundesc}

\begin{fundesc}{SetDisableConfigEnergy}{m}
if \var{m} is 1, the \key{ConfigEnergy} function calls are disabled even if
they are invoked explicitly in the script. If 2, the energy shift only
includes that of the self energy correction.
\end{fundesc}

\begin{fundesc}{SetFields}{b, e, a\opt{, m}}
Set the magnetic and electric fiedls. \var{b} is the magnetic fields in Gauss,
\var{e} is the electrific fields in Volts/cm. \var{a} is the angle between the
magnetic and electric fields. If the optional \var{m} is 1, then the
diamagnetic effects is ignored in the Hamiltonian.
\end{fundesc}

\begin{fundesc}{SetHydrogenicNL}{\opt{n,\opt{l}}}
Set the principle quantum number \var{n} and the orbital angular momentum
\var{l}, beyond which, the hydrogenic approximation for the E1 multipole
integrals should be used. If this routine is not called or the argument is not
given, the default is \var{n}=8, and \var{l}=7.
\end{fundesc}

\begin{fundesc}{SetIEGrid}{g $\mid$ n\opt{, e0, e1}}
Set the ionization threshold energy grid for the collisional ionization. In
the first form, the grid is given by a Python list \var{g}. In the second
form, the grid is constructed with \var{n} points from \var{e0} to
\var{e1}. This routine does not need to be called. A 3 point grid is
constructed according to the transition array being considered by default.
Calling this routine with \var{n}=0, reset the grid to system default.
\end{fundesc}

\begin{fundesc}{SetMaxRank}{k}
Set the maximum rank in the expansion of Slater integrals. The default is 6,
if this routine is not called.
\end{fundesc}

\begin{fundesc}{SetMixCut}{c}
Set cutoff threshold of the mixing basis in the wavefunction. Only the basis
with mixing coefficients greater than \var{c} are included in the wavefunction
expansion. Default is $10^{-5}$.
\end{fundesc}

\begin{fundesc}{SetMS}{nms, sms}
Set flags for normal mass shift and specific mass shift contributions to the
Hamiltonian. 0---disable, 1---enable. Defaults: 1.
\end{fundesc}

\begin{fundesc}{SetNStatesPartition}{\opt{n}}
Set the number of states in one partition. A partition is used to organize
angular integrals. Angular coefficients between partitions are calculated
and tabulated in batches. The more states in partitions, the more efficient
the lookup of these coeffients, and the more memory it takes. The default
value of 512 is normally appropriate.
\end{fundesc}

\begin{fundesc}{SetOptimizeControl}{t, s, m\opt{, p}}
Set the options for radial potential optimization. \var{t} is the tolerance
for the self-consistent field iteration. \var{s} is the stablizer for the
iteration, a number from 0 to 1. \var{m} is the maximum number of iterations
allowed. \var{p}
specifies whether diagnostic information should be printed out during the
optimization. This routine does not need to be called. The default for \var{t}
is $10^{-6}$, \var{s} is determined dynamically according to the
type of ion, \var{m} is 100, and \var{p} is 0 for no printing out of
information.
\end{fundesc}

\begin{fundesc}{SetOptimizeMaxIter}{m}
Set the maximum itneration in \key{OptimizeRadial}.
\end{fundesc}

\begin{fundesc}{SetOptimizePrint}{m}
Set the printing option in \key{OptimizeRadial}.
\end{fundesc}

\begin{fundesc}{SetOptimizeStabilizer}{m}
Set the stablilizer factor in \key{OptimizeRadial}.
\end{fundesc}

\begin{fundesc}{SetOptimizeTolerance}{m}
Set the tolerance factor in \key{OptimizeRadial}.
\end{fundesc}

\begin{fundesc}{SetPEGrid}{g $\mid$ n\opt{, e0, e1}}
Set the free electron energy grid for photoionization, radiative recombination
and autoionization. In the first form, the grid is given by a Python list
\var{g}. In the second form, the grid is constructed with \var{n} points from
\var{e0} to \var{e1}. The energies are in units of eV. This function does not
need to be called. A 6 point grid is constructed according to the transition
array being considered by default. Calling this function with \var{n}=0 reset
the grid to system default.
\end{fundesc}

\begin{fundesc}{SetPEGridLimits}{e0, e1}
Set the minimum and maximum collision energy for photoionization and radiative
recombination for the automatic construction of the grid. They are in units of
average threshold energy of the transition array being considered. The default
is 0.05 and 8.0 if this routine is not called.
\end{fundesc}

\begin{fundesc}{SetRadialGrid}{n\opt{, r0\opt{,r1}\opt{,rmin}}}
Set the radial grid properties. \var{n} is the number of radial grid
points. It must be an even number and less than the macro
\key{MAXRP}. \var{r0} specifies the ratio of successive radial points near
origin, which is approximately logarithmic. \var{r1} specifies the number of
mesh-points per oscillation wavelength for very high-$n$ orbitals at large
radii. \var{rmin} divided by the nuclear charge is the starting point of the
radial mesh.
\end{fundesc}

\begin{fundesc}{SetRRTEGrid}{g $\mid$ n\opt{, e0, e1}}
Set the transition energy grid for photoionization and radiative
recombination. In the first form, the grid is given by a Python list
\var{g}. In the second form, the grid is constructed with \var{n} points from
\var{e0} to \var{e1}. This routine does not need to be called. For E1 type
transitions, the transition energy does not enter the calculation, a 1-point
grid is constructed by default. for other types of multipoles, a 3-point grid
is constructed. Calling this function with \var{n}=0 reset the grid to system default.
\end{fundesc}

\begin{fundesc}{SetRecPWLimits}{l0, l1}
Set the orbital angular momentum range for the spectator electron in the
recombined states to [\var{l0}, \var{l1}] inclusive. The default is [0, 12].
\end{fundesc}

\begin{fundesc}{SetRecPWOptions}{lmax}
Set maximum orbital angular momentum of the spectator electron in the
recombined states to \var{lmax}. The allowed values are also limited by the
setting of \key{SetRecPWLimits}. The default is 12.
\end{fundesc}

\begin{fundesc}{SetRecQkMode}{mode}
Set the computation mode for the photoionization and radiative recombination
radial integrals. \var{mode} may be a string or an integer specifying the
mode. These values are listed in the variable \key{QKMODE}.
\end{fundesc}

\begin{fundesc}{SetRecSpectator}{nmin\opt{, nfrozen}}
Set the minimum principle quantum numbers \var{nmin} and \var{nfrozen} for the
spectator electron. States with $n > nmin$ are constructed with
\key{RecStates} function, and those with $n > nfrozen$ are treated with frozen
core approximation. Default for both \var{nmin} and \var{nfrozen} are 8.
\end{fundesc}

\begin{fundesc}{SetScreening}{ns\opt{,c\opt{,k}}}
Set the orbital parameters of screening electrons. \var{ns} is a list of
integers which are the principle quantum numbers of the screening
orbitals. The optional \var{c} is the total charge to be screened, whose
default is 1.0. \var{k} is the either 1, $-1$, or 0. If \var{k}=$-1$, then
the \var{l=0} orbitals are used. If \var{k}=0, then the the \var{l}=\var{ns}/2
orbitals are used. If \var{k}=1, then the \var{l}=\var{ns}-1 nodeless orbital
is used. The default is \var{k}=1. This function is usually used when
additional screening charge is desired for the mean configuration generating
the optimal central potential. It is quite experimental, and therefore not
recommended for general use.
\end{fundesc}

\begin{fundesc}{SetSE}{n\opt{,m}}
Set the mode of calculatin and maximum principle quantum number of the
orbitals for which the self-energy correction should be included in the
Hamiltonian. Default is 5. \var{n} is the maximum quantum number, \var{m} is
the mode. \var{m=0} use the hydrogenic screening based on the expectation value
of the Uheling potential. \var{m=1} use the screening based on the fermi nuclear
charge distribution (Welton concept). \var{m=2} use the uniform nuclear charge
distribution. \var{m} may contain a 2nd digit, with \var{k=m/10}. \var{k=1}
use the GRASP self energy table for hydrogenic values. \var{k=3} excludes two
loop correction. \var{k=4} use the model qed potential of Shabaev et al. (CPC
189, p175). \var{k=5} same as 4 but excluding the two loop
corrections. \var{k=6} includes the off diagonal matrix element of the self
energy operator in the CI calculation. \var{k=7} same as 6 by excluding the
two loop corrections. If \var{k>4} and \var{m} also has a 3rd digit, then the
local part of the self energy potential is included in the radial Dirac
equation, and only the non-local part is evaluated and added in the
Hamiltonian.
\end{fundesc}

\begin{fundesc}{SetSlaterCut}{k1, k2}
Set the calulation modes of Slater integrals. \var{k1} and \var{k2} are
orbital angular momentum values. When one of the orbitals has $l > k1$, then
exchange integrals are not calculated. When one has $l > k2$, the direct
integrals are evaluated with the multipole moments.
\end{fundesc}

\begin{fundesc}{SetTEGrid}{g $\mid$ n\opt{, e0, e1}}
Set the transition energy grid for collisional excitation. In the first form,
the grid is given by a Python list \var{g}. In the second form, the grid is
constructed with \var{n} points from
\var{e0} to \var{e1}. This routine does not need to be called. A 3-point grid
is constructed by default. Calling this function with \var{n}=0 reset the grid
to system default.
\end{fundesc}

\begin{fundesc}{SetTransitionCut}{c}
Set the cutoff threshold for the radiative transition rates output. Only rates
greater than \var{c} times the total decay rate of the upper level is written
to the output file. The default is $10^{-3}$ if this routine is not called.
\end{fundesc}

\begin{fundesc}{SetTransitionGauge}{m}
Set the gauge for radiative transition.
\end{fundesc}

\begin{fundesc}{SetTransitionMaxE}{m}
Set the maximum rank of electric multipole transitions.
\end{fundesc}

\begin{fundesc}{SetTransitionMaxM}{m}
Set the maximum rank of magnetic multipole transitions.
\end{fundesc}

\begin{fundesc}{SetTransitionMode}{m}
Set the mode for the radiative transition.
\end{fundesc}

\begin{fundesc}{SetTransitionOptions}{g, m}
Set the options for the radiative transition calculation. \var{g} is the gauge
to be used, 1 for Coulomb gauge (velocity form) and 2 for Babushkin gauge
(length form, which is the default). \var{m} is the mode for the multipole
integral, 0 for fully relativistic and 1 for non-relativistic approximation
(the default is 1).
\end{fundesc}

\begin{fundesc}{SetUsrCEGrid}{g $\mid$ n\opt{, e0, e1}}
Set the user collision energy grid for collisional excitation. The collsion
strengths on this grid are output. It is forced to be same as that set by
\key{SetCEGrid} if the \key{QKMODE} is \key{'exact'}.
\end{fundesc}

\begin{fundesc}{SetUsrCIEGrid}{g $\mid$ n\opt{, e0, e1}}
Set the user collision energy grid for collisional ionization. The collsion
strengths on this grid are output.
\end{fundesc}

\begin{fundesc}{SetUsrPEGrid}{g $\mid$ n\opt{, e0, e1}}
Set the user electron energy grid for photoionization and radiative
recombination. The bound-free differential oscillator
strengths on this grid are output. It is forced to be same as that set by
\key{SetPEGrid} if the \key{QKMODE} is \key{'exact'}.
\end{fundesc}

\begin{fundesc}{SetUTA}{m\opt{, ci}}
Set the flag for configuration average models. If $m=1$, all calculations are
carried out in the configuration average approximation. The radiative
transition rates output contains three additional fields, the transition
energy including the UTA shift,
the Gausian standard deviation, and the correction to the line strengths due to
the configuration interaction within the same non-relativisitc
configurations. If $m=0$, which is the default,
the usual detailed term accounting method is used. This function should be
called in the beginning of the script, before \key{Config} function, since it
disables the angular momentum coupling performed by \key{Config}. The optional
argument \var{ci} indicates whether the configuration interaction correction
factors should be included when reading the \key{DB\_TR} files. These
correction factors are always calculated in \key{TransitionTable} irrespective
of the value of \var{ci}.
\end{fundesc}

\begin{fundesc}{SetVP}{vp}
Set the flag for vacuum polarization correction to the
Hamiltonian. 0---disable, 1---only include 2nd order term, 2---include the 4th
order term as well. 3---also include WK term. Default is 3. If the 2nd digit
is 1, then the finite nucleus size is not taken into account in the vacuum
polarizaiton potential. If the 3rd digit is not 0, then the vacuum
polarizaiton potential is included in the radial Dirac equation, instead of
evaluating its expectation values.
\end{fundesc}

\begin{fundesc}{SlaterCoeff}{fn, g, a, b}
Calculate the expansion coefficients of the exchange radial integral
in the Coulomb energy of each state in the
configuration list
\var{g}. The Coulomb energy betwee electrons of a state can be expanded as
\begin{equation}
E = \sum g_kG^k(\alpha\alpha^\prime,\beta\beta^\prime) + \mbox{direct terms},
\end{equation}
where $\alpha$ and $\alpha^\prime$ are the interacting orbitals in the bra and
ket states, which have the same $l$ values. Because we work in the
$jj$-coupling basis, $\alpha$ and $\alpha^\prime$ may have different $j$
values. The results are stored in the text file \var{fn}. These coefficents
are useful in determining the radiative transition rates from level to average
configurations. \var{a} and \var{b} are the orbital lists contain $\alpha$ and
$\beta$, respectively. They are specified as configuration strings.
The format of the file is as follows. For each state, a line
starts with ``\#'' gives the level index, parity, $2J$-value, and the
configuration label. It is followed by a block of lines. In this block, the first
column is the level index. The 2rd, 3th, and 4th columns are the $n$, $kappa$,
$l$ values of the $\alpha$ orbital. The 5th column is either 0 or 1,
indicating whether $\alpha$ and $\alpha^\prime$ have the same $j$ values. The
6-9th  columns are the corresponding values for the $\beta$ orbital. The 10th
column is the expansion coefficients $g_1$. The 11th column is $g_2$. The 12th
column is the the number to be added to $g_1$ to form the relative intensities
of level to configuraiton transitions for dipole transitions. The 13th column
is the number to be added to $g_2$ to form relative intensities for quadrupole
tranisions.
\end{fundesc}

\begin{fundesc}{Structure}{fn, \opt{, hfn}, g\opt{, p\opt{, ip}}}
Diagonalize the Hamiltonian for configurations in the groups \var{g}. The
configurations in the optional groups \var{p} are allowed to interact with
\var{g} but only states within \var{g} are added to the energy level table. If
\var{ip}=0, the interaction between \var{g} and \var{p} are treated exactly,
if \var{ip}=1, this interaction is treated approximately in a way that the
non-diagonal elements within \var{p} are neglected. The energy levels are
output to the file \var{fn}.
If you specify \var{hfn}, the Hamiltonian is saved to the file \var{hfn} in the
binary format.
%
To read the saved Hamiltonian, use \texttt{rfac.read\_ham}.
\end{fundesc}

\begin{fundesc}{Structure}{p, j}
This is the second form of the \key{Structure} function. It distinguishes
itself from the first form because the first argument is an integer. It sets which
$\pi J$ symmetry and/or which level to include in the structure calculation. By
default, all symmetries are processed. But if this function is called with
$p=0$, or 1; $j \ge 0$ or $j$ is a list of integers, then only the specified
symmetry is processed. $j$ or integers in the list if $j$ is a list, is
twice the actual value of the total angular momentum. The symmetry
restrictions specified here also applies to the \key{StructureMBPT} function.
\end{fundesc}

\begin{fundesc}{StructureEB}{fn, g}
Calculate the atomic structure for atom in magnetic and electric fields. The
levels belong to the configuration group \var{g} are allowed to mix in the
external fields.
\end{fundesc}

\begin{fundesc}{StructureMBPT}{efn, hfn, g, kmax, n1, n2, n0}
There are six different forms of this function with different argument
list. This is the primary form, and the rest are documented below. This
function calculates the level energies of the states belonging to the
configuration groups in the \var{g} list. The energy values are stored in the
binary file \var{efn} the same way as \key{Structure} function. The effective
Hamiltonian of the second order MBPT is stored in the file
\var{hfn}. \var{kmax} is the maximum orbital angular quantum number of the
virtual states. \var{n1} and \var{n2} are two lists for the grid of principle
qunatum numbers of the virtual states. The \var{n1} grid is used for one
electron excitations, and the first electron of the double excitations. The
\var{n2} grid is for the second electron of the double excitation. \var{n0} is
an integer specifying the number of configuration groups in
the \var{g} list to be included in the MBPT calculation, and the remaining are
included for all-order perturbation treatment.
\end{fundesc}

\begin{fundesc}{StructureMBPT}{fn, de, eps, g, kmax, n1, n2, n3, n4, gn}
This is the second form of the \key{StructureMBPT} function. This function is
depricated. It examines the
configurations by exciting the electrons in the \var{g} list to orbitals with
principle quantum numbers in the \var{n1} and \var{n2} lists from those in the
\var{n3} and \var{n4} lists. If its estimated
CI mixing to the states in \var{g} is larger than \var{eps}, and the
excitation energy is less than \var{de}, these
configurations are put in to the group named \var{gn}. The final
list of configurations are written to the file \var{fn}. \var{kmax} is the
maximum orbital angular momentum of the excited electrons.
\end{fundesc}

\begin{fundesc}{StructureMBPT}{efn, hfn1, hfns, g, n0}
This is the third form of the \key{StructureMBPT} function. It reads the
effective Hamiltonians in a list of files \var{hfns}, that are produced in
previous \key{StructureMBPT} calls in the first form, and combines the
contributions from different \var{n1} and \var{n2} values for the virtual
state and sum them up to form the total effective Hamiltonian, which is saved
to the text file \var{hfn1}. It then diagonalizes the effective Hamiltonian to
obtain the energy values, which are stored in \var{efn} the same way as in
\key{Structure}. The configuration list \var{g} and \var{n0} must be the same
as those used in generating the \var{hfns} files.
\end{fundesc}

\begin{fundesc}{StructureMBPT}{m}
This is the fourth form of the \key{StructureMBPT} function. If $m$ is an
integer, it sets an option
to indicate whether extrapolation beyond the maximum $n$-values in the
\var{n1} and \var{n2} grid is carried out in forming the effective
Hamiltonian. If $m=0$, no extrapolation, which is the default; if $m>0$, with
extrapolation. $m$ may also be a list of integers, which means that only
levels in this list for each symmetry is to be corrected with MBPT.
\end{fundesc}

\begin{fundesc}{StructureMBPT}{i, m, c}
This is the fifth form of the \key{StructureMBPT} function. It sets three
options. \var{i} indicates inclusion of some part of 3rd order
corrections. \var{m} indicates whether only single excitation or double
excitation should be included. If $m=0$, both single and double excitations
are included;
if $m=1$, only single excitations are included, and if $m=2$, only double
excitations are included. $c$ is the cutof of the mixing coeffients. The
effective Hamiltonian matrix element $h_{ij}$ is not calculated, if
$max(|b_{ki}b_{kj}|) < c$. The default is $c=10^{-4}$.
\end{fundesc}

\begin{fundesc}{StructureMBPT}{o,n}
  This is the sixth form of the \key{StructureMBPT} function. \var{o} is a
  string specifying the orbitals from which single or double excitations to be
  limited in the MBPT calculation. It can be given in either relativistic or
  non-relativistic notation. \var{n} is either an integer or a 2 element
  integer list. It specifies the maximum principle quantum number of the
  single or double excitations. If it is an integer, both single and double
  excitations impose the same limit. If it is a 2 element list, the first
  number is for double excitation, and 2nd number is for single excitation.
\end{fundesc}

\begin{fundesc}{TotalCICross}{ifn, ofn, ilev, energy\opt{, imin, imax}}
Calculate the total collisional ionization cross sections of level \var{ilev}
by reading the data from the \texttt{DB\_CI} database file \var{ifn}. The
results are written to the ASCII file \var{ofn} in a two column
format. The output cross sections are in units of $10^{-20}$ cm$^2$. \var{ofn}
may be ``-'', which means \texttt{stdout}. \var{energy} is a
list or tuple giving the incident electron energies
in eV where CI cross sections are needed. The optional \var{imin} and
\var{imax} specifies the index range of the ionized states. Only ionization to
states with $imin \le i \le imax$ are included.  If \var{imin}$<$0, it is
assumed to be 0, and if \var{imax}$<$0, it is assumed to be the largest index
in the level file. Both \var{imin} and \var{imax} default to -1, i.e., include
all ionized states. Before calling this function, \key{MemENTable()} must
be called.
\end{fundesc}

\begin{fundesc}{TotalPICross}{ifn, ofn, ilev, energy\opt{, imin, imax}}
Calculate the total photoionzation cross sections of level \var{ilev}
by reading the data from the \texttt{DB\_RR} database file \var{ifn}. The
results are written to the ASCII file \var{ofn} in a two column
format. The output cross sections are in units of $10^{-20}$ cm$^2$. \var{ofn}
may be ``-'', which means \texttt{stdout}. \var{energy} is a
list or tuple giving the incident photon energies
in eV where PI cross sections are needed. The optional \var{imin} and
\var{imax} specifies the index range of the ionized states. Only ionization to
states with $imin \le i \le imax$ are included.  If \var{imin}$<$0, it is
assumed to be 0, and if \var{imax}$<$0, it is assumed to be the largest index
in the level file. Both \var{imin} and \var{imax} default to -1, i.e., include
all ionized states. Before calling this function, \key{MemENTable()} must
be called.
\end{fundesc}

\begin{fundesc}{TotalRRCross}{ifn, ofn, ilev, energy\opt{, n0, n1, nmax, imin,
imax}}
Calculate the total radiative recombination cross sections onto level
\var{ilev} by reading the data from the \texttt{DB\_RR} database file
\var{ifn}. The results are written to the ASCII file \var{ofn} in a two column
format. The output cross sections are in units of $10^{-20}$ cm$^2$. \var{ofn}
may be ``-'', which means \texttt{stdout}. \var{energy} is a
list or tuple giving the electron energies in eV
where the RR cross sections are needed. The data for recombination onto $n0 <
n < n1$ and $n > n1$ are not present in the data file \var{ifn}, and the
hydrogenic approximation for $n0 < n < n1$ and $n1 < n \le nmax$ are added to
the calculated cross sections. The optional \var{imin} and \var{imax}
specifies the index range of the recombined states. Only recombination to
states with $imin \le i \le imax$ are included. If \var{imin}$<$0, it is
assumed to be 0, and if \var{imax}$<$0, it is assumed to be the largest index
in the level file. Both \var{imin} and \var{imax} default to -1, i.e., include
all recombined states. Before calling this function, \key{MemENTable()} must
be called.
\end{fundesc}

\begin{fundesc}{TransitionMBPT}{m, n}
\var{m} specifies the maximum multipole for radiative transitions calculated
with MBPT. \var{n} specifies the number transition energy points used in the
relativistic radial multipole integrals.
\end{fundesc}

\begin{fundesc}{TransitionMBPT}{fn, low, up}
This function specify the transitions between \var{low} and \var{up}
configuration groups to be calcualted with MBPT, with transition rates saved
in the file \var{fn}. This function must be issued before the 3rd form of
\key{StructureMBPT}, as the actual calculations are carried out in that
function.
\end{fundesc}

\begin{fundesc}{TransitionTable}{fn, low, up\opt{, m}}
Calculate the weighted oscillator strength and radiative transition rates from
states in \var{up} groups to states in \var{low} groups with multipole type
\var{m}. The default for \var{m} is 0, where we sum up all the multipoles for
the rate. The results are saved to file \var{fn}.
\end{fundesc}

\begin{fundesc}{TRBranch}{fn, upper, lower}
Looking up the radiative decay rate from \var{upper} to \var{lower} and the
total decay rate of \var{upper} in the binary file \var{fn}. It returns a
Tuple consisting of the transition energy, partial decay rate, and total decay rate.
\end{fundesc}

\begin{fundesc}{TRTable}{fn, low, up\opt{, m}}
Same as \key{TransitionTable}
\end{fundesc}

\begin{fundesc}{TRTableEB}{fn, lo, up\opt{, m}}
Calculate radiative transition rates between levels in external magnetic and
electric fields.
\end{fundesc}

\begin{fundesc}{TRRateH}{z, n0, l0, n1, l1\opt{, m}}
Calculate the radiative transition rate or weighted oscillator strength of
the transition from state (\var{n1, l1}) to state (\var{n0,l0}) in the
non-relativistic hydrogenic approximation for the ion with charge \var{z}. If
\var{m}=0, the transition rate is returned, which is the default, if
\var{m}=1, the weighted oscillator strength is returned, and if \var{m}=2, the
dipole radial integral is returned.
\end{fundesc}

\begin{fundesc}{WaveFuncTable}{fn, n, k\opt{, e}}
Print the radial wavefunction of the orbital with the principle quantum number
\var{n}, relativistic angular quantum number $\kappa$=\var{k} to the file
\var{fn}. If \var{n}=0, the orbital is a continuum state. In this case, the
optiontal \var{e} must be a positive number for the energy of the continuum
orbital in unit of eV.
\end{fundesc}

\begin{fundesc}{Y5N}{n, l, r}
Calculate the Seaton's $y5$ (normalized to give the Coulomb wavefunction)
function for negative energies. This is basically the Coulomb wavefunction
with principle quantum number \var{n} and orbital angular momentum \var{l} at
radius \var{r}. It returns a tuple (y5, y5p, err), where y5 is the value, y5p
is the derivative. err is an error code returned by coulcc, which is called
internally to do the calculation.
\end{fundesc}

\section{\mod{crm}--Collisional Radiative Model}
\label{sec:crm}
\index{crm}
The module \mod{crm} implements a collisional radiative spectral model for
optically thin plasmas. It uses an iterative linear equation solver to invert
the level population equations. This method is capable of including a very
large number of atomic states in the model.

\subsection{Functions}
\begin{fundesc}{AddIon}{n, den, pref}
Add an ion to the spectral model. \var{n} is the number of electrons for the
ion to be added. \var{den} is the density of the ion, and \var{pref} is the
base file name for the binary data files associated with this ion. The
standard file extensions are assumed. i.e., \key{.en} for \key{DB\_EN},
\key{.tr} for \key{DB\_TR}, \key{.ce} for \key{DB\_CE}, \key{.rr} for
\key{DB\_RR}, \key{.ai} for \key{DB\_AI}, \key{.ci} for \key{DB\_CI},
\key{.sp} for \key{DB\_SP}, and \key{.rt} for \key{DB\_RT}. This function is
called multiple times to add more than one ion to the model. However, the order
must be such that the number of electrons are in consecutive increasing order.
\end{fundesc}

\begin{fundesc}{Cascade}{}
Carry out the cascade iteration. Some of the levels in the spectral model may
be treated approximately using the cascade matrix.
\end{fundesc}

\begin{fundesc}{CBeli}{Z, n, E}
Calculate the ionization cross sections using the Aladdin database. data were
compiled by Bell et al. (J. Phys. Chem. Ref. Data., 12, 891, 1983). \var{Z} is
the nuclear charge of the ion. \var{n} is the number f electrons, and \var{E}
is the electron energy in eV. It returns a tuple of length 4. First element is
the total ionization cross section, 2nd the excitation autoionization
contribution, 3rd the direct ionization contribution, and the 4th is a relative
error estimate in pencentages.
\end{fundesc}

\begin{fundesc}{CFit}{Z, n, Te}
Calculate the ionization rates using the Fortran subroutine \verb|cfit|
of D. A. Verner for direct ionization, and \verb|colfit| for exciation
autoionization. \var{Z} is the nuclear charge of the ion, \var{n}
is the number of electrons, and \var{Te} is the electron temperature in eV.
It returns a tuple of length 3. First element is the total ionization rate
coefficients in $10^{-10}$ cm$^{3}$ s$^{-1}$., 2nd is the excitation
autoionization contribution, and the 3rd is the direct ionization contribution.
\end{fundesc}

\begin{fundesc}{CheckEndian}{\opt{fn}}
Check the byte order of database file \var{fn}. It returns 0 for little endian
and 1 for big endian. If the optional file name \var{fn} is omitted, the
endian for the current platform is returned. This function exists in module
\mod{fac} as well.
\end{fundesc}

\begin{fundesc}{CloseSCRM}{}
Close the file containing the SFAC input file converted from the current
Python script. This function must be called after \key{ConvertToSCRM}. Only the
statements between the call to \key{ConvertToSCRM} and \key{CloseSCRM} are
converted to SFAC input file. This routine is only available in PFAC interface.
\end{fundesc}

\begin{fundesc}{ColFit}{Z, n, Te\opt{, s}}
Calculate the direct ionization rates and excitation autoionization rates
using the Fortran subroutine \verb|colfit| of D. A. Verner. \var{Z} is the
nuclear charge of the ion, \var{n} is the number of electrons, and \var{Te} is
the electron temperature in eV. The optional argument \var{s} specifies the
subshell for which the rates are to be calculated. 0 for total. 1 for $1s$, 2
for $2s$, 3 for $2p$, 4 for $3s$, 5 for $3p$, 6 for $3d$ and 7 for $4s$. The
function returns a tuple of length 3. The first element is the total rate, the
second is the excitation autoionization rate, and the third is the direct
ionization rate. The result is in unit of $10^{-10}$ cm$^3$ s$^{-1}$.
\end{fundesc}

\begin{fundesc}{ConvertToSCRM}{}
Converts the statements between this call and the \key{CloseSCRM} call to SFAC
input file \var{fn}. The resulting file can then be run using the \key{scrm}
executable. This routine is only available in PFAC interface.
\end{fundesc}

\begin{fundesc}{CxtDist}{fn, n}
Calculate the charge exchange collision energy distribution function, and
print it to file \var{fn}, using \var{n} energy points. The distribution must be already set.
\end{fundesc}

\begin{fundesc}{DRBranch}{}
Calculate the radiative branching ratios of all autoionizing states. It is
calculated iteratively using:
\begin{equation}
B_i = \frac{\sum_j A^r_{ij}B_j}{\sum_j A^a_{ij} + \sum_k A^r_{ik}},
\end{equation}
where $A^r_{ij}$ is the radiative decay rate from state $i$ to state $j$,
$A^a_{ik}$ is the autoionization rate from state $i$ to state $k$. The
branching ratios of non-autoionizing states is 1.0. This function must be
called before \key{DRStrength()} with \var{mode}=0.
\end{fundesc}

\begin{fundesc}{DRFit}{Z, n, Te}
Calculate the dielectronic recombination rates using the fitting formula and
data table of P. Mazzotta. \var{Z} is the nuclear charge of the ion, \var{n}
is the number of electrons, and \var{Te} is the electron temperature in eV.
The result is in unit of $10^{-10}$ cm$^3$ s$^{-1}$.
\end{fundesc}

\begin{fundesc}{DRStrength}{fn, n\opt{, m, i}}
Tabulate the dielectronic recombination resonance strengths onto state \var{i}
of the ion with \var{n} electrons. \var{fn} specifies the file name of the
database file. The optional \var{m} indicates the mode of calculation. If
\var{mode}=0, then the total DR strengths are calculated; if \var{m}=1, then
individual DR satellite lines are caclulated, and if \var{m}=2, then resonance
excitation is calculated. \var{i} is optional, whose default is 0. When \var{i}$\ge$0,
then it is the index of the recombining state relative to the ground state
of the recombining ion; when \var{i}$<$0, then its negative value is the true
index of the recombining state.
\end{fundesc}

\begin{fundesc}{DumpRates}{fn, nele, m\opt{, imax\opt{, a}}}
Dump rate coefficients to a file. \var{fn} is the output file name. \var{nele}
is the number of electrons of the ion whose rates are needed. \var{m}
specifies which rate is output.
\begin{description}
\item[$m=0$]
Dump the level indexes, 2J values, energies, etc.
The record for each level contains 11 fields of the type: short, int, int,
int, short, short, short, double double, double, double. They are number of electrons
of the ion, level index, block index of the level, level index within the
block, 2J value of the level, base level of the level, valence $nl$ of the
level, energy, population, total decay rate, and branching ratio computed in
\key{DRBranch} (if that has been called) respectively.
\item[$m=1$]
Dump the radiative transition rates. Each record contains 4 fields of type:
int, int, double, double. They are upper level index, lower level index, decay
rate, and photo-excitation rate, respectively.
\item[$m=2$]
Dump the 2-photon transition rates. Each record contains 4 fields of type:
int, int, double, double. They are upper level index, lower level index, decay
rate, and inverse rate, respectively.
\item[$m=3$]
Dump the collisional excitation rates. Each record contains 4 fields of type:
int, int, double, double. They are lower level index, upper level index, excitation
rate, and deexcitation rate, respectively.
\item[$m=4$]
Dump the radiative recombination rates. Each record contains 4 fields of type:
int, int, double, double. They are continuum level index, recombined level index,
recombination rate, and photoionization rate, respectively.
\item[$m=5$]
Dump the autoionization rates. Each record contains 4 fields of type:
int, int, double, double. They are bound level index, continuum level index,
autoionization rate, and dielectronic capture rate, respectively.
\item[$m=6$]
Dump the collisional ionization rates. Each record contains 4 fields of type:
int, int, double, double. They are bound level index, continuum level index,
ionization rate, and three-body recombination rate (not implemented), respectively.
\end{description}
The units of the transition rates are s$^{-1}$, and those of the rate
coefficients are $10^{-10}$ cm$^{3}$ s$^{-1}$.
The optional augument \var{imax} indicates the maximum levels to be included
in the dump. a negative number means all levels, which is the default. \var{a}
specifies whether the file should be in a binary format (0) or ascii format
(1). Default is binary format.
\end{fundesc}

\begin{fundesc}{EBeli}{Z, n}
Calculate the ionization threshold of ion with nuclear charge \var{Z} and
number of electrons \var{n}. It returns a double in eV. Data are from aladdin
data base.
\end{fundesc}

\begin{fundesc}{EColFit}{Z, n, s}
Calculate the ionization threshold of ion with nuclear charge \var{Z} and
number of electrons \var{n} for shell \var{s}. \var{s} has the same meaning as
in \key{ColFit}. It returns a double in eV. Data are from subroutine
\key{ColFit}.
\end{fundesc}

\begin{fundesc}{EleDist}{fn, n}
Calculate the electron energy distribution function, and print it to file \var{fn},
using \var{n} energy points. The distribution must be already set.
\end{fundesc}

\begin{fundesc}{EPhFit}{Z, n, s}
Calculate the ionization threshold of ion with nuclear charge \var{Z} and
number of electrons \var{n} for shell \var{s}. \var{s} has the same meaning as
in \key{PhFit}. It returns a double in eV. Data are from subroutine
\key{PhFit}.
\end{fundesc}

\begin{fundesc}{FracAbund}{Z, Te\opt{im, rm}}
Calculate the fractional abundance of the charge states for element with
nuclear charge \var{Z} at temperature \var{Te} for collisionally ionized
plasmas. The optional arguments \var{im} and \var{rm} specifies the functions
used for ionization rates and recombination rates respectively. \var{im} is
passed to the function \key{Ionis}, and \var{rm} is passed to the function
\key{Recomb}. The default for both parameters are 1. It returns a list of
length \var{Z}+1 indexed by the number of electrons the ion have.
\end{fundesc}

\begin{fundesc}{InitBlocks}{}
Initialize the superlevel blocks of the spectral model.
\end{fundesc}

\begin{fundesc}{IonDensity}{fn, k}
Read the \key{DB\_SP} file \var{fn} and return the total density of the ion
with number of electrons \var{k}.
\end{fundesc}

\begin{fundesc}{Ionis}{Z, n, Te\opt{, m}}
Caculate the ionization rate coefficients for the ion with nuclear charge
\var{Z} and number of electrons \var{n}, at the temperature \var{Te}. It
returns a list of length 3. The first element is the total ionization rate,
the second is the excitation-autoionization rate, and the third is the direct
ionization rate. The data used are from \citet{arnaud85, arnaud92} if the
optional argument \var{m} is 0. The function \key{ColFit} is used if \var{m}
is 1. The function \key{CFit} is used for the direct ionization rate and
\key{ColFit} for the autoionization rate, if \var{m} is 2. The data from
Aladdin database is used if \var{m} is 3. The default is 1.
The results are in unit of $10^{-10}$ cm$^3$ s$^{-1}$.
\end{fundesc}

\begin{fundesc}{LevelPopulation}{}
Solve the level population using the superlevel block method.
\end{fundesc}

\begin{fundesc}{MaxAbund}{Z, n\opt{,im, rm, a}}
Calculate the temperature where the fractional abundance of the ion with
nuclear charge \var{Z} and number of electrons \var{n} reaches maximum. The
optional arguments \var{im} and \var{rm} specifies the functions used for
ionization rates and recombination rates respectively. \var{im} is passed to
the function \key{Ionis}, and \var{rm} is passed to the function
\key{Recomb}. The default for both parameters are 1. The parameter \var{a}
specifies the relative accuracy for the solution, which has the default of
$10^{-4}$. This functions returns a tuple of length 2. The
first element is the temperature found, the second is a list for the
fractional abundances of all ions with nuclear charge \var{Z} at the found
temperature.
\end{fundesc}

\begin{fundesc}{NormalizeMode}{m}
Set the mode for normalizing the ion densities. If \var{m} is 0, the density
of the ground state of each ion is fixed at the value given by
\key{SetAbund}. If \var{m} is 1, the total density of the ion is fixed at that
value.
\end{fundesc}

\begin{fundesc}{NDRFit}{Z, n, Te}
Calculate the dielectronic recombination rates using the data calculated with
FAC for H-like through Ne-like ions of Mg, Si, S, Ar, Ca, Fe, and Ni. \var{Z}
is the nuclear charge of the ion, \var{n} is the number of electrons, and
\var{Te} is the electron temperature in eV. The result is in unit of
$10^{-10}$ cm$^3$ s$^{-1}$.
\end{fundesc}

\begin{fundesc}{NRRFit}{Z, n, Te}
Calculate the radiative recombination cross sections using the Fortran
subroutine \verb|nrrfit| using the data calculated with FAC for Bare through
F-like ions of Mg, Si, S, Ar, Ca, Fe. \var{Z} is the nuclear charge of the
ion, \var{n} is the number of electrons of the recombining ion, and \var{Te}
is the electron temperature in eV. The result is in unit of $10^{-10}$ cm$^3$
s$^{-1}$.
\end{fundesc}

\begin{fundesc}{PhFit}{Z, n, E, s}
Calculate the photoionization cross section at energy \var{E} of subshell
\var{s} for the ion with nuclear charge \var{Z}, and number of electrons
\var{n}. The meaning of \var{s} are: 1 for $1s$, 2
for $2s$, 3 for $2p$, 4 for $3s$, 5 for $3p$, 6 for $3d$ and 7 for $4s$.
The Fortran subroutine \verb|phfit2| of D. A. Verner is used. The result is in unit of $10^{-20}$ cm$^2$
\end{fundesc}

\begin{fundesc}{PhoDist}{fn, n}
Calculate the photon energy distribution function, and print it to file \var{fn},
using \var{n} energy points. The distribution must be already set.
\end{fundesc}

\begin{fundesc}{PlotSpec}{ifn, ofn, n, t, e0, e1, de\opt{, s}}
Print the spectrum of lines in the \key{DB\_SP} file \var{ifn} of a given
type \var{t} to the file \var{ofn}. See \key{SelectLines} for the meaning of
this parameter except when $t=0$, in which case all lines are included in the
output. The lines are convolved with a Gausian
with FWMH of \var{de} in unit of eV. \var{e0} and \var{e1} are the spectral
range to be considered. The optional \var{s} gives a cutoff
threshold for the lines. Lines weaker than \var{s} times the strongest line
are not included.
\end{fundesc}

\begin{fundesc}{Print}{args}
Print out the string representation of \var{args}. This function exists to
asist the conversion to SFAC interface, since Python's \key{print} statement
is not converted.
\end{fundesc}

\begin{fundesc}{PrintTable}{fnb, fna\opt{, v}}
Convert the binary database file \var{fnb} to the ASCII file \var{fna}. The
optional argument \var{v} = 1 requires the conversion be done in verbose
mode, otherwise it is done in simple mode. Note that before conversion in
verbose mode is carried out, one must call \key{MemENTable} first. This
function also exists in the module \mod{fac}.
\end{fundesc}

\begin{fundesc}{RateTable}{fn\opt{, cfg\opt{, m}}}
Output rates for all processes included in the spectral model to the
\key{DB\_RT} database file \var{fn}. It may contain an additional argument,
which is a list of complex names. \var{m} controls the amount of information
to be output.
\end{fundesc}

\begin{fundesc}{RBeli}{Z, n, T}
Calculate the ionization rate coefficients using the Aladdin data base. data
were  compiled by Bell et al. (J. Phys. Chem. Ref. Data., 12, 891,
1983). \var{Z} is the nuclear charge of the ion. \var{n} is the number f
electrons, and \var{T} is the electron temperature in eV. It returns a tuple
of length 3. First element is the total ionization rate coefficients in
$10^{-10}$ cm$^{3}$ s$^{-1}$., 2nd is the excitation autoionization
contribution, and the 3rd is the direct ionization contribution.
\end{fundesc}

\begin{fundesc}{ReadKronos}{d, z, k, p, t\opt{, m, l, s}}
Read Kronos charge exchange db files, and load data into CRM model. \var{d} is
the top level path of the database. \var{z} is the nuclear charge of the ion,
\var{k} is the number of electrons of the initial ion, \var{p} is the atomic
symbol of the ion, \var{t} is the chemical symbol of the target, \var{m} is
the method to use, \var{l} is the $l$-distribution model, \var{s} indicate if
interpolation of cross sections to be performed in logrithmic scale.
\end{fundesc}

\begin{fundesc}{Recomb}{Z, n, Te\opt{, m}}
Caclulate the recombination rate coefficients for the ion with nuclear charge
\var{Z} and number of electrons \var{n} at temperature \var{Te}. It returns a
tuple of length 3. The first element is the total recombination rate, the
second is the radiative recombination rate, and the third is the dielectronic
recombination rate. The data used are from \citet{arnaud85, arnaud92} if the
optional argument \var{m} is 0. The functions \key{RRFit} and \key{DRFit}
are used if \var{m} is 1. If \var{m} is 2, then the New RR and DR rate
coefficients are used for Bare through Ne-like ions of Mg, Si, S, Ar, Ca, Fe,
and Ni calculated with \key{NRRFit} and \key{NDRFit}, for other ions,
\key{RRFit} and \key{DRFit} are used. The default is 1. The results are in unit
of $10^{-10}$ cm$^3$ s$^{-1}$.
\end{fundesc}

\begin{fundesc}{ReinitCRM}{\opt{m}}
Reinitialize the module \mod{crm}. \var{m}=0 by default, which requests a full
reinitialization. If \var{m}=1, only the database headers are reinitialized
and the rates are cleared. If \var{m}=2, only the database headers are
reinitialized and the rates that depend on electron energy distribution are
cleared. If \var{m}=3, only the database headers are reinitialized.
\end{fundesc}

\begin{fundesc}{RRFit}{Z, n, Te}
Calculate the radiative recombination cross sections using the Fortran
subroutine \verb|rrfit| of D. A. Verner. \var{Z} is the nuclear charge of the
ion, \var{n} is the number of electrons, and \var{Te} is the electron
temperature in eV. Note that in this routine, \var{n} is the number of
electrons of the recombining ion, while in the original subroutine of Verner,
it is for the recombined ion. The result is in unit of $10^{-10}$ cm$^3$
s$^{-1}$.
\end{fundesc}

\begin{fundesc}{RRRateH}{Z, n, Te}
Caculate the radiative recombination rates of H-like ions with nuclear charge
\var{Z} at temperature \var{Te}. \var{n} is the principle quantum number of
the recombined electron. This function returns a tuple of length 2. The first
element is the RR rate on to all states with principle quantum number
\var{n}. The second is the RR rate on to all states with principle quantum
numbers $>$\var{n}.
\end{fundesc}

\begin{fundesc}{SelectLines}{ifn, ofn, n, t, e0, e1\opt{, s}}
Print the selected lines from the \key{DB\_SP} database file \var{ifn} to the
file \var{ofn}. \var{n} is the number of electrons of the ion. \var{e0} and
\var{e1} are the energy range in units of eV. \var{s} is the cutoff threshold
for the lines. If \var{s}$>=$1, then \var{e0} and \var{e1} are interpreted as
the lower and upper level indexes for the line selected. In this case a single
line will be output, and \var{t} is ignored. Otherwise, \var{t} is the type of
the lines to be selected. If \var{t} is 0, then all transition types are
allowed. Otherwise, the value of \var{t} is decomposed into 4 fields,
say, \var{t0}, \var{t1}, \var{t2}, and \var{t3}, where \var{t0} is the lowest 2
decimal digits of \var{t}, \var{t1} is the next 2 digits, and so on. e.g., if
\var{t}=1000201, then \var{t0}=1, \var{t1}=2, \var{t2}=0, and
\var{t3}=1. If \var{t3}=0, then the lines have type equal to
10000\var{t2}+100\var{t1}+\var{t0} are selected. If \var{t3}=1, then lines of
type 10000\var{q}+100\var{t1}+\var{t0} with \var{q}$\ge$\var{t2} are
selected. If \var{t3}$<$0, then lines of type \var{q} with
\var{q}$\ge$\var{t0} are selected. The physical meaning of line types are
discussed in \S\ref{subsec:sp_header}
\end{fundesc}

\begin{fundesc}{SetAIRates}{inv}
Set autoionization rates in the spectral model. If \var{inv}=1, the rates for
the inverse process, dielectronic capture, are also set.
\end{fundesc}

\begin{fundesc}{SetAIRatesInner}{fn}
Read the AI file \var{fn} to obtain the inner Auger transition rates. This
file must only contain the the rates of such transitions. and the energy level
indexes of the bound states must be continuous, and have exactly the same
order as the corresponding levels in the level table already setup.
\end{fundesc}

\begin{fundesc}{SetAbund}{n, den}
Set the abundance of the ion with number of electrons \var{n} to
\var{den}. This overrides the settings given by \key{AddIon}.
\end{fundesc}

\begin{fundesc}{SetBlocks}{\opt{den, pref}}
Read the energy levels and setup the superlevel blocks for the spectral
model. The optional \var{den} is the abundance of the ion with one less
electron than the lowest charge state included in the model, and \var{pref} is
the base file name for that ion. If \var{den} is $< 0$, only processes
connects the levels within the same ion are included. If \var{pref} is not
specified, the transitions in that ion are not included in the model, which
may cause convergence problems in some cases.
\end{fundesc}

\begin{fundesc}{SetCERates}{inv}
Set the collisional excitation rates. If \var{inv}=1, the inverse process,
collisional deexcitation rates are also set.
\end{fundesc}

\begin{fundesc}{SetCIRates}{inv}
Set the collisional ionization rates. If \var{inv}=1, the inverse process,
three-body recombination rates are also set (not implemented yet).
\end{fundesc}

\begin{fundesc}{SetCascade}{c\opt{, a}}
If \var{c}=1,  some levels should be treated in cascade approximation. The
optional \var{a} specifies the accuracy in the cascade iteration. The default
is $10^{-4}$.
\end{fundesc}

\begin{fundesc}{SetCxtDensity}{den}
  Set the charge exchange neutral taget density in unit of $10^{10}$ cm$^{-3}$.
\end{fundesc}

\begin{fundesc}{SetCxtDist}{i, ...}
  Set the charge exchange collision energy distribution.
\end{fundesc}

\begin{fundesc}{SetCXRate}{m|t}
  Set the charge exchange rates. If $m=2$, the charge exchange cross sections
  are read from FAC \texttt{DB\_CX} data file. If $m=0$, the cross sections
  are from Kronos db read by a previous ReadKronos call. In this mode, the
  data for the ion must be present in the Kronos db. If $m=1$, the bare charge
  exchange cross sections in the Kronos db is converted into fine-structure
  cross sections with the help of the FAC \texttt{DB\_RO} data. If $m\ge 10$,
  the second digit indicates if the energy is given per ion instead of per
  AMU. \var{t} specifies the charge exchange neutral target's chemical
  symbol. It is used to select the appropriate cross sections in the
  \texttt{DB\_CX} data file when $m=2$. Finally, if $m\ge 100$, the charge
  exchange rates are setup so that only one $nl$ electron is captured with
  unit rate, where $m=n\times 100+l$.
\end{fundesc}

\begin{fundesc}{SetEleDensity}{den}
Set the electron density in unit of $10^{10}$ cm$^{-3}$.
\end{fundesc}

\begin{fundesc}{SetEleDist}{i, ...}
Set the electron energy distribution. The first argument \var{i} specifies the
type of distributions, and the remaining give the parameters of the
distribution. The avaliable distributions and their parameters are listed in
\S\ref{subsec:rt_header}. In addition of these distributions, one can use a
text file to define a distribution. It is specified with $i=-1$ and a file
name as the parameters. The text file must contain three integers in the first
line, specifying the number of energy points, $n$, the scale of energy and
distribution (1 for log scale, 0 for linear scale). It is then followed by $n$
lines giving energy in eV, and distribution function values.
\end{fundesc}

\begin{fundesc}{SetExtrapolate}{i}
Set if the extrapolation of the high-n levels should be carried out. If
$i$ is negative, then no extrapolate is carried out, if $i$ is non-negative,
then only the DR channels $<= i$ will be included in the extrapolation. The DR
channels are defined in the order KLn, KMn, ... for K-shell ions, and LLn,
KLn, LMn, MMn, ... for L-shell ions. If unsure how to extrapolate, set it to
-1 to disable it.
\end{fundesc}

\begin{fundesc}{SetInnerAuger}{i}
Set if the inner-shell Auger transitions should be extrapolated to those
levels that are not explicitly calculated. There are several different ways
such extrapolation can be done. \var{i}=0 indicates no
extrapolation. \var{i}=1 is to extrapolate by averaging the Auger rates of the
last calculated complex. \var{i}=2 is to extrapolate by reading an extra file
that contains the Auger rates in the core, which must be supplied by calling
\key{SetAIRatesInner}. \var{i}=3 is to extrapolate by reading the same AI file
that contains the normal AI rates. In this case, the AI file must contain
those rates between the core. \var{i}=4 is to read the rates from the AI file
of the next higher charge state.
\end{fundesc}

\begin{fundesc}{SetIteration}{a\opt{, s, m}}
Set the options for population iteration. \var{a} is the accuracy, which
defaults to $10^{-4}$. \var{s} is a stablizer defaults to 0.75, and \var{m} is
the maximum number of iterations allowed, which defaults to 256.
\end{fundesc}

\begin{fundesc}{SetNumSingleBlocks}{n}
Set the number of states in the beginning of the energy table that should not
be grouped into a super block.
\end{fundesc}

\begin{fundesc}{SetPhoDensity}{den}
Set the photon energy density in unit of erg cm$^{-3}$.
\end{fundesc}

\begin{fundesc}{SetPhoDist}{i, ...}
Set the photon energy distribution. The first argument \var{i} specifies the
type of distributions, and the remaining give the parameters of the
distribution. The avaliable distributions and their parameters are listed in
\S\ref{subsec:rt_header}. A text file as described in \key{SetEleDist} can
also be used to specify the distribution.
\end{fundesc}

\begin{fundesc}{SetRRRates}{inv}
Set the radiative recombination rates. If \var{inv}=1, the inverse process,
photoionization rates are also set.
\end{fundesc}

\begin{fundesc}{SetRateAccuracy}{r\opt{, a}}
Set the accuracy for the numerical integration in the calculation of rate
coefficients. \var{r} is the relative accuracy, \var{a} is the absolute
accuracy. The default for \var{r} is 0.01, that for \var{a} is $10^{-8}$ in
unit of $10^{-10}$ cm$^3$ s$^{-1}$.
\end{fundesc}

\begin{fundesc}{SetTRRates}{inv}
Set the radiative transition rates. If \var{inv}=1, the inverse process,
photo-excitation rates are also set.
\end{fundesc}

\begin{fundesc}{SetUTA}{m\opt{, ci}}
Set the flag for configuration average models. If $m=1$, all calculations are
carried out in the configuration average approximation. The radiative
transition rates output contains three additional fields, the transition
energy including the UTA shift,
the Gausian standard deviation, and the correction to the line strengths due to
the configuration interaction within the same non-relativisitc
configurations. If $m=0$, which is the default,
the usual detailed term accounting method is used. This function should be
called in the beginning of the script, before \key{Config} function, since it
disables the angular momentum coupling performed by \key{Config}. The optional
argument \var{ci} indicates whether the configuration interaction correction
factors should be included when reading the \key{DB\_TR} files. These
correction factors are always calculated in \key{TransitionTable} irrespective
of the value of \var{ci}.
\end{fundesc}

\begin{fundesc}{SpecTable}{fn\opt{, rrc}}
Output the level populations and line emissivities to the \key{DB\_SP}
database file \var{fn}. If \var{rrc}=1, the radiative recombination continuum
strength should also be included. If \var{rrc}=$-1$, only level populations are
output.
\end{fundesc}

\begin{fundesc}{TwoPhoton}{z, t}
Calculate the two-photon decay rate of H-like and He-like transitions
$2s_{1/2}\to 1s_{1/2}$ and $1s2s S_{0}\to 1s^2 S_{0}$ for nuclear charge
\var{z}. \var{t} = 0 is for H-like, and 1 is for He-like.
\end{fundesc}

\section{\mod{pol}--Line Polarizations}
\label{sec:pol}
\index{pol}
This module is used to calculate line polarizations due to directional
electrons. It takes into account the radiative cascades effects
if the atomic data provided include the necessary transitions.
\subsection{Functions}
\begin{fundesc}{CloseSPOL}{}
Close the file containing the SFAC input file converted from the current
Python script. This function must be called after \key{ConvertToSPOL}. Only the
statements between the call to \key{ConvertToSPOL} and \key{CloseSPOL} are
converted to SFAC input file. This routine is only available in PFAC interface.
\end{fundesc}

\begin{fundesc}{ConvertToSPOL}{}
Converts the statements between this call and the \key{CloseSPOL} call to SFAC
input file \var{fn}. The resulting file can then be run using the \key{spol}
executable. This routine is only available in PFAC interface.
\end{fundesc}

\begin{fundesc}{Orientation}{\opt{etrans\opt{,fn}}}
Calculate the orientation parameters with optional transverse energy component \var{etrans}. This routine must be called after \key{PopulationTable} and
before \key{PolarizationTable}. The orientation parameters are output in file
\var{fn} if it is given.
\end{fundesc}

\begin{fundesc}{PolarizationTable}{fn\opt{, ifn}}
Calcuates the line polarizations, and output the results to file \var{fn} in a
simple ASCII format. It contains 8 columns, which are number of electrons in
the ion, upper level index, lower level index, multipole type (as in the
output of \key{TransitionTable}), transition energy in eV, total line
emissivity, emission anisotropy factor at $90^\circ$, and linear polarization,
respectively. \var{ifn} is an optional file name which specifies the
transitions whose polarizations are to be calculated. It should contain 4
columns, number of electrons, upper level, lower level, and multipole type,
respectively. Each of the columns may be in the form ``\verb|*|'', which
matches any transitions. The first 3 columns may also specify a range in the
form, e.g., ``0-5'', which matches any number between 0 and 5 inclusive. Either
limit of the range may also be ``\verb|*|'', so that that limit is not
enforced. The second argument may also be given as a list of strings instead
of a file name, which correspond to rows of the equivelent file.
\end{fundesc}

\begin{fundesc}{PopulationTable}{fn}
Calculate the magnetic sublevel populations, and
output the results to file \var{fn} in a simple ASCII format. For each level,
it tabulates the total population, the orientation parameters $B_\lambda$,
and magnetic sublevel fractions.
\end{fundesc}

\begin{fundesc}{Print}{args}
Print out the string representation of \var{args}. This function exists to
asist the conversion to SFAC interface, since Python's \key{print} statement
is not converted.
\end{fundesc}

\begin{fundesc}{SetDensity}{d}
Set the electron density in $10^{10}$ cm$^{-3}$ for collsional radiative
model.
\end{fundesc}

\begin{fundesc}{SetEnergy}{e\opt{, s}}
Set the electron energy \var{e} in eV. If the optional \var{s} is positive,
the energy distribution is a Gaussian with standard deviation $\sigma=s$ in
eV, and mean $e$.
\end{fundesc}

\begin{fundesc}{SetIDR}{idr\opt{, p}}
Set if the DR satellites polarization to be calculated for a specific target
level \var{idr}. The routine must be called after \key{SetMLevels}. If
\var{idr} is set to be a valid level, then only DR onto this target will be
allowed. The optional \var{p} specifies the sublevel
population of \var{idr}. \var{p} must be a List or Tuple with number of
elements equal to the number of sublevels of level \var{idr}. In counting
number of sublevels, $\pm M$ are treated as a single sublevel. When \var{p} is
not given, sublevels are assumed to be equally populated.
\end{fundesc}

\begin{fundesc}{SetMIteration}{a\opt{, m}}
Set the accuracy \var{a} and maximum iteration \var{m} for iterative solution
of the level population.
\end{fundesc}

\begin{fundesc}{SetMaxLevels}{m}
Set the maximum number of levels to be retained in the rate matrix. Levels
higher than that are treated iteratively. 0 means retain all levels.
\end{fundesc}

\begin{fundesc}{SetMAIRates}{fn}
Set the magnetic sublevel autoionization rates and dielectronic capture
rates by reading data from binary file \var{fn}
\end{fundesc}

\begin{fundesc}{SetMCERates}{fn}
Setup the magnetic sublevel excitation rates. Excitation cross sections are
interpolated from the binary data file \var{fn}.
\end{fundesc}

\begin{fundesc}{SetMLevels}{efn, tfn}
Setup the magnetic sublevel table and the radiative transitions rates between
them. \var{efn} is the binary data file for energy levels, and \var{tfn} is
that of transition rates.
\end{fundesc}

\section{\mod{util}--Utility Functions}
\label{sec:util}
\index{util}
This module contains some unitity functions which may be useful in various
situations.
\subsection{Functions}
\begin{fundesc}{Spline}{x, y\opt{, dy1, dy2}}
Prepare cubic spline table. \var{x} and \var{y} are the independent and
dependent variables to be interpolated. \var{dy1} and \var{dy2} specifies the
optional boundry conditions at the two ends in the form of first
derivatives. The default is to use natural cubic spline. It returns a list of
second derivatives to be used in \key{Splint}.
\end{fundesc}

\begin{fundesc}{Splint}{x, y, y2, x0}
Calculate the interpolated value at a single point \var{x0}. \var{y2} is a
list of second derivatives returned by \key{Spline} using the same \var{x} and
\var{y}. This routine and \key{Spline} are adapted from Numerical Recipe.
\end{fundesc}

\begin{fundesc}{UVIP3P}{x, y, x0\opt{,n}}
Local piece-wise 3rd polynomial interpolation from \var{x}, \var{y} to a list
of independent values in the list \var{x0}. If \var{n} is present, an
\var{n}-th order polynomial interpolation function will be used instead of 3rd
order. It returns a list of the results. The Akima interpolation method is
used.
\end{fundesc}

\section{\mod{config}--Electronic Configuration Specification}
\label{sec:config}
\index{config}
This module is written in pure Python. It is deprecated now. It includes three
main functions. \key{closed}, \key{config}, and \key{avgconfig}. They have the
same calling syntax as \key{Closed}, \key{Config}, and
\key{AvgConfig} in the module \mod{fac}. One should always use these later
ones instead of the counterparts in the \mod{config} module.

\section{\mod{const}--Physical Constants}
\label{sec:const}
\index{const}
This module defines some useful physical constants. They are listed below:
\begin{verbatim}
Hartree_eV = 27.2113962            # Hartree in eV
Rate_AU = 4.13413733E16            # Atomic Rate Unit in s-1
Rate_AU10 = 4.13413733E06          # Atomic Rate Unit in 10^10 s-1
Rate_AU12 = 4.13413733E04          # Atomic Rate Unit in 10^12 s-1
Area_AU20 = 2.80028560859E3        # Atomic Area Unit in 10^{-20} cm2
Alpha = 7.29735308E-3              # Fine Structure Constant
Ryd_eV = 13.6056981                # Rydberg in eV
RBohr = 0.529177249                # Bohr radius in A
FWHM = 2.35482005                  # conversion from sigma to FWHM
hc = 1.239842E4                    # hc in eV*A
hbc = 1.97327053E3                 # h_bar*c in eV*A
Me_eV = 5.1099906E5                # electron mass in eV
Me_keV = 5.1099906E2               # electron mass in keV
Mp_MeV = 9.38271998E2              # proton mass in MeV
Mp_keV = 9.38271998E5              # proton mass in keV
c  = 2.99792458E10                 # speed of light in cm/s
c10 = 2.99792458                   # speed of light in 10^10 cm/s
e = 1.60217733E-19                 # electron charge in Coulomb
e19 = 1.60217733                   # electron charge in 10^-19 Coulomb
erg_eV = 6.241506363E-13           # erg in eV
re = 2.81794092                    # electron classical radius in fm
sig_t = 6.6524616                  # Thompson cross section in 10^{-25} cm2
kb = 8.617385E-5                   # Boltzman constant in ev/K
\end{verbatim}

\section{\mod{table}--Text Tabulation}
\label{sec:table}
\index{table}
\subsection{Format of Text Table}
\label{subsec:format}
The \key{TABLE} class implemented in this module creates and manipulates
text tables in a format similar to those of machine-readable tables in AAS
journals. It is independent of FAC, and may be used elsewhere. A table
contains a header providing the byte-by-byte description of the columns and
other explanatory materials. The body of the table are arranged in the
conventional multi-column format. The following list shows an example created
by \key{TABLE}:
\begin{verbatim}
Title:   Total Ionization and Recombination Rate Coefficients
Authors: M. F. Gu
========================================================================
byte-by-byte description of file: trates.tbl
------------------------------------------------------------------------
     Bytes Format                Units      Label Explanation
------------------------------------------------------------------------
    1-   2     I2                 None       NELE Num. of Electrons
    4-   7   F4.2                  [K]       Temp Temperature
    9-  16   E8.2       10^-10^cm^3^/s         DR Total DR rate coefficients
   18-  25   E8.2       10^-10^cm^3^/s      DR_AR Total DR Arnaud & Raymond
   27-  34   E8.2       10^-10^cm^3^/s         RR Total RR rate coefficients
   36-  43   E8.2       10^-10^cm^3^/s      RR_AR Total RR Arnaud & Raymond
   45-  52   E8.2       10^-10^cm^3^/s         CI Total DCI rate coefficients
   54-  61   E8.2       10^-10^cm^3^/s     DCI_AR Total DCI Arnaud & Raymond
   63-  70   E8.2       10^-10^cm^3^/s         EA Total EA rate coefficients
   72-  79   E8.2       10^-10^cm^3^/s      EA_AR Total EA Arnaud & Raymond
------------------------------------------------------------------------
 2 6.80 3.77E-05 3.77E-05 5.94E-02 5.55E-02 0.00E+00 1.44E-09 0.00E+00 0.00E+00
 2 6.95 3.10E-04 3.05E-04 4.45E-02 4.14E-02 0.00E+00 1.93E-07 0.00E+00 0.00E+00
 2 7.10 1.25E-03 1.21E-03 3.31E-02 3.06E-02 0.00E+00 6.32E-06 0.00E+00 0.00E+00
 2 7.25 3.00E-03 2.88E-03 2.44E-02 2.26E-02 0.00E+00 7.80E-05 0.00E+00 0.00E+00
 2 7.40 4.89E-03 4.66E-03 1.78E-02 1.65E-02 0.00E+00 4.79E-04 0.00E+00 0.00E+00
 2 7.55 6.03E-03 5.69E-03 1.28E-02 1.20E-02 0.00E+00 1.78E-03 0.00E+00 0.00E+00
 2 7.70 6.05E-03 5.66E-03 9.11E-03 8.71E-03 0.00E+00 4.62E-03 0.00E+00 0.00E+00
 2 7.85 5.23E-03 4.86E-03 6.42E-03 6.27E-03 0.00E+00 9.01E-03 0.00E+00 0.00E+00
 2 8.00 4.07E-03 3.75E-03 4.47E-03 4.49E-03 0.00E+00 1.49E-02 0.00E+00 0.00E+00
 3 6.80 1.08E-01 1.25E-01 5.49E-02 4.98E-02 2.11E-03 3.21E-03 1.84E-07 2.42E-07
 3 6.95 8.63E-02 1.01E-01 4.10E-02 3.70E-02 7.32E-03 1.06E-02 5.61E-06 7.08E-06
 3 7.10 6.61E-02 7.70E-02 3.03E-02 2.73E-02 1.82E-02 2.50E-02 6.07E-05 7.43E-05
 3 7.25 4.91E-02 5.51E-02 2.21E-02 2.00E-02 3.57E-02 4.57E-02 3.14E-04 3.79E-04
 3 7.40 3.58E-02 3.76E-02 1.59E-02 1.46E-02 5.89E-02 7.14E-02 9.65E-04 1.16E-03
 3 7.55 2.58E-02 2.47E-02 1.14E-02 1.06E-02 8.55E-02 9.81E-02 2.05E-03 2.52E-03
 3 7.70 1.83E-02 1.58E-02 8.00E-03 7.62E-03 1.13E-01 1.23E-01 3.37E-03 4.27E-03
 3 7.85 1.26E-02 9.91E-03 5.56E-03 5.46E-03 1.39E-01 1.45E-01 4.63E-03 6.12E-03
 3 8.00 8.44E-03 6.12E-03 3.82E-03 3.88E-03 1.61E-01 1.62E-01 5.64E-03 7.81E-03
 4 6.80 1.86E-01 1.79E-01 4.71E-02 4.42E-02 5.44E-03 8.22E-03 3.89E-07 0.00E+00
 4 6.95 1.46E-01 1.40E-01 3.51E-02 3.27E-02 1.79E-02 2.58E-02 1.17E-05 0.00E+00
 4 7.10 1.09E-01 1.03E-01 2.58E-02 2.40E-02 4.31E-02 5.87E-02 1.25E-04 0.00E+00
 4 7.25 7.89E-02 7.20E-02 1.88E-02 1.75E-02 8.24E-02 1.04E-01 6.36E-04 0.00E+00
 4 7.40 5.52E-02 4.84E-02 1.35E-02 1.27E-02 1.33E-01 1.59E-01 1.92E-03 0.00E+00
 4 7.55 3.79E-02 3.15E-02 9.54E-03 9.13E-03 1.90E-01 2.15E-01 4.02E-03 0.00E+00
 4 7.70 2.55E-02 2.00E-02 6.67E-03 6.53E-03 2.46E-01 2.64E-01 6.50E-03 0.00E+00
 4 7.85 1.68E-02 1.24E-02 4.59E-03 4.65E-03 2.97E-01 3.03E-01 8.82E-03 0.00E+00
 4 8.00 1.09E-02 7.65E-03 3.12E-03 3.28E-03 3.38E-01 3.31E-01 1.07E-02 0.00E+00
\end{verbatim}

\subsection{Class Attributes and Methods}
The \key{TABLE} class has a number of attributes, which may be set during or
after the initialization, and a few method functions for the creation and
manipulation of the table:

\begin{vardesc}{fname}
The file name of the text table. It may be set through the keyword argument
\var{fname} when the \key{TABLE} instance is created.
\end{vardesc}

\begin{vardesc}{title}
The title of the table. Set through the keyword \var{title}.
\end{vardesc}

\begin{vardesc}{authors}
A list of string for the authors of the table. Set through the keyword
\var{authors}.
\end{vardesc}

\begin{vardesc}{date}
A string for the date when the table is created. During creation, the date
returned by the Python function \key{time.localtime()} is used. It may be set
to other values though the keyword \var{date}.
\end{vardesc}

\begin{vardesc}{separator0}
The string separates the title and authors information from the byte-by-byte
description. May be set through the keyword \var{separator0}. The default is
\key{'='*72}.
\end{vardesc}

\begin{vardesc}{separator}
The string separates the header and the body. Set through the keyword
\var{separator}. The default is \key{'-'*72}.
\end{vardesc}

\begin{fundesc}{add\_column}{\textnormal{**}c}
Add a column to the table. The variable length keyword arguments specify the
column attributes. It must contain \var{label} and \var{format}. A label is a
short identifier of the column. The format is a string starts with \key{A},
\key{I}, \key{F}, or \key{E} for characters, integers, decimal floating
points, and exponential floating points, and followed by a width and possible
precision specfication. e.g. \key{'A10'}, \key{'F10.3'}, and
\key{'E11.4'}. Other possible keywords are \var{unit}, \var{description} and
\var{note}. \var{unit} specifies the unit of the column (use the string
\key{'None'} for dimensionless quantities). \var{description} is a short
description of the column. \var{note} are some lengthy explanation for the
column which cannot be fit in the byte-by-byte description section. The notes
are arranged in the end of the table header.
\end{fundesc}

\begin{fundesc}{open}{mode\opt{, fname}}
Open the file associated with the table. \var{mode} is either \key{'r'} for
read or \key{'w'} for write. The optional \var{fname} changes the file name of
the table.
\end{fundesc}

\begin{fundesc}{close}{}
Closes the file associated with the table.
\end{fundesc}

\begin{fundesc}{write\_header}{}
Write the table header to the file.
\end{fundesc}

\begin{fundesc}{write\_row}{...}
Write a row of data to the file. It must contain as many arguments as the
columns, and the arguments are arranged in the same order as the columns were
added.
\end{fundesc}

\begin{fundesc}{read\_header}{}
Read the header of the table.
\end{fundesc}

\begin{fundesc}{read\_columns}{index\opt{, filter, start, stop}}
Read columns from the table. \var{index} is a list of integers for columns to
be read counting from 0. \var{filter} is a Python logical expression to be
evaluated for each row. Only rows that pass this expression are read. The k-th
column is referred to as \key{c[k]} in the expression. e.g.,
\var{filter}=\key{'c[0]$>$0'} only reads the rows that have the first column
greater than 0. \var{start} is the row
number where the read should begin, and \var{stop} is the row number where the
read should stop, all counting from 0. If \var{stop}=$-1$, read through the
end of the table. This function returns a list with each element being the
column read.
\end{fundesc}

\begin{fundesc}{convert2tex}{fn, index\opt{, filter, start, stop}}
Convert the table to LaTeX format and save the results to file \var{fn}. The
remaining arguments have the same meaning as those in
\key{read\_columns}. Note that this function only adds the LaTex column
delimiter and the line break for the table contents. The results are meant to
be cut and pasted to LaTex documents.
\end{fundesc}

\begin{fundesc}{rewind}{}
Reset the file position to the start of table body.
\end{fundesc}

\subsection{Example}
The example table in \S\ref{subsec:format} is generated with the following
script:
\begin{verbatim}
from pfac.table import *

####
# create an instance of the class.
####
tbl = TABLE(fname=dfile,
	    title='Total Ionization and Recombination Rate Coefficients',
            authors=['M. F. Gu'])
####
# add each column to the table.
####
d = 'Num. of Electrons'
tbl.add_column(label='NELE', unit='None',
               description=d, format='I2')
d = 'Temperature'
tbl.add_column(label='Temp', unit='[K]',
               description=d, format='F4.2')
d = 'Total DR rate coefficients'
tbl.add_column(label='DR', unit='10^-10^cm^3^/s',
               description=d, format='E8.2')
d = 'Total DR Arnaud & Raymond'
tbl.add_column(label='DR_AR', unit='10^-10^cm^3^/s',
               description=d, format='E8.2')
d = 'Total RR rate coefficients'
tbl.add_column(label='RR', unit='10^-10^cm^3^/s',
               description=d, format='E8.2')
d = 'Total RR Arnaud & Raymond'
tbl.add_column(label='RR_AR', unit='10^-10^cm^3^/s',
               description=d, format='E8.2')
d = 'Total DCI rate coefficients'
tbl.add_column(label='CI', unit='10^-10^cm^3^/s',
               description=d, format='E8.2')
d = 'Total DCI Arnaud & Raymond'
tbl.add_column(label='DCI_AR', unit='10^-10^cm^3^/s',
               description=d, format='E8.2')
d = 'Total EA rate coefficients'
tbl.add_column(label='EA', unit='10^-10^cm^3^/s',
               description=d, format='E8.2')
d = 'Total EA Arnaud & Raymond'
tbl.add_column(label='EA_AR', unit='10^-10^cm^3^/s',
               description=d, format='E8.2')
####
# open the file to write.
####
tbl.open('w')

####
# write the table header
####
tbl.write_header()

####
# write each row of data here
# ......
####

tbl.close()

\end{verbatim}

\section{\mod{atom}--Application of \mod{fac} Module}
\label{sec:atom}
\index{atom}
This is a quite extensive application of \mod{fac} module to the
calculation of atomic processes for K-shell and L-shell ions. It implements a
class \key{ATOM} that may be customized through its attributes. The details of
this class is not to be documented, since this a highly specialized case. The
users are encouraged to write their own scripts to deal with specific
problems. Hower, one function from this module is described here which may
come to be handy to obtain atomic data for K-shell and L-shell ions.
\begin{fundesc}{atomic\_data}{nele, asym\opt{, iprint, dir, \textnormal{**}kw}}
This function calculates all collsional and radiative atomic data for K-shell
and L-shell ions. The atomic processes includes radiative transition,
collisional excitation, photoionization, radiative recombination,
autoionization, and collisional ionization. \var{nele} is a list of integers
specify the number of electrons for each ion to be calculated. \var{asym} is
the elemental symbol of the atom to be calculated. \var{iprint} specifies if
the binary data files need to be converted to ASCII format. If it is $-1$, no
conversion is carried out, if it is 0, the conversion is in simple format,
if it is 1, the conversion is in verbose format. The default is 1. \var{dir}
is an existing directory where the data files to be stored. Each ion will have
a directory named \key{asym\#\#} under this directory, where \key{\#\#} is a
2-digit number equal to the number of electrons the ion has. The remaining
keywords are used to specify the processes to be excluded in the calculation
and other parameters customizing the \key{ATOM} class. Here we only mention 5
of them, \key{no\_ce}, \key{no\_tr}, \key{no\_rr}, \key{no\_ci}, and
\key{no\_ai}. If any of these keywords is set to 1, the corresponding processes
will be ignored in the calculation. By default, none of them are set. Note
that in typical situations, the computation of collsional excitation and
autoionization takes most of the time. If these data are not needed,
\key{no\_ce} and \key{no\_ai} should be set to 1.
\end{fundesc}

\section{\mod{spm}--Application of \mod{crm} Module}
\label{sec:spm}
\index{spm}
This is an application of \mod{crm} module for the spectral modeling. Most of
the functions in this module are highly specialized, therefore not documented
here. One of them, \key{spectrum}, is relatively more general, and with minor
modifications may be used in various situations to construct simple spectral
models for single temperature, optically thin plasmas. The inclusion of this
module in the distribution is mainly for demonstration purpose. Anyone
interested has to read the source code directly.

\chapter{Frequently Asked Questions (FAQ) To FAC}
\section{General}
\faq{Where can I obtain FAC}{
FAC is free software. It can be used, modified and redistributed
without restriction. Currently, it can be obtained from anonymous
\textbf{https://github.com/flexible-atomic-code/fac}, or one may request to
\textbf{mfgu@ssl.berkeley.edu} for a copy through email.}

\faq{What operating systems does FAC run on}{
FAC is written in a mixture of ANSI C, Fortran 77, and Python, all of them are
in principle platform independent. However, the mixed language programming and
the dynamically loadable Python modules makes it more easyly installed in
modern UNIX-like systems than others. So far, it has been tested to work under
solaris, linux, Mac OS X, and windows with the UNIX API emulation provided by
Cygwin.}

\faq{How does FAC differ from other atomic codes}{
The theoretical methods used in FAC are similar to some other distorted-wave
atomic codes, such as HULLAC and differ from more elaborate programs based on
close-coupling approximations, such as the Belfast R-Matrix code. The biggest
advatage of FAC is its ease of use and its scriptability.}

\faq{How does FAC differ from plasma synthetic codes}{
There are various plasma codes widely used in X-ray astronomy, such as APEC,
MEKAL, SPEX, XSTAR and Cloudy. These codes use the existing atomic data to
construct spectral models under different physical conditions. FAC is an
atomic code, whose primarily purpose is to generate atomic data, which can be
used in these plasma codes. However, FAC includes a collisional radiative
model that is able to compute spectral models for optically thin plasmas at a
given electron temperature and density. Non-Maxwellian distribution of electron
energy may be easily implemented as well. A power law ionizing continuum
radiation may also be included.}

\faq{Can FAC be incorporated into XSPEC or other spectral analysis programs}{
In principle, the collisional radiative model comes with FAC can be used in
XSPEC or similar spectral analysis programs. However, this is not practicle,
as the model inlcudes a large number of atomic states, especially, the doubly
excited states to treat the resonant processes, and is therefore very time
consumming. The best strategy of incorporating the FAC results to external
spectral models is to extract basic atomic or plasma parameters and use them
to build table models or implement dedicated subroutines.}

\faq{Which atomic processes can or cannot be calculated with FAC}{
FAC can calculate energy levels, radiative transition rates of arbitrary
multipole type, collisional excitation and ionization cross sections by
electron impact, photionization and radiative recombination cross sections and
autoionization rates. In the currect form, FAC does not treat two-photon
decay, although such decay rates of $2s S_{1/2}$ state of H-like ions and
$1s2s S_{0}$ state of He-like ions are included in the collisional radiative
model using interpolation formulae taken from literature. The three-body
recombination are not implemented as well.}

\faq{What are the typical accuracies of the atomic parameters caculated with
FAC}{
The ions other than H-like, the accuracy of energy levels are usually a few
eV, which translates to 10--30 m{\AA} for the wavelength at $\sim$10{\AA}. For
radiative transition rates and cross sections, the accuracies are
$\sim$10--20\%. Data for near-neutral ions or atoms may have even larger
errors.}

\faq{Can FAC be used to calculate atomic parameters for non-X-ray (UV,
optical, etc.) lines}{
For multiply-charged ions, non-X-ray lines usually result from the transitions
within the same complex, which usually have large relative uncertainties in
the calculated wavelengths and transition rates. The accuracy for UV and
optical lines from near-neutral ions is also very limited.}

\faq{Are there standard references to FAC}{
Currently, no papers have been published describing the code, though
drafts have been written, which are included in the FAC distribution. I have
been trying to find a suitable journal willing to publish
them. I originally submitted them to Computer Physics Communications
(CPC). However, the referee and editor complained about the lack of
documentation and code comments. The documentation has improved since
then, the code commenting still needs extensive work, which may not be done
soon. The first paper that used results of FAC is Gu, M.F., 2003, ApJ, 582,
1241, which can be used as the reference.}

\faq{How do I report bugs, make suggestions and get updated about new
versions}{
Please raise an issue on \textbf{https://github.com/flexible-atomic-code/fac}
for bugs and suggestions.
I maintain a small email address list of people who expressed
interest in FAC and send release anouncements to them. Let me know if you want
to be added to this list. If the list ever grows to the point when I can no
longer put it under my personal address book, we may have to create a
dedicated mailing list.}

\section{Atomic Structure}
\faq{How do I specify a bare ion}{
The bare ion is indicated by a call to \key{fac.Config('', group='b')}, i.e.,
the first argument of \key{Config} is an empty string.}

\faq{Which configurations should be used as basis for the mean cofiguration
which optimizes the radial potential}{
The function \key{fac.OptimizeRadial} accepts a list of configurations, which
form the basis for the construction of the mean configuration. Usually, only
the lowest lying configurations should be used in \key{fac.OptimizeRadial} for
the construction of mean configuration. I have found that using configurations
corresponding to the ground complex is always a good idea. Sometimes, it maybe
worthwile to inlcude the first excited configurations. However, it is always a
bad practice to inlcude very highly excited configurations, especially those
inner shell excited ones.}

\faq{Can I use a specific mean configuration to be used in potential
optimization}{
The function \key{fac.AvgConfig} may be used to set the mean configuration for
the potential optimization. In this case, the function
\key{fac.OptimizeRadial} must be called with no arguments.}

\faq{How do I know what mean configuration is used in the potential
optimization, If one is not given specificly by \key{AvgConfig}}{
The function \key{fac.GetPotential} may be called after \key{OptimizeRadial}
to obtain the mean configuration used, and the resulting radial potential.}

\faq{When calculating ionization or recombination processes, should I use the
mean configuration for recombined (ionizing) or recombining (ionized) ion}{
Usually, it does not matter for highly charged ions, and I usually use the
recombined ion. The difference of one more or less electron screening the
nuclear charge may be substaintial for low-$Z$ elements, in which case, one
may have to make a decision by comparing the results to experimental values or
other theoretical works.}

\faq{How do I determine which configurations should be interacting}{
This depends on the computer resource available, and the desired accuracy. The
dimension of the Hamiltonian matrix increases very rapidly as the number of
interacting configurations grows. The convergence with respect to the
configuration interaction is usually slow. For applications which FAC is
primirily designed for, one typically inlcudes only configurations within the
same complex, except for some low-lying configurations.}

\faq{What relativistic effects are included}{
The standard Dirac-Coulomb Hamiltonian is used in FAC, which means that the
spin-orbit interaction, mass-effect and other leading relativistic effects are
fully treated. However, higher-order QED effects, such as retardation and
recoil are only included in the Breit interaction with zero energy limit for
the exchanged photon. Vacuum polarization and self-energy corrections are
treated in the screened hydrogenic approximation.}

\faq{Why the transition rate between two specific states is not in the output
file, although it should have been calculated}{
Not every calculated transition rate are output. Some weak transitions are
discarded to avoid very large files. A small number, which may be set by the
function \key{fac.SetTransitionCut}, controlls this behavior. If the
transition rate of $2\to 1$ devided by the total decay rate of state 2 is less
than this number, this rate is not output. The default for this number is
$10^{-4}$.}

\section{Collisional Excitation}
\faq{Why is there multiple data blocks in the output corresponding to a single
call of \key{CETable} sometimes}{
Sometimes, the transition array corresponding to a given call to \key{CETable}
include transitions with a wide range of excitation energies. This typically
happens for transitions within a single complex or transitions between more
than 2 complexes are mixed together in one call of \key{CETable} (which should
generally be avoided). Since the excitation radial integrals are only
calculated on a few-point transition energy grid, it is undesirable to have a
very wide range in the actual transition energies. \key{CETable} avoid this by
subdivide the transitions in groups. Within each group, the transition
energies does not vary by more than a factor of 5. A different transition
energy grid and collision energy grid are used for different groups, and
therefore, corresponding to different data blocks in the output.}

\faq{Why is the default \key{QKMODE} for excitation is \key{EXACT}, not
\key{FIT}}{
It used to be in \key{FIT} mode. However, the fitting formulae sometimes fail
to reproduce the calculated collision strengths. After playing with different
fitting formulae for a while, I decided that the user should do the fitting
(if one is desired) on the case by case basis. The function \key{SetCEQkMode}
may be used to specify a different \key{QKMODE}.}

\faq{Can I use a different collision energy grid}{
A collision energy grid in terms of the energy of the scattered electron is
automatically constructed if one is not specified prior to calling
\key{CETable}. One may use the function \key{SetCEGrid} to specify a different
grid. Or one may use \key{SetUsrCEGrid} to have a user grid different from the
grid on which the collision strengths are calculated, and use
\key{INTERPOLATE} mode for the \key{QKMODE}.}

\faq{The collision strengths at very high energies are incorrect}{
Due to the limited radial grid size, the collision strengths at energies much
higher than the excitation energy (more than a few hundred times higher) are
unlikely to be reliabe. However, the high energy collision strengths should
not be calculated directly. The Bethe and Born limit parameters in the output
should be used to obtain them.}

\section{Photoionization and Radiative Recombination}
\faq{When can I use the function \key{RecStates} to construct the
recombined states instead of specifying their configurations by \key{Config}}{
The function \key{RecStates} can only be used if the free electron is captured
to an empty orbital.}

\faq{Why is the bound-free oscillator strength differ from some other
theoretical calculation by a constant factor}{
The bound-free differential oscillator strengths calculated by FAC have units
of Hartree$^{-1}$. Also the values depend on the normalization of continuum
orbitals. It is therefore possible that they differ from other theorectial
calculations by a constant factor. One should always use the formula in this
manual or the accompanying papers to convert them to photoionization or
radiative recombination cross sections.}

\faq{Can I use only the fitting formula and ignore the tabulated $gf$
values}{
The fitting formula and the parameters given for the bound-free oscillator
strengths is only valid at high energies (beyond the largest energy of the
photo-electron energy grid). This means that the $gf$ values at energies
within the photo-electron energy grid should be calculated by interpolation
instead of the fitting formula. However, this is often only necessary for the
ionization of valence shells of near neutral ions and atoms, since only in
these cases, the near threashold behavior of the $gf$ values differ from the
fitting formula significantly.}

\section{Autoionization and Dielectronic Recombination}{
\faq{What does the channel number in the call to \key{AITable} mean}{
It does not mean anything. It is simply an identifier which may be useful
sometimes to tag a certain autoionization channel.}

\faq{How do I improve the resonance energies of low-lying $\Delta n = 0$
resonances}{
The energyies of low-lying $\Delta n = 0$ resonances can not the calculated
accurately. This greatly reduces the reliability of the resulting dielectronic
recombination rates. In FAC, there is a method to improve the accuracy of
these energies by adjusting the core transition energies according to the
experimental values. The function \key{CorrectEnergy} is used to modify the
calculated energy levels by specified amount.}

\faq{Why is it advised to make seperate calls to \key{AITable} for different
bound or free state complexes}{
The autoionization radial integrals are only calculated for a few free
electron energies. The actual values are interpolated from them. This only
works well if the the free electron energy after autoionization does not vary
widely. Therefore, one should avoid calling \key{AITable} with bound or free
states in different complexes. e.g., KLL and KLM resonances should be
calculated with two seperate calls to \key{AITable}.}

\section{Collisional Ionization}
\faq{Why are FAC ionization cross sections calculated with \key{BED} mode
usually much smaller at near threshold energies as compared with
distorted-wave calculations}{
When using \key{BED} mode to calculate the ionization radial integrals, the
total ionization cross sections are scaled by a factor $E/(E+I)$, where $E$ is
the energy of incident electron, and $I$ is the ionization threshold
energy. The result of this scalling is usually desirable as distorted-wave
method overestimates the near threshold cross sections.}

\faq{Why is the \key{DW} mode so slow as compared with \key{BED} and \key{CB}
modes}{
In the \key{DW} mode, the ionization radial integrals are calculated by
summing up partial wave contributions. This is a time consuming process as
now there are two continuum electrons involved. In the \key{CB} mode, the
radial integrals are simply looked up in a table, is therefore the fastest
method. In the \key{BED} mode, the radial integrals are calculated using the
bound-free differential oscillator strengths, which can be computed much
faster than the \key{DW} ionization radial integrals.}

\section{Collisional Radiative Model}
\faq{What kinds of electron energy distributions are built in the \mod{crm}
module}{
Currently, the Maxwellian and Gaussian energy distributions are supported for
electrons. The Gaussian distribution is meant to simulate a monoenergetic
electron beam with finite energy width.}

\faq{Are photoionization and photo-excitation supported}{
A power law ionization continuum can be used for the photoionization and
photo-excitation sources. However, the radiative transfer effects are not
implemented.}

\faq{How can I add more electron energy and/or photon energy distributions}{
Take a look at the C header and source files \texttt{faclib/rates.h} and
\texttt{faclib/rates.c}, and follow the implementation of built-in
distributions. The FAC package must be recompiled and relinked for added
distributions to work.}

\bibliographystyle{plain}
\bibliography{papers/facref}

\printindex
\end{document}
