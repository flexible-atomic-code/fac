\documentclass[12pt,preprint]{aastex}
\setlength{\oddsidemargin}{0in}
\setlength{\topmargin}{12pt}
\setlength{\textwidth}{6.5in}
\setlength{\textheight}{9.0in}


\newcommand{\threej}[6]{\ensuremath{\left({#1\atop #4}{#2\atop #5}
{#3\atop #6}\right)}}
\newcommand{\sixj}[6]{\ensuremath{\left\{{#1\atop #4}{#2\atop #5}
{#3\atop #6}\right\}}}

\begin{document}

\pagestyle{plain}
\setcounter{page}{1}
\pagenumbering{roman}

\section{Atomic Theory in External Electric and Magnetic Fields}
\label{sec:theory}
The magnetic fields we are concerned with, $10^5$--$10^8$~G, and the associated
electric fields either intrinsic to mCVs or the motion-induced 
Lorentz fields, are
considered to be in the weak-field regime, in the sense that the 
Zeeman and Stark energy splitting of atomic
ions is much smaller than the Coulomb interaction between the electrons and the
nucleus. However, the magnitude of electron-electron correlation energies may
be comparable or even smaller than the interaction with the external
fields. Therefore, the two effects must be taken into account in the
theoretical treatment with the same
level of detail. The natural method of solving this problem
is the configuration
interaction approximation, where the effects of both electron correlation and
interaction with the external fields are included in all order within a limited
configuration space.

For atomic ions in a field-free environment, the total angular 
momentum, $J$, and
parity, $\pi$, are good quantum numbers. Therefore, the total 
Hamiltonian is block
diagonal when basis states of definite $J\pi$ symmetries are used. The
magnetic field
breaks the spherical symmetry, although $\pi$ and the angular momentum
projection on the field direction are still conserved. The electric field in
a direction different from that of the magnetic field
further destroys the cylindrical and mirror symmetries, and no quantum number
other than the energy can be considered conserved. The Hamiltonian matrix in
the external fields is therefore typically much larger than in the zero-field
case.

Relativistic effects can be important for highly charged ions of
interest in this study, and we will start from the Dirac Hamiltonian to solve
the atomic structure
\begin{equation}
H = H_0 + H_B^{(1)} + H_B^{(2)} + H_E, 
\end{equation}
where
\begin{eqnarray}
H_B^{(1)} &=& \sum_i \mu_B\left(2\vec{S}_i + \vec{L}_i\right)\cdot\vec{B}
\nonumber\\
H_B^{(2)} &=& \sum_i \frac{1}{2}\mu_B^2\left|\vec{B}\times\vec{r}_i\right|^2
\nonumber\\
H_E &=& \sum_i \vec{E}\cdot\vec{r}_i.
\end{eqnarray}
The summation in the above equations is over all electrons, $H_0$ is the field-free
Hamiltonian, $H_B^{(1)}$ is the linear Zeeman term, $H_B^{(2)}$ is the
diamagnetic Zeeman term, $H_E$ is the interaction with the electric field,
$\vec{S}_i$, $\vec{L}_i$, $\vec{r}_i$, are the spin angular momentum, orbital
angular momentum, and position operators of the $i$-th electron, $\mu_B =
5.788\times 10^{-5}$~eV/T is the Bohr magneton, and $\vec{E}$ and $\vec{B}$ are the
electric and magnetic field vectors.

In the configuration interaction approximation, the wavefunction of the system
is assumed to be $\psi = \sum_i b_i \phi_i$, where $\phi_i$ is the
antisymmetrized product of single-electron Dirac wavefunctions for any given
electronic configuration. These single-electron wavefunctions are normally
derived from self-consistent Dirac-Fock calculations without consideration of
external fields. The mixing coefficients, $b_i$, and the energy value
associated with the total wavefunction $\psi$ are the eigenvalue solution of
the Hamiltonian matrix in the representation of $\phi_i$ basis. The matrix
elements of the Hamiltonian are given by $H_{ij} = <\phi_i|H|\phi_j>$.

The first step in solving the atomic structure is therefore to construct the
Hamiltonian matrix with a suitable set of basis wavefunctions. Because we are
only concerned with the weak-field regime in this project, the basis
wavefunctions are derived using the same method used as in the zero-field
approximation. In particular, we will make use of the existing 
Flexible Atomic Code (FAC, see \S\ref{sec:implementation} for more details)
as a general framework, within which the new atomic theory will be
implemented. The calculation of Hamiltonian matrix elements of $H_0$ is also
the same as in the zero-field case. Therefore, we will only need to develop
new codes to obtain matrix elements of $H_B^{(1)}$, $H_B^{(2)}$, and $H_E$.
These matrix elements are more easily calculated by converting the vector
products into spherical tensor form, and separating the radial and angular
integrations using the angular momentum theory.

Using the spherical tensor components of a vector
\begin{eqnarray}
T_1 &=& -\frac{1}{\sqrt{2}}\left(V_x+iV_y\right) \nonumber\\
T_0 &=& V_z \nonumber\\
T_{-1} &=& \frac{1}{\sqrt{2}}\left(V_x-iV_y\right).
\end{eqnarray}
The $H_E$ term is rewritten as
\begin{eqnarray}
\label{eq:he}
H_E &=& \sum_q (-1)^q E_q r_iC^1_{-q}(i) \nonumber\\
     &=& \sum_q (-1)^q E_q \sum_{\alpha\beta}
     Z^1_{-q}(\alpha\beta)<\alpha||C^1||\beta>r,
\end{eqnarray}
where $q=-1$,0, or 1, $C^k_q$ is the normalized spherical harmonics defined as
\begin{equation}
C^k_q = \left(\frac{4\pi}{2k+1}\right)^{1/2}Y_{kq}(\theta,\phi),
\end{equation}
and $Z^k_q(\alpha\beta)$ is the second quantized form of the angular
operator, and $\alpha$ and $\beta$ are the single-electron Dirac 
orbitals present
in the basis states. The reduced matrix elements of $C^k$ are given by
\begin{equation}
<\alpha||C^k||\beta> =
(-1)^{j_\alpha+1/2}\left[(2j_\alpha+1)(2j_\beta+1)\right]^{1/2}
\threej{j_\alpha}{k}{j_\beta}{\frac{1}{2}}{0}{-\frac{1}{2}},
\end{equation}
where \threej{a}{b}{c}{d}{e}{f} represents the Wigner $3j$ symbol.
This standard way of separating radial and angular
integration is used throughout FAC to compute matrix elements of
various operators, including, but not limited to, the Hamiltonian.

After some algebraic manipulation, the angular reduction of the $H_B^{(1)}$
term results in
\begin{eqnarray}
\label{eq:hb1}
H_B^{(1)} &=& \mu_B\sum_q (-1)^q B_q\sum_{\alpha\beta}
Z^1_{-q}(\alpha\beta)<\alpha||J^1 + S^1||\beta> \nonumber\\
&=& \mu_B\sum_q (-1)^q B_q\sum_{\alpha\beta}Z^1_{-q}(\alpha\beta)
\Bigg\{\delta_{j_\alpha
   j_\beta}\left[j_\alpha(j_\alpha+1)(2j_\alpha+1)\right]^{1/2} \nonumber\\
&+&
(-1)^{j_\alpha+l_\alpha-1/2}\left(\frac{3}{2}\right)^{1/2}\left[(2j_\alpha+1)(2j_\beta+1)\right]^{1/2}\sixj{l_\alpha}{\frac{1}{2}}{j_\alpha}{1}{j_\beta}{\frac{1}{2}}\Bigg\},
\end{eqnarray}
where $l_\alpha$, $j_\alpha$ are the orbital and total angular momentum of
the Dirac orbital $\alpha$, and \sixj{a}{b}{c}{d}{e}{f} represents the Wigner
$6j$ symbol, which comes as part of the reduced matrix elements of the spin
operator.

The reduction of the $H_{B}^{(2)}$ term is more complicated, as it involves
the cross product, and the quadratic term causes a rank-2 tensor to
appear in the expression. It can be shown that
\begin{eqnarray}
\label{eq:hb2}
H_B^{(2)} &=& \sqrt{3}\mu_B^2r^2\sum_{q_1p_1}
\threej{1}{1}{0}{q_1}{p_1}{0}B_{q_1}B_{p_1}\sixj{1}{1}{0}{1}{1}{1}\sum_{\alpha\beta}Z^0_0(\alpha\beta)<\alpha||C^0||\beta>
\nonumber \\
&-&\sqrt{30}\mu_B^2r^2\sum_{qq_1p1}\threej{1}{1}{2}{q_1}{p_1}{q}B_{q_1}B_{p_1}
\sixj{1}{1}{2}{1}{1}{1}
  \sum_{\alpha\beta}Z^2_q(\alpha\beta)<\alpha||C^2||\beta>.
\end{eqnarray}

Using Equations~\ref{eq:he}, \ref{eq:hb1} and \ref{eq:hb2}, the construction
of the Hamiltonian matrix becomes a simple matter of evaluating the angular
coefficients $<\phi_i|Z^k(\alpha\beta)|\phi_j>$ and the average values of the
radial operators $r$ and $r^2$. These quantities are already calculated in FAC
in the zero-field atomic structure theory, and existing code can be used.

After the Hamiltonian matrix is constructed, a standard linear 
algebra algorithm
is used to solve the eigenvalue problem to obtain the field-modified energy
levels and mixing coefficients, $b_i$, of the wavefunctions.

The calculation of radiative transition rates proceeds as in the zero-field
theory, except that the
wavefunctions now do not have a definite total angular momentum and
parity. For example, the line strength of E1 transitions can be calculated as
\begin{equation}
S_{fi} = \sum_M\left|\sum_{\mu\nu}b_{f\mu}b_{i\nu}\sum_{\alpha\beta}
<\phi_\mu||Z^1_M(\alpha,\beta)||\phi_\nu><\alpha||C^1||\beta>
M^1_{\alpha\beta}\right|^2 ,
\end{equation}
where $M^1_{\alpha\beta}$ is the radial part of the relativistic
single-electron E1 transition operator, as defined by \citet{grant74}. The
oscillator strength and transition rates are proportional to the line 
strengths.

\end{document}
