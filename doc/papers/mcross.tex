\documentclass[preprint,floatfix,pra,showpacs,showkeys]{revtex4}
\usepackage{graphicx}
\usepackage{dcolumn}

\newcommand{\threej}[6]{\ensuremath{\left({#1\atop #4}{#2\atop #5}
{#3\atop #6}\right)}}
\newcommand{\sixj}[6]{\ensuremath{\left\{{#1\atop #4}{#2\atop #5}
{#3\atop #6}\right\}}}

\begin{document}
\title{Electron Collisional Excitation Between Magnetic Sublevels: A
Factorization Interpolation Approach}
\author{Ming Feng Gu}
\affiliation{Center for Space Research, Massachusetts Institute of Technology,
Cambridge, MA 02139} 
\email[Email: ]{mfgu@space.mit.edu}

\begin{abstract}
It is shown that the factorization theory of distorted-wave electron
collisional excitation can be generalized to treat the excitation between
magnetic sublevels, which is needed to model the line polarization resulting
from the collisional excitation of electron beams. In this theory, the
collision strengths are factorized into two parts. A radial part involving
one-electron radial wavefunctions only, and an angular part involving the
angular coupling of the target states. Such factorization facilitates the
rapid evaluation of collision strengths between magnetic sublevels within an
entire transition array through the appropriate use of interpolation
procedures.
\end{abstract}

\pacs{34.80.Kw}
\keywords{Distorted-wave; Collision strength}
\maketitle

\section{Introduction}
Electron collisional excitation cross sections are required for calculation of
the level population and line emission of hot plasmas, such as those
encountered in astrophysical environments. In many cases, only the total
excitation cross sections are of importance, at least when the electron
distribution functions are isotropic. However, there exist situations where
aligned excitation produces polarized line emission. Suggestions have been
made to use such polarized light to study beam-plasma interactions in solar
flares \citep{haug72, haug81} and the properties of tokamak plasmas
\citep{fujimoto96}. Line polarization is also an important factor to take into
account in the analysis of laboratory spectroscopic data involving a
directional electron beam, such as electron beam ion traps
\citep{beiersdorfer96, takacs96}. 

In order to determine the degree of polarization and angular distribution of
emitted lines driven by electron collisional excitation, one needs detailed
cross sections between the magnetic sublevels of lower and upper
states. Several authors have made calculations of such cross sections
involving magnetic sublevels. \citet{mitroy88a} and \citet{mitroy88b} used a
nonrelativistic LS-coupling approach. \citet{inal87} made distorted-wave (DW)
calculations using nonrelativistic radial wavefunctions, but they included
intermediate-coupling effects by transforming the reactance matrices from
LS-coupling to relativistic pair-coupling scheme \citep{eissner72,
saraph78}. \citet{zhang90} performed the first fully relativistic DW
calculations of magnetic sublevel collision strengths with a modified version
of their previous program for total cross sections.

Since the pioneering work of \citet{bar-shalom88}, the factorization theory of
DW collisional excitation is in wide use for calculating total cross
sections. This theory enables one to obtain a complete collisional excitation
array without calculating a large number of radial integrals by using
appropriate interpolation procedures. However, DW calculations of magnetic
sublevel sollision strengths have not utilized such
factorization-interpolation techniques. In this paper, we generalize the
theory of \citet{bar-shalom88} to the excitation of magnetic sublevels and
present its computer implimentation.

\section{Factorization Theory}
The scattering amplitude $B_{m_{si}}^{m_{sf}}$ can be written as
(given in Zhang, Sampson, and Clark, 1990, PRA, 41, 198)
\begin{eqnarray}
B_{m_{si}}^{m_{sf}}&=&\frac{2\pi}{k_i}
  \sum_{l_i,m_{li},j_i,m_i \atop l_f,m_{lf},j_f,m_f}
  (i)^{l_i-l_f+1}\exp[i(\delta_i+\delta_f)]
  Y_{l_i}^{m_{li}*}(\hat{\textbf{k}}_i)
  Y_{l_f}^{m_{lf}}(\hat{\textbf{k}}_f) \nonumber\\
&&\times C(l_i\frac{1}{2}m_{li}m_{si};j_im_i)
  C(l_f\frac{1}{2}m_{lf}m_{sf};j_fm_f)
  T(\alpha_i,\alpha_f)
\end{eqnarray}
where $T(\alpha_i,\alpha_f)$ is the transition matrix elements in the 
representation where the free electrons are uncoupled to the targets,
\begin{equation}
\alpha_i=k_i\tilde{j_i}m_iJ_iM_i,\ \alpha_f=k_f\tilde{j_f}m_fJ_fM_f,
\end{equation}
where $\tilde{j}$ denotes $\{l,j\}$.
The differential cross section is 
\begin{equation}
\frac{d\sigma}{d\hat{k_f}}=\left|B_{m_{si}}^{m_{sf}}\right|^2. 
\end{equation}
Choosing the direction of the incident electron as the $z$ axis,
integrating over $\hat{k_f}$, summing over $m_{sf}$, and averaging over
$m_{si}$ gives
\begin{eqnarray}
\label{eq_cross0}
\sigma(J_fM_f,J_iM_i)&=&\frac{\pi}{2k_i^2}
  \sum_{\tilde{j_i},\tilde{j_i^\prime},\tilde{j_f} \atop m_{si},m_f}
  (i)^{l_i-l_i^\prime}\exp[i(\delta_i-\delta_{i^\prime})]
  ([l_i][l_i^\prime])^\frac{1}{2} \nonumber\\
&&\times
  (-1)^{j_i+j_i^\prime+2m_i}([j_i][j_i^\prime])^{\frac{1}{2}}
  \pmatrix{j_i & \frac{1}{2} & l_i \cr -m_i & m_{si} & 0 \cr}
  \pmatrix{j_i^\prime & \frac{1}{2} & l_i^\prime \cr -m_i & m_{si} & 0 \cr}
  \nonumber \\
&&\times
  T(\alpha_i,\alpha_f)T^{*}(\alpha_i^\prime,\alpha_f)
\end{eqnarray}
In the first order purterbation theory
\begin{equation}
T(\alpha_i,\alpha_f)=-2i<J_iM_i\tilde{j_i}m_i|V|J_fM_f\tilde{j_f}m_f>
\end{equation}
where $V$ is the coulomb interaction. In your derivation, you started off 
with the square of this matrix element, however, it is the two matrix elements 
involve different initial states that enter the expression for the cross 
sections. When both initial and final states
involve one free electron, it can be expanded as
\begin{equation}
V=2\sum_{t}\sum_{\tilde{j_i}j_0 \atop \tilde{j_f}j_1}\sum_{i\neq j}
  \left(Z^{t}(j_0,j_1)\cdot Z^{t}(\tilde{j_i},\tilde{j_f})\right)
  \Phi^{t}(j_0\tilde{j_i},j_1\tilde{j_f})
\end{equation}
\pagebreak
substitute into Eq. \ref{eq_cross0}
\begin{eqnarray}
\label{eq_cross1}
\sigma(J_fM_f,J_iM_i)&=&\frac{8\pi}{k_i^2}
  \sum_{\tilde{j_i},\tilde{j_i^\prime},\tilde{j_f} \atop m_{si},m_f}
  \sum_{j_0,j_1 \atop j_0^\prime,j_1^\prime}
  \sum_{t,t^\prime}
  (i)^{l_i-l_i^\prime}\exp[i(\delta_i-\delta_{i^\prime})]
  ([l_i][l_i^\prime])^\frac{1}{2} \nonumber \\
&&\times 
  (-1)^{j_i+j_i^\prime+2m_i}([j_i][j_i^\prime])^{\frac{1}{2}}
  \pmatrix{j_i & \frac{1}{2} & l_i \cr -m_i & m_{si} & 0 \cr}
  \pmatrix{j_i^\prime & \frac{1}{2} & l_i^\prime \cr -m_i & m_{si} & 0 \cr}
  \nonumber \\
&&\times \left<J_iM_i\tilde{j_i}m_i\left|
  Z^{t}(j_0j_1)\cdot Z^{t}(\tilde{j_i}\tilde{j_f})
  \right|J_fM_f\tilde{j_f}m_f\right> \nonumber \\
&&\times \left<J_iM_i\tilde{j_i^\prime}m_i\left|
  Z^{t^\prime}(j_0^\prime j_1^\prime)\cdot Z^{t^\prime}
  (\tilde{j_i^\prime}\tilde{j_f})
  \right|J_fM_f\tilde{j_f}m_f\right> \nonumber\\
&&\times \Phi^{t}(j_0\tilde{j_i},j_1\tilde{j_f})
  \Phi^{t^\prime}(j_0^\prime\tilde{j_i^\prime},j_1^\prime\tilde{j_f})
\end{eqnarray}
Expanding the scalar product in the spherical tensors
\begin{eqnarray}
\left<J_iM_i\tilde{j_i}m_i\left|
  Z^{t}(j_0j_1)\cdot Z^{t}(\tilde{j_i}\tilde{j_f})
  \right|J_fM_f\tilde{j_f}m_f\right> 
&=& \sum_{q}(-1)^{q}<\tilde{j_i}m_i|z^{t}_{-q}(\tilde{j_f}\tilde{j_f})
  |\tilde{j_i}m_i> \nonumber\\
& & \times <J_iM_i|Z^{t}_{q}(j_0j_1)|J_fM_f>  \nonumber\\
&=& (-1)^{q}(-1)^{j_i-m_i}
  \pmatrix{j_i & t & j_f \cr -m_i & -q & m_f \cr}
  \nonumber \\
& & \times (-1)^{J_i-M_i}
  \pmatrix{J_i & t & J_f \cr -M_i & q & M_f \cr}
  \nonumber \\
& & \times \left<J_iM_i\left|\left|Z^{t}(j_0j_1)\right|\right|J_fM_f\right>
\end{eqnarray}
The summation over $q$ may be dropped since it is fixed by the $3j$ 
symbols and is $M_i-M_f$, therefore
\begin{eqnarray}
\label{eq_cross2}
\sigma(J_fM_f,J_iM_i)&=&\frac{8\pi}{k_i^2}\sum_{tt^\prime}
  \pmatrix{J_i & t & J_f \cr -M_i & q & M_f \cr}
  \pmatrix{J_i & t^\prime & J_f \cr -M_i & q^\prime & M_f \cr}
  \nonumber \\
&&\times\sum_{j_0j_1 \atop j_0^\prime j_1^\prime}
  \left<J_iM_i\left|\left|Z^{t}(j_0j_1)\right|\right|J_fM_f\right>
  \left<J_iM_i\left|\left|Z^{t^\prime}(j_0^\prime j_1^\prime)
  \right|\right|J_fM_f\right> \nonumber\\
&&\times\sum_{\tilde{j_i},\tilde{j_i^\prime},\tilde{j_f} \atop m_{si},m_f}
  (i)^{l_i-l_i^\prime}\exp[i(\delta_i-\delta_{i^\prime})]
  ([l_i][l_i^\prime])^\frac{1}{2} \nonumber \\
&&\times
  ([j_i][j_i^\prime])^{\frac{1}{2}}
  \pmatrix{j_i & \frac{1}{2} & l_i \cr -m_i & m_{si} & 0 \cr}
  \pmatrix{j_i^\prime & \frac{1}{2} & l_i^\prime \cr -m_i & m_{si} & 0 \cr}
  \nonumber \\
&&\times
  \pmatrix{j_i & t & j_f \cr -m_i & -q & m_f \cr}
  \pmatrix{j_i^\prime & t^\prime & j_f \cr -m_i & -q^\prime & m_f \cr}
  \nonumber \\
&&\times \Phi^{t}(j_0\tilde{j_i},j_1\tilde{j_f})
  \Phi^{t^\prime}(j_0^\prime\tilde{j_i^\prime},j_1^\prime\tilde{j_f})
\end{eqnarray}
\pagebreak
or it may be written as
\begin{eqnarray}
\sigma(J_fM_f,J_iM_i)&=&\frac{8\pi}{k_i^2}\sum_{tt^\prime}
  \pmatrix{J_i & t & J_f \cr -M_i & q & M_f \cr}
  \pmatrix{J_i & t^\prime & J_f \cr -M_i & q^\prime & M_f \cr}
  \nonumber \\
&&\times\sum_{j_0j_1 \atop j_0^\prime j_1^\prime}
A^{t}(j_0j_1)A^{t^\prime}(j_0^\prime j_1^\prime)
Q^{tt^\prime}(j_0j_1,j_0^\prime j_1^\prime)
\end{eqnarray}
with
\begin{equation}
A^{t}(j_0j_1) =
\left<J_iM_i\left|\left|Z^{t}(j_0j_1)\right|\right|J_fM_f\right> 
\end{equation}
and
\begin{eqnarray}
Q^{tt^\prime}(j_0j_1,j_0^\prime j_1^\prime) &=&
  \sum_{\tilde{j_i},\tilde{j_i^\prime},\tilde{j_f} \atop m_{si},m_f}
  (i)^{l_i-l_i^\prime}\exp[i(\delta_i-\delta_{i^\prime})]
  ([l_i][l_i^\prime])^\frac{1}{2} \nonumber \\
&&\times
  ([j_i][j_i^\prime])^{\frac{1}{2}}
  \pmatrix{j_i & \frac{1}{2} & l_i \cr -m_i & m_{si} & 0 \cr}
  \pmatrix{j_i^\prime & \frac{1}{2} & l_i^\prime \cr -m_i & m_{si} & 0 \cr}
  \nonumber \\
&&\times 
  \pmatrix{j_i & t & j_f \cr -m_i & -q & m_f \cr}
  \pmatrix{j_i^\prime & t^\prime & j_f \cr -m_i & -q^\prime & m_f \cr}
  \nonumber \\
&&\times \Phi^{t}(j_0\tilde{j_i},j_1\tilde{j_f})
  \Phi^{t^\prime}(j_0^\prime\tilde{j_i^\prime},j_1^\prime\tilde{j_f})
\end{eqnarray}

In Eq. \ref{eq_cross2},
ony after summation over $M_f$ and averaging over $M_i$, the first two 
$3j$ symbols produces $\delta_{tt^\prime}\delta_{qq^\prime}$, then
the summation over $m_f$ and $q$ in the last two $3j$ symbols gives
$\delta_{j_i j_i^\prime}$, and finally, the summation over $m_{si}$ and
$m_i$ gives $\delta_{l_i l_i^\prime}$,
therefore cancels the phase factors, we recover the HULLAC factorization 
for the total cross sections.

\begin{acknowledgments}
This work is supported by NASA through Chandra Postdoctoral Fellowship Award
Number PF01-10014 issued by the Chandra X-ray Observatory Center, which is
operated by Smithsonian Astrophysical Observatory for and on behalf of NASA
under contract NAS8-39073.
\end{acknowledgments}

\end{document}
